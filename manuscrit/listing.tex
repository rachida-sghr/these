\usepackage{courier,listings}

\definecolor{superlightgray}{gray}{0.95}
\definecolor{lightgray}{gray}{0.75}
\definecolor{darkgray}{gray}{0.25}
\definecolor{superdarkgray}{gray}{0.25}

% Define minizinc language
\lstdefinelanguage{minizinc} { %
    morekeywords={set,of,min,max,in,int,var,show,show_int,constraint,include,forall,array},
    sensitive=true,
    morecomment=[l]{\%},
    morestring=[b]" % defines that strings are enclosed in double quotes
}
% general parameters
\lstset{ %
  backgroundcolor=\color{superlightgray},   % choose the background color; you must add \usepackage{color} or \usepackage{xcolor}
  basicstyle=\tiny\ttfamily,                % the size of the fonts that are used for the code
  breakatwhitespace=false,                  % sets if automatic breaks should only happen at whitespace
  breaklines=false,                         % sets automatic line breaking
  captionpos=b,                             % sets the caption-position to bottom
  commentstyle=\itshape\color{gray},        % comment style
  deletekeywords={...},                     % if you want to delete keywords from the given language
  escapeinside={\%*}{*)},                   % if you want to add LaTeX within your code
  extendedchars=true,                       % lets you use non-ASCII characters; for 8-bits encodings only, does not work with UTF-8
  frame=single,	                            % adds a frame around the code
  keepspaces=true,                          % keeps spaces in text, useful for keeping indentation of code (possibly needs columns=flexible)
  keywordstyle=\bfseries,                   % keyword style
  language=minizinc,                        % the language of the code
  % otherkeywords={},                       % if you want to add more keywords to the set
  numbers=left,                             % where to put the line-numbers; possible values are (none, left, right)
  numbersep=5pt,                            % how far the line-numbers are from the code
  numberstyle=\tiny\color{lightgray},       % the style that is used for the line-numbers
  rulecolor=\color{black},                  % if not set, the frame-color may be changed on line-breaks within not-black text (e.g. comments (green here))
  showspaces=false,                         % show spaces everywhere adding particular underscores; it overrides 'showstringspaces'
  showstringspaces=false,                   % underline spaces within strings only
  showtabs=false,                           % show tabs within strings adding particular underscores
  stepnumber=2,                             % the step between two line-numbers. If it's 1, each line will be numbered
  stringstyle=\color{darkgray},             % string literal style
  tabsize=2,	                            % sets default tabsize to 2 spaces
  % title=\lstname                          % show the filename of files included with \lstinputlisting; also try caption instead of title
}

\lstdefinestyle{xmlfig}{
  backgroundcolor=\color{superlightgray},
  stringstyle=\color{darkgray},
  basicstyle=\tiny\ttfamily,
  language=XML,
  frame=single,
  numbers=none,
  basicstyle=\tiny\ttfamily,
  keywordstyle=\bfseries,
}
