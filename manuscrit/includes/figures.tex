\usepackage{adjustbox}
\usepackage{arydshln,tabulary,multirow,booktabs,array,bigdelim}
\usepackage{graphicx}
\usepackage{pbox}
\usepackage{rotating}
\usepackage{tikz}
\usetikzlibrary{calc,mindmap,trees,fit,positioning,arrows}
\tikzset{concept/.append style={fill={none}}}

\newlength{\mytablewidth}
\newlength{\myfirstcolwidth}
\newlength{\mycolwidth}
\newlength{\mylastcolwidth}
\newcommand\mrows[1]{\multirow{#1}{\mycolwidth}{}}
% accolade verticale sur plusieurs lignes
% http://tex.stackexchange.com/a/218053/32098
\newcommand\multibrace[3]{\rdelim\}{#1}{3mm}[\pbox{#2}{#3}]}
% pour laisser une "marque" utilisable plus tard avec tikz
% utile pour tracer des fleches au dessus des tableaux par exemple
\newcommand\tikzmark[1]{\tikz[overlay,remember picture] \coordinate (#1);}
\newcommand\centbf[1]{\centering\textbf{#1}}
\newcommand\centit[1]{\centering\textit{#1}}

% checkmarks
% http://tex.stackexchange.com/a/132800/32098
\def\checkmark{\tikz\fill[scale=0.4](0,.35) -- (.25,0) -- (1,.7) -- (.25,.15) -- cycle;}
\def\scalecheck{\resizebox{\widthof{\checkmark}*\ratio{\widthof{x}}{\widthof{\normalsize x}}}{!}{\checkmark}}

% http://tex.stackexchange.com/questions/245825/how-to-draw-an-ekg-tracing-with-tikz/245839#245839
% http://tex.stackexchange.com/questions/173569/tikz-pic-parameter
% carte de france
% TODO: rendre les couleurs parametrables
\definecolor{cFFFFFF}{RGB}{255,255,255}
\definecolor{cE3ECF6}{RGB}{227,236,246}
\definecolor{cBCBD43}{RGB}{188,189,67}
\definecolor{c999999}{RGB}{153,153,153}
\definecolor{c666666}{RGB}{102,102,102} % frontieres et regions
\definecolor{cCCCCCC}{RGB}{204,204,204}
\tikzset{
    pics/france/.style args={scale #1}{
        code={
            \begin{scope}[y=0.80pt, x=0.80pt, yscale=-1.000000, xscale=1.000000, inner sep=0pt, outer sep=0pt, scale=#1]
                %\definecolor{cFFFFFF}{RGB}{255,255,255}
%\definecolor{cE3ECF6}{RGB}{227,236,246}
%\definecolor{cBCBD43}{RGB}{188,189,67}
%\definecolor{c999999}{RGB}{153,153,153}
%\definecolor{c666666}{RGB}{102,102,102} % frontieres et regions
%\definecolor{cCCCCCC}{RGB}{204,204,204}


%\begin{tikzpicture}[y=0.80pt, x=0.80pt, yscale=-1.000000, xscale=1.000000, inner sep=0pt, outer sep=0pt, scale=1]
\begin{scope}% departements
  \begin{scope}% g4612
    \begin{scope}% g4614
      \begin{scope}% g4616
        % path4618
        \path[draw,color=cFFFFFF,densely dotted] (442.2000,80.2000) -- (439.0000,73.7000) --
          (439.0000,69.1000) -- (441.9000,63.4000) -- (438.9000,61.6000) --
          (434.3000,66.5000) -- (432.4000,73.7000) -- (422.6000,78.1000) --
          (416.5000,76.1000) -- (413.3000,76.9000) -- (414.3000,83.3000) --
          (412.7000,87.2000) -- (414.3000,90.6000) -- (408.6000,98.7000) --
          (405.5000,99.1000) -- (405.9000,115.1000) -- (413.4000,116.2000) --
          (424.5000,123.6000) -- (447.0000,125.7000) -- (447.0000,125.6000) --
          (451.7000,121.5000) -- (450.2000,118.7000) -- (453.8000,113.0000) --
          (452.8000,103.2000) -- (455.0000,100.9000) -- (461.4000,103.2000) --
          (463.7000,100.2000) -- (465.7000,98.9000) -- (462.1000,99.2000) --
          (462.1000,95.6000) -- (455.3000,93.9000) -- (447.8000,87.3000) --
          (441.1000,87.5000) -- (442.2000,80.2000) -- cycle;

        % path4620
        \path[draw,color=cFFFFFF,densely dotted] (455.0000,100.9000) -- (452.8000,103.2000) --
          (453.8000,113.0000) -- (450.2000,118.7000) -- (451.7000,121.5000) --
          (447.0000,125.6000) -- (447.0000,125.7000) -- (449.5000,135.9000) --
          (448.4000,140.4000) -- (451.3000,141.9000) -- (450.6000,146.4000) --
          (447.2000,147.5000) -- (445.9000,154.1000) -- (449.5000,159.2000) --
          (450.0000,163.8000) -- (453.8000,169.6000) -- (472.8000,181.2000) --
          (480.6000,177.4000) -- (484.0000,178.0000) -- (486.2000,175.4000) --
          (486.4000,172.3000) -- (483.6000,170.0000) -- (485.4000,166.8000) --
          (484.0000,157.1000) -- (486.5000,141.7000) -- (487.5000,138.6000) --
          (482.0000,125.7000) -- (483.7000,119.1000) -- (482.3000,116.1000) --
          (477.4000,110.6000) -- (471.3000,112.9000) -- (469.9000,106.9000) --
          (469.6000,107.0000) -- (465.7000,98.9000) -- (463.7000,100.2000) --
          (461.4000,103.2000) -- (455.0000,100.9000) -- cycle;

        % path4622
        \path[draw,color=cFFFFFF,densely dotted] (469.9000,106.9000) -- (471.3000,112.9000) --
          (477.4000,110.6000) -- (482.3000,116.1000) -- (483.7000,119.1000) --
          (482.0000,125.7000) -- (487.5000,138.6000) -- (486.5000,141.7000) --
          (484.0000,157.1000) -- (485.4000,166.8000) -- (483.6000,170.0000) --
          (486.4000,172.3000) -- (486.2000,175.4000) -- (492.6000,176.1000) --
          (492.5000,179.4000) -- (497.1000,184.8000) -- (503.6000,184.0000) --
          (504.8000,180.9000) -- (510.8000,178.7000) -- (513.5000,180.6000) --
          (520.3000,179.5000) -- (526.2000,175.3000) -- (528.0000,178.2000) --
          (534.8000,180.1000) -- (546.7000,171.5000) -- (548.4000,171.0000) --
          (546.3000,169.6000) -- (544.0000,166.6000) -- (527.9000,160.5000) --
          (521.9000,156.3000) -- (512.4000,154.6000) -- (509.3000,148.8000) --
          (509.8000,145.6000) -- (496.9000,141.1000) -- (491.9000,132.7000) --
          (495.2000,132.5000) -- (494.1000,126.2000) -- (495.8000,122.8000) --
          (494.5000,118.8000) -- (491.7000,117.0000) -- (489.0000,106.4000) --
          (485.0000,103.0000) -- (477.3000,103.0000) -- (475.9000,106.1000) --
          (469.9000,106.9000) -- cycle;

        % path4624
        \path[draw,color=cFFFFFF,densely dotted] (489.0000,106.4000) -- (491.7000,117.0000) --
          (494.5000,118.8000) -- (495.8000,122.8000) -- (494.1000,126.2000) --
          (495.2000,132.5000) -- (491.9000,132.7000) -- (496.9000,141.1000) --
          (509.8000,145.6000) -- (509.3000,148.8000) -- (512.4000,154.6000) --
          (521.9000,156.3000) -- (527.9000,160.5000) -- (544.0000,166.6000) --
          (546.3000,169.6000) -- (550.2000,169.9000) -- (552.7000,168.0000) --
          (556.1000,153.0000) -- (551.1000,149.2000) -- (548.0000,148.8000) --
          (546.4000,151.6000) -- (543.8000,149.8000) -- (545.6000,147.0000) --
          (539.5000,144.2000) -- (545.0000,132.1000) -- (546.0000,135.2000) --
          (556.1000,140.4000) -- (566.2000,140.1000) -- (567.8000,137.2000) --
          (569.3000,132.5000) -- (569.1000,132.4000) -- (559.7000,124.0000) --
          (552.6000,128.6000) -- (544.8000,127.3000) -- (542.0000,128.7000) --
          (539.5000,124.3000) -- (536.1000,122.7000) -- (532.9000,124.5000) --
          (532.6000,127.7000) -- (529.5000,126.9000) -- (520.5000,116.7000) --
          (518.6000,110.1000) -- (515.3000,107.8000) -- (502.8000,104.9000) --
          (497.9000,108.9000) -- (494.7000,109.0000) -- (489.0000,106.4000) -- cycle;

        % path4626
        \path[draw,color=cFFFFFF,densely dotted] (492.6000,176.1000) -- (486.2000,175.4000) --
          (484.0000,178.0000) -- (480.6000,177.4000) -- (472.8000,181.2000) --
          (470.1000,183.9000) -- (484.5000,194.6000) -- (482.5000,204.2000) --
          (488.7000,209.4000) -- (488.9000,213.0000) -- (491.9000,211.8000) --
          (493.4000,214.6000) -- (493.4000,214.6000) -- (502.5000,208.2000) --
          (510.4000,213.0000) -- (516.8000,211.1000) -- (522.3000,215.2000) --
          (528.0000,212.7000) -- (537.4000,220.2000) -- (538.4000,219.4000) --
          (541.5000,216.8000) -- (541.6000,209.8000) -- (546.9000,201.1000) --
          (551.6000,185.8000) -- (552.8000,184.8000) -- (547.0000,182.3000) --
          (548.4000,171.0000) -- (546.7000,171.5000) -- (534.8000,180.1000) --
          (528.0000,178.2000) -- (526.2000,175.3000) -- (520.3000,179.5000) --
          (513.5000,180.6000) -- (510.8000,178.7000) -- (504.8000,180.9000) --
          (503.6000,184.0000) -- (497.1000,184.8000) -- (492.5000,179.4000) --
          (492.6000,176.1000) -- cycle;

        % path4628
        \path[draw,color=cFFFFFF,densely dotted] (450.6000,146.4000) -- (451.3000,141.9000) --
          (448.4000,140.4000) -- (449.5000,135.9000) -- (447.0000,125.7000) --
          (424.5000,123.6000) -- (413.4000,116.2000) -- (405.9000,115.1000) --
          (404.8000,118.3000) -- (398.8000,116.2000) -- (389.8000,120.7000) --
          (388.4000,123.7000) -- (392.0000,132.7000) -- (385.9000,135.2000) --
          (387.8000,138.0000) -- (386.8000,141.0000) -- (389.6000,142.4000) --
          (381.0000,153.9000) -- (380.5000,156.1000) -- (377.6000,158.6000) --
          (381.5000,168.0000) -- (384.4000,169.9000) -- (388.1000,174.9000) --
          (398.2000,176.2000) -- (398.2000,172.5000) -- (408.8000,164.0000) --
          (416.1000,163.0000) -- (419.6000,163.6000) -- (420.0000,170.4000) --
          (422.9000,172.7000) .. controls (425.1000,173.5000) and (427.1000,174.4000) ..
          (429.3000,175.2000) -- (432.1000,173.5000) -- (435.8000,174.8000) --
          (441.1000,174.8000) -- (440.3000,171.6000) -- (442.9000,169.0000) --
          (440.4000,166.3000) -- (450.0000,163.8000) -- (449.5000,159.2000) --
          (445.9000,154.1000) -- (447.2000,147.5000) -- (450.6000,146.4000) -- cycle;

        % path4630
        \path[draw,color=cFFFFFF,densely dotted] (435.8000,174.8000) -- (432.1000,173.5000) --
          (429.3000,175.2000) .. controls (427.1000,174.4000) and (425.1000,173.5000) ..
          (422.9000,172.7000) -- (420.0000,170.4000) -- (419.6000,163.6000) --
          (416.1000,163.0000) -- (408.8000,164.0000) -- (398.2000,172.5000) --
          (398.2000,176.2000) -- (388.1000,174.9000) -- (384.4000,169.9000) --
          (381.1000,175.4000) -- (377.8000,176.3000) -- (378.2000,185.9000) --
          (382.3000,187.5000) -- (387.3000,193.8000) -- (390.2000,203.1000) --
          (392.4000,200.6000) -- (396.4000,205.2000) -- (402.1000,217.5000) --
          (416.2000,215.7000) -- (419.5000,217.3000) -- (422.0000,215.1000) --
          (428.9000,214.0000) -- (434.1000,209.6000) -- (438.4000,210.2000) --
          (438.4000,210.2000) -- (438.3000,204.8000) -- (444.5000,202.0000) --
          (444.7000,198.9000) -- (444.1000,187.8000) -- (437.8000,184.5000) --
          (435.1000,178.6000) -- (435.8000,174.8000) -- cycle;

        % path4632
        \path[draw,color=cFFFFFF,densely dotted] (441.1000,174.8000) -- (435.8000,174.8000) --
          (435.1000,178.6000) -- (437.8000,184.5000) -- (444.1000,187.8000) --
          (444.7000,198.9000) -- (444.5000,202.0000) -- (438.3000,204.8000) --
          (438.4000,210.2000) -- (442.4000,211.0000) -- (443.0000,214.0000) --
          (445.9000,215.2000) -- (445.5000,218.4000) -- (448.8000,218.7000) --
          (450.8000,221.2000) -- (452.0000,224.7000) -- (448.8000,227.4000) --
          (451.4000,232.8000) -- (461.8000,235.0000) -- (465.1000,236.8000) --
          (465.1000,240.0000) -- (470.5000,237.8000) -- (471.9000,235.0000) --
          (479.2000,230.7000) -- (481.7000,232.7000) -- (485.2000,231.5000) --
          (485.5000,222.2000) -- (487.9000,219.4000) -- (491.1000,219.9000) --
          (493.6000,216.3000) -- (493.4000,214.6000) -- (493.4000,214.6000) --
          (491.9000,211.8000) -- (488.9000,213.0000) -- (488.7000,209.4000) --
          (482.5000,204.2000) -- (484.5000,194.6000) -- (470.1000,183.9000) --
          (472.8000,181.2000) -- (453.8000,169.6000) -- (450.0000,163.8000) --
          (440.4000,166.3000) -- (442.9000,169.0000) -- (440.3000,171.6000) --
          (441.1000,174.8000) -- cycle;

        % path4634
        \path[draw,color=cFFFFFF,densely dotted] (450.8000,221.2000) -- (448.8000,218.7000) --
          (445.5000,218.4000) -- (445.9000,215.2000) -- (443.0000,214.0000) --
          (442.4000,211.0000) -- (438.4000,210.2000) -- (438.4000,210.2000) --
          (434.1000,209.6000) -- (428.9000,214.0000) -- (422.0000,215.1000) --
          (419.5000,217.3000) -- (417.4000,220.7000) -- (420.9000,225.1000) --
          (420.5000,229.3000) -- (417.3000,231.7000) -- (419.3000,234.2000) --
          (411.8000,245.9000) -- (410.3000,255.0000) -- (411.7000,258.0000) --
          (411.6000,258.0000) -- (413.0000,264.8000) -- (417.2000,266.1000) --
          (416.8000,269.7000) -- (416.8000,270.7000) -- (419.5000,273.5000) --
          (425.9000,275.4000) -- (427.4000,278.4000) -- (433.6000,280.5000) --
          (439.8000,287.6000) -- (449.7000,282.8000) -- (466.6000,281.2000) --
          (466.6000,281.2000) -- (467.8000,276.6000) -- (474.8000,269.1000) --
          (476.8000,259.8000) -- (478.1000,258.3000) -- (473.7000,249.2000) --
          (470.8000,247.5000) -- (476.1000,243.0000) -- (475.9000,239.4000) --
          (473.6000,236.0000) -- (470.5000,237.8000) -- (465.1000,240.0000) --
          (465.1000,236.8000) -- (461.8000,235.0000) -- (451.4000,232.8000) --
          (448.8000,227.4000) -- (452.0000,224.7000) -- (450.8000,221.2000) -- cycle;

        % path4636
        \path[draw,color=cFFFFFF,densely dotted] (479.2000,230.7000) -- (471.9000,235.0000) --
          (470.5000,237.8000) -- (473.6000,236.0000) -- (475.9000,239.4000) --
          (476.1000,243.0000) -- (470.8000,247.5000) -- (473.7000,249.2000) --
          (478.1000,258.3000) -- (480.7000,260.7000) -- (486.8000,260.8000) --
          (511.9000,249.3000) -- (515.2000,244.0000) -- (521.5000,241.9000) --
          (524.6000,242.8000) -- (526.7000,240.4000) -- (530.3000,239.9000) --
          (530.8000,236.8000) -- (537.5000,237.6000) -- (537.6000,237.6000) --
          (534.8000,226.5000) -- (537.4000,220.2000) -- (528.0000,212.7000) --
          (522.3000,215.2000) -- (516.8000,211.1000) -- (510.4000,213.0000) --
          (502.5000,208.2000) -- (493.4000,214.6000) -- (493.6000,216.3000) --
          (491.1000,219.9000) -- (487.9000,219.4000) -- (485.5000,222.2000) --
          (485.2000,231.5000) -- (481.7000,232.7000) -- (479.2000,230.7000) -- cycle;

        % path4638
        \path[draw,color=cFFFFFF,densely dotted] (595.2000,136.4000) -- (585.6000,133.0000) --
          (569.3000,132.4000) -- (569.3000,132.5000) -- (567.8000,137.2000) --
          (566.2000,140.1000) -- (556.1000,140.4000) -- (546.0000,135.2000) --
          (545.0000,132.1000) -- (539.5000,144.2000) -- (545.6000,147.0000) --
          (543.8000,149.8000) -- (546.4000,151.6000) -- (548.0000,148.8000) --
          (551.1000,149.2000) -- (556.1000,153.0000) -- (552.7000,168.0000) --
          (550.2000,169.9000) -- (546.3000,169.6000) -- (548.4000,171.0000) --
          (547.0000,182.3000) -- (552.8000,184.8000) -- (556.3000,185.0000) --
          (565.8000,191.4000) -- (566.2000,194.4000) -- (570.8000,196.6000) --
          (577.4000,181.3000) -- (579.0000,163.7000) -- (590.8000,148.4000) --
          (595.2000,136.4000) -- cycle;

        % path4640
        \path[draw,color=cFFFFFF,densely dotted] (570.8000,196.6000) -- (566.2000,194.4000) --
          (565.8000,191.4000) -- (556.3000,185.0000) -- (552.8000,184.8000) --
          (551.6000,185.8000) -- (546.9000,201.1000) -- (541.6000,209.8000) --
          (541.5000,216.8000) -- (538.4000,219.4000) -- (547.0000,225.1000) --
          (547.5000,235.2000) -- (550.6000,235.0000) -- (553.0000,240.7000) --
          (552.6000,240.9000) -- (555.9000,241.6000) -- (555.0000,244.9000) --
          (563.6000,245.0000) -- (566.6000,243.9000) -- (567.1000,240.4000) --
          (570.4000,240.3000) -- (570.6000,236.8000) -- (573.1000,234.5000) --
          (569.9000,225.8000) -- (572.9000,208.5000) -- (570.7000,196.9000) --
          (570.8000,196.6000) -- cycle;

        % path4642
        \path[draw,color=cFFFFFF,densely dotted] (537.4000,220.2000) -- (534.8000,226.5000) --
          (537.6000,237.6000) -- (537.6000,237.6000) -- (544.4000,240.6000) --
          (542.9000,243.4000) -- (544.2000,245.9000) -- (548.3000,241.4000) --
          (548.6000,241.4000) -- (548.6000,241.2000) -- (553.0000,240.7000) --
          (550.6000,235.0000) -- (547.5000,235.2000) -- (547.0000,225.1000) --
          (538.4000,219.4000) -- (537.4000,220.2000) -- cycle;

        % path4644
        \path[draw,color=cFFFFFF,densely dotted] (542.9000,243.4000) -- (544.4000,240.6000) --
          (537.6000,237.6000) -- (537.6000,237.6000) -- (537.5000,237.6000) --
          (530.8000,236.8000) -- (530.3000,239.9000) -- (526.7000,240.4000) --
          (524.6000,242.8000) -- (521.5000,241.9000) -- (515.2000,244.0000) --
          (511.9000,249.3000) -- (486.8000,260.8000) -- (492.6000,269.4000) --
          (489.3000,275.2000) -- (490.6000,278.1000) -- (492.8000,275.6000) --
          (494.5000,278.5000) -- (498.0000,277.6000) -- (504.4000,288.9000) --
          (507.7000,289.5000) -- (512.4000,294.0000) -- (510.0000,296.6000) --
          (506.0000,305.0000) -- (509.4000,308.5000) -- (510.4000,305.0000) --
          (523.5000,292.9000) -- (522.6000,282.0000) -- (532.3000,275.4000) --
          (545.4000,259.1000) -- (545.2000,255.9000) -- (547.8000,253.6000) --
          (542.9000,249.5000) -- (544.0000,246.2000) -- (544.2000,245.9000) --
          (542.9000,243.4000) -- cycle;

        % path4646
        \path[draw,color=cFFFFFF,densely dotted] (540.6000,321.6000) -- (541.1000,318.4000) --
          (537.3000,317.7000) -- (527.9000,318.3000) -- (522.3000,323.0000) --
          (518.3000,321.4000) -- (515.2000,327.9000) -- (517.9000,331.4000) --
          (510.0000,337.4000) -- (502.5000,338.4000) -- (499.4000,341.7000) --
          (496.1000,341.7000) -- (497.0000,352.1000) -- (497.0000,352.2000) --
          (500.0000,359.7000) -- (503.1000,360.7000) -- (504.5000,363.5000) --
          (507.8000,365.4000) -- (510.8000,363.6000) -- (514.4000,364.6000) --
          (519.6000,368.6000) -- (530.1000,352.9000) -- (532.9000,354.7000) --
          (532.3000,358.1000) -- (534.8000,360.2000) -- (541.6000,362.7000) --
          (544.1000,361.0000) -- (544.1000,360.8000) -- (544.3000,360.6000) --
          (553.6000,351.6000) -- (554.6000,348.1000) -- (550.9000,342.9000) --
          (542.1000,334.6000) -- (544.7000,327.7000) -- (540.6000,321.6000) -- cycle;

        % path4648
        \path[draw,color=cFFFFFF,densely dotted] (532.9000,354.7000) -- (530.1000,352.9000) --
          (519.6000,368.6000) -- (514.4000,364.6000) -- (510.8000,363.6000) --
          (507.8000,365.4000) -- (504.5000,363.5000) -- (503.1000,360.7000) --
          (500.0000,359.7000) -- (497.0000,352.2000) -- (494.4000,367.4000) --
          (488.0000,374.9000) -- (495.3000,386.8000) -- (502.2000,387.0000) --
          (502.6000,383.8000) -- (513.2000,386.2000) -- (516.2000,393.3000) --
          (513.6000,399.4000) -- (514.8000,405.5000) -- (520.5000,407.2000) --
          (529.3000,411.1000) -- (531.5000,408.9000) -- (538.3000,407.6000) --
          (538.3000,407.5000) -- (560.0000,398.2000) -- (562.2000,394.1000) --
          (560.9000,391.1000) -- (564.1000,385.3000) -- (555.8000,379.6000) --
          (553.4000,373.4000) -- (554.1000,369.9000) -- (547.9000,367.4000) --
          (544.6000,364.4000) -- (544.1000,361.0000) -- (541.6000,362.7000) --
          (534.8000,360.2000) -- (532.3000,358.1000) -- (532.9000,354.7000) -- cycle;

        % path4650
        \path[draw,color=cFFFFFF,densely dotted] (486.8000,260.8000) -- (480.7000,260.7000) --
          (478.1000,258.3000) -- (476.8000,259.8000) -- (474.8000,269.1000) --
          (467.8000,276.6000) -- (466.6000,281.2000) -- (467.1000,284.6000) --
          (476.3000,289.5000) -- (470.7000,293.7000) -- (473.5000,295.5000) --
          (476.0000,305.2000) -- (472.8000,310.8000) -- (473.7000,316.8000) --
          (470.6000,318.0000) -- (475.4000,326.7000) -- (478.7000,326.7000) --
          (478.8000,330.1000) -- (482.1000,329.9000) -- (486.6000,325.0000) --
          (490.3000,326.8000) -- (490.8000,329.9000) -- (495.2000,329.7000) --
          (501.3000,326.1000) -- (506.6000,318.7000) -- (506.4000,318.6000) --
          (509.4000,308.5000) -- (506.0000,305.0000) -- (510.0000,296.6000) --
          (512.4000,294.0000) -- (507.7000,289.5000) -- (504.4000,288.9000) --
          (498.0000,277.6000) -- (494.5000,278.5000) -- (492.8000,275.6000) --
          (490.6000,278.1000) -- (489.3000,275.2000) -- (492.6000,269.4000) --
          (486.8000,260.8000) -- cycle;

        % path4652
        \path[draw,color=cFFFFFF,densely dotted] (449.7000,282.8000) -- (439.8000,287.6000) --
          (433.6000,280.5000) -- (427.4000,278.4000) -- (425.9000,275.4000) --
          (419.5000,273.5000) -- (416.8000,270.7000) -- (415.5000,271.0000) --
          (412.6000,272.9000) -- (409.4000,274.0000) -- (409.7000,280.6000) --
          (407.4000,282.6000) -- (411.9000,290.3000) -- (409.7000,292.8000) --
          (410.5000,295.9000) -- (397.8000,302.4000) -- (390.0000,299.2000) --
          (397.5000,314.4000) -- (406.3000,317.7000) -- (408.2000,320.4000) --
          (407.6000,326.5000) -- (403.5000,331.7000) -- (403.5000,335.1000) --
          (409.3000,338.7000) -- (418.2000,337.1000) -- (420.6000,339.5000) --
          (427.0000,335.1000) -- (427.3000,331.8000) -- (430.0000,329.6000) --
          (432.4000,331.7000) -- (434.9000,329.8000) -- (437.7000,331.5000) --
          (439.5000,328.9000) -- (445.9000,337.5000) -- (452.5000,314.1000) --
          (459.1000,315.7000) -- (465.2000,314.3000) -- (470.6000,318.0000) --
          (473.7000,316.8000) -- (472.8000,310.8000) -- (476.0000,305.2000) --
          (473.5000,295.5000) -- (470.7000,293.7000) -- (476.3000,289.5000) --
          (467.1000,284.6000) -- (466.6000,281.2000) -- (466.6000,281.2000) --
          (449.7000,282.8000) -- cycle;

        % path4654
        \path[draw,color=cFFFFFF,densely dotted] (470.6000,318.0000) -- (465.2000,314.3000) --
          (459.1000,315.7000) -- (452.5000,314.1000) -- (445.9000,337.5000) --
          (444.2000,346.5000) -- (447.7000,356.9000) -- (450.9000,356.3000) --
          (455.0000,362.7000) -- (462.2000,362.1000) -- (469.0000,364.4000) --
          (472.3000,358.6000) -- (475.5000,357.9000) -- (488.0000,374.9000) --
          (494.4000,367.4000) -- (497.0000,352.2000) -- (497.0000,352.1000) --
          (496.1000,341.7000) -- (499.4000,341.7000) -- (502.5000,338.4000) --
          (502.2000,338.5000) -- (502.2000,338.4000) -- (502.8000,332.8000) --
          (509.1000,328.8000) -- (510.5000,321.0000) -- (506.6000,318.7000) --
          (501.3000,326.1000) -- (495.2000,329.7000) -- (490.8000,329.9000) --
          (490.3000,326.8000) -- (486.6000,325.0000) -- (482.1000,329.9000) --
          (478.8000,330.1000) -- (478.7000,326.7000) -- (475.4000,326.7000) --
          (470.6000,318.0000) -- cycle;

        % path4656
        \path[draw,color=cFFFFFF,densely dotted] (450.9000,356.3000) -- (447.7000,356.9000) --
          (444.2000,346.5000) -- (445.9000,337.5000) -- (439.5000,328.9000) --
          (437.7000,331.5000) -- (434.9000,329.8000) -- (432.4000,331.7000) --
          (430.0000,329.6000) -- (427.3000,331.8000) -- (427.0000,335.1000) --
          (427.0000,335.1000) -- (429.3000,338.5000) -- (423.9000,342.3000) --
          (422.1000,352.0000) -- (427.9000,359.6000) -- (426.0000,366.1000) --
          (428.5000,368.0000) -- (427.1000,370.8000) -- (428.4000,374.1000) --
          (433.6000,378.8000) -- (440.2000,381.2000) -- (441.2000,384.6000) --
          (446.3000,387.4000) -- (451.3000,383.0000) -- (449.2000,380.2000) --
          (452.0000,377.4000) -- (459.7000,375.7000) -- (462.6000,370.1000) --
          (462.2000,362.1000) -- (455.0000,362.7000) -- (450.9000,356.3000) -- cycle;

        % path4658
        \path[draw,color=cFFFFFF,densely dotted] (495.3000,386.8000) -- (488.0000,374.9000) --
          (475.5000,357.9000) -- (472.3000,358.6000) -- (469.0000,364.4000) --
          (462.2000,362.1000) -- (462.6000,370.1000) -- (459.7000,375.7000) --
          (452.0000,377.4000) -- (449.2000,380.2000) -- (451.3000,383.0000) --
          (446.3000,387.4000) -- (446.3000,387.4000) -- (446.5000,388.6000) --
          (446.4000,393.6000) -- (448.8000,398.3000) -- (458.8000,394.9000) --
          (464.3000,397.8000) -- (468.0000,406.7000) -- (466.8000,413.7000) --
          (469.8000,412.6000) -- (479.0000,414.7000) -- (482.4000,412.8000) --
          (483.4000,432.5000) -- (486.7000,431.9000) -- (492.8000,436.3000) --
          (499.4000,437.3000) -- (500.5000,433.8000) -- (506.6000,432.4000) --
          (507.7000,429.3000) -- (521.8000,425.0000) -- (525.3000,426.2000) --
          (525.7000,419.6000) -- (520.8000,416.1000) -- (518.8000,411.6000) --
          (520.5000,407.2000) -- (514.8000,405.5000) -- (513.6000,399.4000) --
          (516.2000,393.3000) -- (513.2000,386.2000) -- (502.6000,383.8000) --
          (502.2000,387.0000) -- (495.3000,386.8000) -- cycle;

        % path4660
        \path[draw,color=cFFFFFF,densely dotted] (529.3000,411.1000) -- (520.5000,407.2000) --
          (518.8000,411.6000) -- (520.8000,416.1000) -- (525.7000,419.6000) --
          (525.3000,426.2000) -- (521.8000,425.0000) -- (507.7000,429.3000) --
          (506.6000,432.4000) -- (500.5000,433.8000) -- (499.4000,437.3000) --
          (499.0000,441.0000) -- (492.5000,441.4000) -- (489.7000,449.0000) --
          (493.1000,451.5000) -- (490.3000,454.4000) -- (485.6000,452.8000) --
          (482.0000,458.4000) -- (483.7000,461.5000) -- (494.9000,467.8000) --
          (494.7000,473.1000) -- (501.4000,471.5000) -- (505.6000,472.7000) --
          (501.7000,467.2000) -- (506.1000,468.5000) -- (507.7000,458.8000) --
          (513.6000,454.3000) -- (517.4000,453.4000) -- (521.3000,458.5000) --
          (522.1000,454.2000) -- (526.3000,449.5000) -- (538.7000,453.3000) --
          (548.9000,441.8000) -- (555.6000,438.1000) -- (555.7000,437.9000) --
          (555.8000,437.8000) -- (557.9000,435.4000) -- (561.9000,435.5000) --
          (557.6000,425.0000) -- (550.8000,424.6000) -- (545.0000,420.9000) --
          (544.0000,413.6000) -- (540.9000,413.4000) -- (538.4000,407.6000) --
          (538.3000,407.6000) -- (531.5000,408.9000) -- (529.3000,411.1000) -- cycle;

        % path4662
        \path[draw,color=cFFFFFF,densely dotted] (440.2000,381.2000) -- (433.6000,378.8000) --
          (428.4000,374.1000) -- (427.1000,370.8000) -- (428.5000,368.0000) --
          (426.0000,366.1000) -- (427.9000,359.6000) -- (422.1000,352.0000) --
          (423.9000,342.3000) -- (429.3000,338.5000) -- (427.0000,335.1000) --
          (427.0000,335.1000) -- (420.6000,339.5000) -- (418.2000,337.1000) --
          (409.3000,338.7000) -- (403.5000,335.1000) -- (403.5000,331.7000) --
          (399.1000,333.1000) -- (398.5000,336.5000) -- (400.2000,350.5000) --
          (394.1000,355.8000) -- (396.9000,358.9000) -- (394.5000,366.0000) --
          (399.7000,374.6000) -- (404.9000,378.8000) -- (407.9000,387.3000) --
          (404.6000,395.3000) -- (407.7000,394.3000) -- (410.2000,396.6000) --
          (413.4000,396.2000) -- (420.4000,392.8000) -- (427.1000,395.3000) --
          (427.9000,401.5000) -- (430.9000,400.4000) -- (433.4000,403.1000) --
          (437.2000,402.1000) -- (440.9000,396.7000) -- (446.4000,393.6000) --
          (446.5000,388.6000) -- (446.3000,387.4000) -- (446.3000,387.4000) --
          (441.2000,384.6000) -- (440.2000,381.2000) -- cycle;

        % path4664
        \path[draw,color=cFFFFFF,densely dotted] (448.8000,398.3000) -- (446.4000,393.6000) --
          (440.9000,396.7000) -- (437.2000,402.1000) -- (433.4000,403.1000) --
          (432.3000,407.7000) -- (432.3000,411.1000) -- (429.4000,409.8000) --
          (427.8000,415.7000) -- (424.8000,418.2000) -- (425.6000,421.4000) --
          (421.7000,421.8000) -- (417.9000,428.5000) -- (411.3000,430.5000) --
          (403.8000,437.8000) -- (406.1000,447.9000) -- (410.8000,457.2000) --
          (413.9000,463.9000) -- (422.7000,469.6000) -- (430.6000,464.6000) --
          (430.7000,468.2000) -- (433.6000,465.7000) -- (436.7000,465.7000) --
          (443.4000,469.5000) -- (443.3000,465.3000) -- (445.6000,449.2000) --
          (449.0000,442.6000) -- (448.0000,434.6000) -- (451.9000,429.5000) --
          (453.5000,422.6000) -- (449.3000,409.9000) -- (448.8000,398.3000) -- cycle;

        % path4666
        \path[draw,color=cFFFFFF,densely dotted] (432.3000,411.1000) -- (432.3000,407.7000) --
          (433.4000,403.1000) -- (430.9000,400.4000) -- (427.9000,401.5000) --
          (427.1000,395.3000) -- (420.4000,392.8000) -- (413.4000,396.2000) --
          (410.2000,396.6000) -- (407.7000,394.3000) -- (404.6000,395.3000) --
          (392.1000,396.1000) -- (386.2000,392.5000) -- (376.7000,392.4000) --
          (365.9000,396.0000) -- (366.0000,400.5000) -- (372.1000,402.1000) --
          (375.8000,411.9000) -- (378.7000,413.2000) -- (375.0000,413.4000) --
          (378.9000,422.4000) -- (381.4000,424.3000) -- (383.1000,430.7000) --
          (390.3000,432.1000) -- (390.5000,428.8000) -- (393.6000,429.1000) --
          (403.8000,437.8000) -- (411.3000,430.5000) -- (417.9000,428.5000) --
          (421.7000,421.8000) -- (425.6000,421.4000) -- (424.8000,418.2000) --
          (427.8000,415.7000) -- (429.4000,409.8000) -- (432.3000,411.1000) -- cycle;

        % path4668
        \path[draw,color=cFFFFFF,densely dotted] (448.8000,398.3000) -- (449.3000,409.9000) --
          (453.5000,422.6000) -- (451.9000,429.5000) -- (448.0000,434.6000) --
          (449.0000,442.6000) -- (445.6000,449.2000) -- (443.3000,465.3000) --
          (449.9000,466.1000) -- (453.8000,470.9000) -- (463.6000,466.5000) --
          (466.3000,468.1000) -- (469.1000,465.7000) -- (469.7000,472.1000) --
          (480.2000,473.0000) -- (480.3000,476.3000) -- (486.1000,478.8000) --
          (490.9000,473.5000) -- (493.6000,473.4000) -- (494.7000,473.1000) --
          (494.9000,467.8000) -- (483.7000,461.5000) -- (482.0000,458.4000) --
          (485.6000,452.8000) -- (490.3000,454.4000) -- (493.1000,451.5000) --
          (489.7000,449.0000) -- (492.5000,441.4000) -- (499.0000,441.0000) --
          (499.4000,437.3000) -- (492.8000,436.3000) -- (486.7000,431.9000) --
          (483.4000,432.5000) -- (482.4000,412.8000) -- (479.0000,414.7000) --
          (469.8000,412.6000) -- (466.8000,413.7000) -- (468.0000,406.7000) --
          (464.3000,397.8000) -- (458.8000,394.9000) -- (448.8000,398.3000) --
          cycle(456.1000,460.5000) -- (459.1000,458.0000) -- (464.2000,461.1000) --
          (461.5000,465.1000) -- (457.0000,466.4000) -- (454.7000,463.9000) --
          (456.1000,460.5000) -- cycle;

        % path4670
        \path[draw,color=cFFFFFF,densely dotted] (459.1000,458.0000) -- (456.1000,460.5000) --
          (454.7000,463.9000) -- (457.0000,466.4000) -- (461.5000,465.1000) --
          (464.2000,461.1000) -- (459.1000,458.0000) -- cycle;

        % path4672
        \path[draw,color=cFFFFFF,densely dotted] (443.4000,469.5000) -- (446.4000,474.5000) --
          (446.6000,480.7000) -- (453.7000,487.9000) -- (448.6000,493.4000) --
          (460.3000,496.7000) -- (466.4000,503.1000) -- (471.5000,505.7000) --
          (475.4000,505.0000) -- (488.9000,510.3000) -- (492.5000,510.0000) --
          (500.0000,505.3000) -- (500.2000,504.9000) -- (493.3000,498.1000) --
          (489.5000,497.0000) -- (491.4000,494.2000) -- (487.2000,488.3000) --
          (488.6000,484.9000) -- (486.1000,478.8000) -- (480.3000,476.3000) --
          (480.2000,473.0000) -- (469.7000,472.1000) -- (469.1000,465.7000) --
          (466.3000,468.1000) -- (463.6000,466.5000) -- (453.8000,470.9000) --
          (449.9000,466.1000) -- (443.3000,465.3000) -- (443.4000,469.5000) -- cycle;

        % path4674
        \path[draw,color=cFFFFFF,densely dotted] (446.4000,474.5000) -- (443.4000,469.5000) --
          (436.7000,465.7000) -- (433.6000,465.7000) -- (430.7000,468.2000) --
          (430.6000,464.6000) -- (422.7000,469.6000) -- (413.9000,463.9000) --
          (410.8000,457.2000) -- (409.6000,459.5000) -- (405.5000,462.5000) --
          (410.0000,471.1000) -- (408.6000,474.4000) -- (410.2000,477.7000) --
          (403.2000,479.8000) -- (393.4000,477.2000) -- (393.2000,481.0000) --
          (388.9000,481.6000) -- (380.1000,477.7000) -- (377.8000,482.1000) --
          (374.7000,483.4000) -- (384.2000,488.0000) -- (379.3000,493.9000) --
          (379.7000,495.5000) -- (379.7000,495.6000) -- (382.9000,495.7000) --
          (383.3000,498.9000) -- (385.8000,497.1000) -- (391.7000,499.9000) --
          (391.9000,496.7000) -- (396.6000,492.3000) -- (401.0000,491.6000) --
          (403.1000,494.6000) -- (402.1000,497.6000) -- (409.7000,499.2000) --
          (409.9000,502.5000) -- (413.3000,502.8000) -- (420.2000,510.3000) --
          (421.8000,513.6000) -- (417.2000,519.8000) -- (417.2000,519.9000) --
          (420.4000,525.3000) -- (423.7000,526.1000) -- (424.0000,526.1000) --
          (428.3000,519.4000) -- (434.8000,517.4000) -- (433.6000,514.4000) --
          (438.4000,509.0000) -- (442.0000,510.0000) -- (444.2000,498.1000) --
          (448.6000,493.4000) -- (453.7000,487.9000) -- (446.6000,480.7000) --
          (446.4000,474.5000) -- cycle;

        % path4676
        \path[draw,color=cFFFFFF,densely dotted] (501.4000,471.5000) -- (494.7000,473.1000) --
          (493.6000,473.4000) -- (490.9000,473.5000) -- (486.1000,478.8000) --
          (488.6000,484.9000) -- (487.2000,488.3000) -- (491.4000,494.2000) --
          (489.5000,497.0000) -- (493.3000,498.1000) -- (500.2000,504.9000) --
          (506.1000,505.1000) -- (509.1000,503.1000) -- (513.6000,508.6000) --
          (525.0000,499.0000) -- (531.6000,503.0000) -- (533.0000,499.1000) --
          (543.9000,498.7000) -- (547.3000,497.0000) -- (545.9000,494.1000) --
          (551.7000,489.8000) -- (557.6000,490.6000) -- (549.8000,481.9000) --
          (545.0000,472.3000) -- (547.2000,466.3000) -- (554.1000,458.6000) --
          (553.9000,458.4000) -- (556.0000,452.9000) -- (551.6000,446.8000) --
          (554.1000,444.7000) -- (555.6000,438.1000) -- (548.9000,441.8000) --
          (538.7000,453.3000) -- (526.3000,449.5000) -- (522.1000,454.2000) --
          (521.3000,458.5000) -- (517.4000,453.4000) -- (513.6000,454.3000) --
          (507.7000,458.8000) -- (506.1000,468.5000) -- (501.7000,467.2000) --
          (505.6000,472.7000) -- (501.4000,471.5000) -- cycle;

        % path4678
        \path[draw,color=cFFFFFF,densely dotted] (554.1000,458.6000) -- (547.2000,466.3000) --
          (545.0000,472.3000) -- (549.8000,481.9000) -- (557.6000,490.6000) --
          (551.7000,489.8000) -- (545.9000,494.1000) -- (547.3000,497.0000) --
          (543.9000,498.7000) -- (550.1000,501.9000) -- (552.0000,508.1000) --
          (558.3000,511.1000) -- (557.4000,514.3000) -- (559.9000,518.8000) --
          (562.5000,514.2000) -- (569.0000,511.1000) -- (570.1000,506.5000) --
          (586.4000,497.6000) -- (586.9000,489.5000) -- (593.9000,479.8000) --
          (596.1000,474.9000) -- (591.6000,469.0000) -- (578.3000,473.8000) --
          (568.0000,468.3000) -- (561.0000,466.9000) -- (554.1000,458.6000) -- cycle;

        % path4680
        \path[draw,color=cFFFFFF,densely dotted] (403.8000,50.4000) -- (397.0000,51.7000) --
          (390.8000,50.2000) -- (388.4000,52.4000) -- (385.9000,42.2000) --
          (379.8000,40.8000) -- (378.6000,37.7000) -- (375.3000,40.2000) --
          (371.8000,38.9000) -- (368.3000,29.4000) -- (368.8000,26.1000) --
          (364.5000,21.1000) -- (357.9000,21.7000) -- (351.7000,26.0000) --
          (348.3000,24.5000) -- (345.1000,18.7000) -- (341.4000,18.7000) --
          (340.0000,15.7000) -- (341.1000,9.2000) -- (338.3000,2.5000) --
          (335.1000,0.6000) -- (316.9000,5.1000) -- (324.4000,19.5000) --
          (330.6000,22.4000) -- (329.8000,25.6000) -- (334.4000,30.3000) --
          (347.5000,30.8000) -- (349.4000,28.3000) -- (350.8000,31.2000) --
          (349.7000,38.0000) -- (356.3000,39.1000) -- (360.7000,45.0000) --
          (357.2000,46.8000) -- (361.5000,52.7000) -- (360.8000,55.8000) --
          (364.1000,56.2000) -- (366.2000,58.8000) -- (362.2000,70.9000) --
          (365.9000,73.7000) -- (380.0000,73.1000) -- (382.9000,70.6000) --
          (385.6000,72.5000) -- (391.4000,70.5000) -- (409.0000,75.2000) --
          (409.0000,75.3000) -- (412.6000,68.5000) -- (409.2000,62.5000) --
          (411.5000,55.3000) -- (408.3000,56.0000) -- (403.8000,50.4000) -- cycle;

        % path4682
        \path[draw,color=cFFFFFF,densely dotted] (347.5000,30.8000) -- (334.4000,30.3000) --
          (329.8000,25.6000) -- (330.6000,22.4000) -- (324.4000,19.5000) --
          (316.9000,5.1000) -- (301.5000,10.0000) -- (295.0000,15.6000) --
          (294.9000,35.8000) -- (298.2000,39.7000) -- (295.1000,37.7000) --
          (294.4000,44.4000) -- (294.9000,48.6000) -- (297.4000,50.5000) --
          (311.2000,51.7000) -- (312.2000,55.1000) -- (320.3000,61.5000) --
          (333.7000,58.7000) -- (330.8000,65.1000) -- (332.7000,68.7000) --
          (335.5000,65.2000) -- (344.9000,69.1000) -- (345.8000,66.0000) --
          (349.0000,71.2000) -- (351.6000,69.2000) -- (352.6000,72.9000) --
          (362.2000,70.9000) -- (366.2000,58.8000) -- (364.1000,56.2000) --
          (360.8000,55.8000) -- (361.5000,52.7000) -- (357.2000,46.8000) --
          (360.7000,45.0000) -- (356.3000,39.1000) -- (349.7000,38.0000) --
          (350.8000,31.2000) -- (349.4000,28.3000) -- (347.5000,30.8000) --
          cycle(358.2000,66.4000) -- (362.1000,62.7000) -- (361.3000,65.6000) --
          (358.2000,66.4000) -- cycle;

        % path4684
        \path[draw,color=cFFFFFF,densely dotted] (365.9000,73.7000) -- (365.4000,77.9000) --
          (361.1000,87.0000) -- (363.7000,94.9000) -- (364.1000,106.4000) --
          (363.2000,111.4000) -- (365.7000,113.4000) -- (362.8000,114.7000) --
          (360.1000,121.1000) -- (357.2000,122.9000) -- (360.4000,128.1000) --
          (357.0000,127.9000) -- (357.9000,131.1000) -- (360.5000,129.4000) --
          (363.8000,130.6000) -- (364.4000,133.9000) -- (361.9000,135.7000) --
          (364.9000,136.5000) -- (366.9000,139.3000) -- (366.2000,142.5000) --
          (370.6000,147.1000) -- (381.0000,153.9000) -- (389.6000,142.4000) --
          (386.8000,141.0000) -- (387.8000,138.0000) -- (385.9000,135.2000) --
          (392.0000,132.7000) -- (388.4000,123.7000) -- (389.8000,120.7000) --
          (398.8000,116.2000) -- (404.8000,118.3000) -- (405.9000,115.1000) --
          (405.5000,99.1000) -- (408.6000,98.7000) -- (414.3000,90.6000) --
          (412.7000,87.2000) -- (414.3000,83.3000) -- (413.3000,76.9000) --
          (413.0000,76.9000) -- (408.8000,75.6000) -- (409.0000,75.2000) --
          (391.4000,70.5000) -- (385.6000,72.5000) -- (382.9000,70.6000) --
          (380.0000,73.1000) -- (365.9000,73.7000) -- cycle;

        % path4686
        \path[draw,color=cFFFFFF,densely dotted] (365.4000,77.9000) -- (365.9000,73.7000) --
          (362.2000,70.9000) -- (352.6000,72.9000) -- (351.6000,69.2000) --
          (349.0000,71.2000) -- (345.8000,66.0000) -- (344.9000,69.1000) --
          (335.5000,65.2000) -- (332.7000,68.7000) -- (330.8000,65.1000) --
          (333.7000,58.7000) -- (320.3000,61.5000) -- (312.2000,55.1000) --
          (311.2000,51.7000) -- (297.4000,50.5000) -- (293.3000,52.1000) --
          (292.9000,55.4000) -- (296.3000,61.6000) -- (293.1000,60.0000) --
          (285.8000,69.9000) -- (285.9000,70.0000) -- (299.9000,82.0000) --
          (303.8000,91.4000) -- (303.8000,91.5000) -- (308.4000,95.6000) --
          (312.0000,94.4000) -- (328.1000,96.7000) -- (342.5000,104.4000) --
          (352.1000,96.6000) -- (363.7000,94.9000) -- (361.1000,87.0000) --
          (365.4000,77.9000) -- cycle;

        % path4688
        \path[draw,color=cFFFFFF,densely dotted] (352.1000,96.6000) -- (342.5000,104.4000) --
          (328.1000,96.7000) -- (312.0000,94.4000) -- (308.4000,95.6000) --
          (303.8000,91.5000) -- (303.8000,91.4000) -- (300.4000,96.7000) --
          (301.9000,106.5000) -- (300.4000,115.5000) -- (304.3000,124.9000) --
          (299.9000,127.9000) -- (301.6000,131.5000) -- (307.9000,132.7000) --
          (315.0000,130.6000) -- (325.3000,133.1000) -- (327.9000,131.3000) --
          (340.0000,138.5000) -- (347.8000,139.6000) -- (350.0000,137.2000) --
          (352.9000,139.0000) -- (361.9000,135.7000) -- (364.4000,133.9000) --
          (363.8000,130.6000) -- (360.5000,129.4000) -- (357.9000,131.1000) --
          (357.0000,127.9000) -- (360.4000,128.1000) -- (357.2000,122.9000) --
          (360.1000,121.1000) -- (362.8000,114.7000) -- (365.7000,113.4000) --
          (363.2000,111.4000) -- (364.1000,106.4000) -- (363.7000,94.9000) --
          (352.1000,96.6000) -- cycle;

        % path4690
        \path[draw,color=cFFFFFF,densely dotted] (236.4000,113.3000) -- (236.3000,113.3000) --
          (228.6000,115.0000) -- (223.1000,119.8000) -- (212.3000,122.9000) --
          (202.8000,118.9000) -- (188.8000,117.4000) -- (178.1000,113.5000) --
          (172.3000,116.1000) -- (172.3000,116.2000) -- (173.0000,123.6000) --
          (182.4000,127.1000) -- (180.2000,130.0000) -- (182.7000,139.2000) --
          (173.6000,144.7000) -- (175.3000,147.3000) -- (170.3000,153.2000) --
          (172.3000,156.0000) -- (183.2000,158.3000) -- (189.2000,155.0000) --
          (190.2000,152.1000) -- (193.3000,153.5000) -- (200.4000,150.9000) --
          (206.0000,154.3000) -- (208.8000,152.3000) -- (215.3000,153.6000) --
          (218.6000,153.2000) -- (226.6000,147.1000) -- (234.5000,145.3000) --
          (240.8000,146.4000) -- (241.6000,143.9000) -- (238.9000,139.3000) --
          (240.5000,135.9000) -- (237.5000,125.4000) -- (240.0000,123.1000) --
          (236.8000,123.1000) -- (236.4000,113.3000) -- cycle;

        % path4692
        \path[draw,color=cFFFFFF,densely dotted] (236.4000,113.2000) -- (236.4000,113.3000) --
          (236.8000,123.1000) -- (240.0000,123.1000) -- (237.5000,125.4000) --
          (240.5000,135.9000) -- (238.9000,139.3000) -- (241.6000,143.9000) --
          (240.8000,146.4000) -- (241.0000,150.0000) -- (248.1000,150.6000) --
          (256.9000,161.2000) -- (254.9000,163.8000) -- (258.8000,166.1000) --
          (261.4000,163.2000) -- (268.2000,162.0000) -- (274.1000,158.4000) --
          (277.3000,160.2000) -- (284.1000,158.7000) -- (283.8000,155.6000) --
          (290.5000,147.9000) -- (288.6000,141.4000) -- (291.1000,138.4000) --
          (295.4000,138.5000) -- (299.9000,127.9000) -- (304.3000,124.9000) --
          (300.4000,115.5000) -- (285.7000,111.7000) -- (280.3000,119.7000) --
          (271.1000,122.5000) -- (269.5000,125.4000) -- (263.5000,122.8000) --
          (261.2000,120.4000) -- (263.2000,117.7000) -- (257.7000,114.4000) --
          (253.9000,115.2000) -- (245.4000,109.6000) -- (238.0000,112.5000) --
          (238.2000,112.6000) -- (236.4000,113.2000) -- cycle;

        % path4694
        \path[draw,color=cFFFFFF,densely dotted] (238.0000,112.5000) -- (245.4000,109.6000) --
          (253.9000,115.2000) -- (257.7000,114.4000) -- (263.2000,117.7000) --
          (261.2000,120.4000) -- (263.5000,122.8000) -- (269.5000,125.4000) --
          (271.1000,122.5000) -- (280.3000,119.7000) -- (285.7000,111.7000) --
          (300.4000,115.5000) -- (301.9000,106.5000) -- (300.4000,96.7000) --
          (303.8000,91.4000) -- (299.9000,82.0000) -- (285.9000,70.0000) --
          (285.6000,70.3000) -- (273.4000,78.8000) -- (248.6000,84.9000) --
          (232.7000,93.6000) -- (226.2000,107.7000) -- (228.1000,110.2000) --
          (238.0000,112.5000) -- cycle;

        % path4696
        \path[draw,color=cFFFFFF,densely dotted] (327.9000,131.3000) -- (325.3000,133.1000) --
          (315.0000,130.6000) -- (307.9000,132.7000) -- (301.6000,131.5000) --
          (299.9000,127.9000) -- (295.4000,138.5000) -- (306.0000,139.4000) --
          (316.2000,143.5000) -- (322.3000,150.4000) -- (322.3000,150.3000) --
          (322.4000,150.3000) -- (326.4000,147.4000) -- (330.1000,145.9000) --
          (333.0000,147.1000) .. controls (334.0000,146.4000) and (335.0000,145.7000) ..
          (336.0000,145.0000) .. controls (336.7000,144.5000) and (337.5000,143.9000) ..
          (338.3000,143.4000) -- (338.4000,143.4000) -- (340.3000,140.8000) --
          (340.0000,138.5000) -- (327.9000,131.3000) -- cycle;

        % path4698
        \path[draw,color=cFFFFFF,densely dotted] (291.1000,138.4000) -- (288.6000,141.4000) --
          (290.5000,147.9000) -- (294.6000,155.3000) -- (295.2000,167.5000) --
          (302.1000,173.5000) -- (303.5000,179.8000) -- (306.0000,182.3000) --
          (309.5000,181.5000) -- (314.7000,171.8000) -- (314.0000,168.6000) --
          (315.7000,165.7000) -- (323.6000,159.6000) -- (320.1000,153.8000) --
          (322.3000,150.3000) -- (322.3000,150.4000) -- (316.2000,143.5000) --
          (306.0000,139.4000) -- (295.4000,138.5000) -- (291.1000,138.4000) -- cycle;

        % path4700
        \path[draw,color=cFFFFFF,densely dotted] (283.8000,155.6000) -- (284.1000,158.7000) --
          (277.3000,160.2000) -- (274.1000,158.4000) -- (268.2000,162.0000) --
          (261.4000,163.2000) -- (258.8000,166.1000) -- (258.9000,169.8000) --
          (264.2000,174.6000) -- (264.1000,179.7000) -- (264.6000,184.9000) --
          (256.2000,190.6000) -- (257.4000,199.1000) -- (257.4000,199.1000) --
          (259.8000,203.3000) -- (259.3000,205.4000) -- (266.4000,203.9000) --
          (280.8000,216.0000) -- (286.9000,212.1000) -- (290.6000,214.2000) --
          (290.7000,214.2000) -- (290.4000,211.1000) -- (297.2000,207.9000) --
          (306.7000,207.3000) -- (312.1000,195.1000) -- (312.7000,193.4000) --
          (312.6000,193.4000) -- (309.5000,181.5000) -- (306.0000,182.3000) --
          (303.5000,179.8000) -- (302.1000,173.5000) -- (295.2000,167.5000) --
          (294.6000,155.3000) -- (290.5000,147.9000) -- (283.8000,155.6000) -- cycle;

        % path4702
        \path[draw,color=cFFFFFF,densely dotted] (326.4000,147.4000) -- (322.4000,150.3000) --
          (322.3000,150.3000) -- (320.1000,153.8000) -- (323.6000,159.6000) --
          (323.7000,159.6000) -- (333.6000,163.7000) -- (336.7000,162.7000) --
          (339.3000,165.4000) -- (340.5000,160.6000) -- (340.2000,157.4000) --
          (340.2000,157.4000) -- (338.7000,152.0000) -- (340.2000,149.2000) --
          (338.3000,143.4000) .. controls (337.5000,143.9000) and (336.7000,144.5000) ..
          (336.0000,145.0000) .. controls (335.0000,145.7000) and (334.0000,146.4000) ..
          (333.0000,147.1000) -- (330.1000,145.9000) -- (326.4000,147.4000) --
          cycle(327.8000,150.9000) -- (331.4000,151.4000) -- (332.2000,154.4000) --
          (332.2000,154.5000) -- (334.2000,156.6000) -- (328.3000,156.7000) --
          (324.9000,152.7000) -- (327.8000,150.9000) -- cycle;

        % path4704
        \path[draw,color=cFFFFFF,densely dotted] (323.7000,159.6000) -- (323.6000,159.6000) --
          (315.7000,165.7000) -- (314.0000,168.6000) -- (314.7000,171.8000) --
          (309.5000,181.5000) -- (312.6000,193.4000) -- (312.7000,193.4000) --
          (320.4000,192.6000) -- (322.6000,189.4000) -- (324.3000,192.4000) --
          (327.4000,190.5000) -- (331.6000,191.1000) -- (335.4000,185.8000) --
          (339.8000,166.3000) -- (339.3000,165.4000) -- (336.7000,162.7000) --
          (333.6000,163.7000) -- (323.7000,159.6000) -- cycle;

        % path4706
        \path[draw,color=cFFFFFF,densely dotted] (241.0000,150.0000) -- (240.8000,146.4000) --
          (234.5000,145.3000) -- (226.6000,147.1000) -- (218.6000,153.2000) --
          (215.3000,153.6000) -- (208.8000,152.3000) -- (206.0000,154.3000) --
          (200.4000,150.9000) -- (193.3000,153.5000) -- (190.2000,152.1000) --
          (189.2000,155.0000) -- (183.2000,158.3000) -- (187.8000,163.0000) --
          (185.7000,170.6000) -- (181.7000,175.6000) -- (185.4000,179.7000) --
          (188.1000,177.9000) -- (190.9000,179.8000) -- (198.5000,176.2000) --
          (208.7000,176.2000) -- (209.5000,173.1000) -- (212.5000,173.4000) --
          (214.7000,180.1000) -- (218.4000,181.8000) -- (218.4000,185.1000) --
          (218.3000,185.2000) -- (221.7000,185.3000) -- (234.1000,178.7000) --
          (237.2000,180.4000) -- (238.9000,189.9000) -- (247.1000,195.5000) --
          (250.3000,194.3000) -- (253.8000,198.3000) -- (257.4000,199.1000) --
          (256.2000,190.6000) -- (264.6000,184.9000) -- (264.1000,179.7000) --
          (264.2000,174.6000) -- (258.9000,169.8000) -- (258.8000,166.1000) --
          (254.9000,163.8000) -- (256.9000,161.2000) -- (248.1000,150.6000) --
          (241.0000,150.0000) -- cycle;

        % path4708
        \path[draw,color=cFFFFFF,densely dotted] (259.3000,205.4000) -- (259.8000,203.3000) --
          (257.4000,199.1000) -- (257.4000,199.1000) -- (253.8000,198.3000) --
          (250.3000,194.3000) -- (247.1000,195.5000) -- (238.9000,189.9000) --
          (237.2000,180.4000) -- (234.1000,178.7000) -- (221.7000,185.3000) --
          (218.3000,185.2000) -- (213.5000,188.3000) -- (213.8000,196.1000) --
          (208.8000,200.5000) -- (210.2000,203.8000) -- (204.6000,208.9000) --
          (206.2000,212.8000) -- (200.0000,221.5000) -- (201.7000,227.7000) --
          (201.8000,227.7000) -- (202.4000,230.7000) -- (209.9000,230.6000) --
          (210.1000,234.5000) -- (212.4000,236.7000) -- (215.9000,235.3000) --
          (221.7000,239.0000) -- (230.0000,239.0000) -- (236.9000,241.9000) --
          (236.3000,238.3000) -- (239.5000,238.6000) -- (248.1000,233.6000) --
          (248.1000,233.5000) -- (248.8000,229.6000) -- (252.4000,228.4000) --
          (259.4000,216.5000) -- (259.0000,210.3000) -- (257.1000,207.7000) --
          (259.3000,205.4000) -- cycle;

        % path4710
        \path[draw,color=cFFFFFF,densely dotted] (259.3000,205.4000) -- (257.1000,207.7000) --
          (259.0000,210.3000) -- (259.4000,216.5000) -- (252.4000,228.4000) --
          (248.8000,229.6000) -- (248.1000,233.5000) -- (248.1000,233.6000) --
          (259.9000,234.2000) -- (259.6000,240.3000) -- (262.1000,238.3000) --
          (268.9000,242.0000) -- (268.1000,245.1000) -- (271.9000,250.9000) --
          (270.6000,261.1000) -- (273.2000,263.2000) -- (276.2000,261.8000) --
          (280.9000,269.3000) -- (287.7000,265.7000) -- (290.4000,267.1000) --
          (292.2000,264.2000) -- (299.3000,263.2000) -- (305.1000,267.1000) --
          (308.3000,267.2000) -- (309.0000,264.1000) -- (319.9000,261.8000) --
          (317.6000,255.2000) -- (324.2000,250.6000) -- (318.4000,242.0000) --
          (324.0000,239.4000) -- (323.1000,236.4000) -- (306.0000,234.7000) --
          (303.9000,237.2000) -- (300.7000,236.5000) -- (299.4000,232.3000) --
          (296.8000,230.7000) -- (293.4000,232.1000) -- (291.5000,225.7000) --
          (293.4000,222.3000) -- (290.6000,214.2000) -- (286.9000,212.1000) --
          (280.8000,216.0000) -- (266.4000,203.9000) -- (259.3000,205.4000) -- cycle;

        % path4712
        \path[draw,color=cFFFFFF,densely dotted] (248.1000,233.6000) -- (239.5000,238.6000) --
          (236.3000,238.3000) -- (236.9000,241.9000) -- (230.0000,239.0000) --
          (229.2000,241.8000) -- (227.3000,252.2000) -- (220.8000,269.4000) --
          (220.9000,269.4000) -- (227.0000,273.0000) -- (227.3000,276.6000) --
          (230.6000,276.3000) -- (232.0000,283.2000) -- (235.2000,284.8000) --
          (246.4000,284.3000) -- (244.8000,281.0000) -- (250.8000,284.4000) --
          (251.1000,287.9000) -- (259.0000,299.1000) -- (265.4000,298.6000) --
          (264.3000,295.6000) -- (267.9000,282.2000) -- (273.7000,279.2000) --
          (276.3000,281.1000) -- (279.2000,277.1000) -- (282.0000,274.8000) --
          (280.9000,269.3000) -- (276.2000,261.8000) -- (273.2000,263.2000) --
          (270.6000,261.1000) -- (271.9000,250.9000) -- (268.1000,245.1000) --
          (268.9000,242.0000) -- (262.1000,238.3000) -- (259.6000,240.3000) --
          (259.9000,234.2000) -- (248.1000,233.6000) -- cycle;

        % path4714
        \path[draw,color=cFFFFFF,densely dotted] (331.6000,191.1000) -- (327.4000,190.5000) --
          (324.3000,192.4000) -- (322.6000,189.4000) -- (320.4000,192.6000) --
          (312.7000,193.4000) -- (312.1000,195.1000) -- (306.7000,207.3000) --
          (297.2000,207.9000) -- (290.4000,211.1000) -- (290.7000,214.2000) --
          (290.6000,214.2000) -- (293.4000,222.3000) -- (291.5000,225.7000) --
          (293.4000,232.1000) -- (296.8000,230.7000) -- (299.4000,232.3000) --
          (300.7000,236.5000) -- (303.9000,237.2000) -- (306.0000,234.7000) --
          (323.1000,236.4000) -- (324.0000,239.4000) -- (339.2000,242.9000) --
          (343.5000,248.2000) -- (353.7000,246.4000) -- (355.4000,243.7000) --
          (358.5000,242.9000) -- (356.0000,235.1000) -- (352.7000,232.0000) --
          (358.6000,228.2000) -- (360.4000,225.6000) -- (359.1000,222.3000) --
          (364.3000,216.8000) -- (363.9000,210.3000) -- (356.3000,201.9000) --
          (343.8000,204.9000) -- (333.5000,204.1000) -- (336.4000,202.4000) --
          (337.0000,199.3000) -- (332.3000,194.5000) -- (331.6000,191.1000) -- cycle;

        % path4716
        \path[draw,color=cFFFFFF,densely dotted] (366.2000,142.5000) -- (366.9000,139.3000) --
          (364.9000,136.5000) -- (361.9000,135.7000) -- (352.9000,139.0000) --
          (350.0000,137.2000) -- (347.8000,139.6000) -- (340.0000,138.5000) --
          (340.3000,140.8000) -- (338.4000,143.4000) -- (338.3000,143.4000) --
          (340.2000,149.2000) -- (338.7000,152.0000) -- (340.2000,157.4000) --
          (340.2000,157.4000) -- (340.5000,160.6000) -- (339.3000,165.4000) --
          (339.8000,166.3000) -- (335.4000,185.8000) -- (331.6000,191.1000) --
          (332.3000,194.5000) -- (337.0000,199.3000) -- (336.4000,202.4000) --
          (333.5000,204.1000) -- (343.8000,204.9000) -- (356.3000,201.9000) --
          (360.7000,195.8000) -- (361.0000,189.3000) -- (378.2000,185.9000) --
          (377.8000,176.3000) -- (381.1000,175.4000) -- (384.4000,169.9000) --
          (381.5000,168.0000) -- (377.6000,158.6000) -- (380.5000,156.1000) --
          (381.0000,153.9000) -- (370.6000,147.1000) -- (366.2000,142.5000) -- cycle;

        % path4718
        \path[draw,color=cFFFFFF,densely dotted] (361.0000,189.3000) -- (360.7000,195.8000) --
          (356.3000,201.9000) -- (363.9000,210.3000) -- (364.3000,216.8000) --
          (359.1000,222.3000) -- (360.4000,225.6000) -- (358.6000,228.2000) --
          (352.7000,232.0000) -- (356.0000,235.1000) -- (358.5000,242.9000) --
          (358.5000,243.0000) -- (362.3000,243.0000) -- (370.7000,248.3000) --
          (377.4000,248.3000) -- (382.6000,243.6000) -- (386.9000,249.9000) --
          (393.1000,253.2000) -- (399.6000,252.2000) -- (400.7000,256.3000) --
          (403.8000,255.3000) -- (405.4000,258.5000) -- (411.6000,258.0000) --
          (411.7000,258.0000) -- (410.3000,255.0000) -- (411.8000,245.9000) --
          (419.3000,234.2000) -- (417.3000,231.7000) -- (420.5000,229.3000) --
          (420.9000,225.1000) -- (417.4000,220.7000) -- (419.5000,217.3000) --
          (416.2000,215.7000) -- (402.1000,217.5000) -- (396.4000,205.2000) --
          (392.4000,200.6000) -- (390.2000,203.1000) -- (387.3000,193.8000) --
          (382.3000,187.5000) -- (378.2000,185.9000) -- (361.0000,189.3000) -- cycle;

        % path4720
        \path[draw,color=cFFFFFF,densely dotted] (156.4000,93.7000) -- (149.0000,94.2000) --
          (136.3000,89.0000) -- (136.1000,92.1000) -- (139.7000,94.2000) --
          (140.3000,99.7000) -- (138.2000,101.9000) -- (141.0000,113.4000) --
          (150.6000,123.7000) -- (149.1000,135.0000) -- (151.3000,138.1000) --
          (150.7000,144.9000) -- (148.2000,151.3000) -- (152.9000,161.9000) --
          (155.8000,163.2000) -- (158.6000,161.4000) -- (157.5000,164.8000) --
          (149.5000,165.6000) -- (154.9000,176.4000) -- (157.9000,177.4000) --
          (165.8000,172.1000) -- (172.0000,174.6000) -- (181.6000,175.6000) --
          (181.7000,175.6000) -- (185.7000,170.6000) -- (187.8000,163.0000) --
          (183.2000,158.3000) -- (172.3000,156.0000) -- (170.3000,153.2000) --
          (175.3000,147.3000) -- (173.6000,144.7000) -- (182.7000,139.2000) --
          (180.2000,130.0000) -- (182.4000,127.1000) -- (173.0000,123.6000) --
          (172.3000,116.2000) -- (172.1000,116.2000) -- (164.9000,104.4000) --
          (167.6000,95.5000) -- (166.5000,92.4000) -- (159.2000,91.7000) --
          (156.4000,93.7000) -- cycle;

        % path4722
        \path[draw,color=cFFFFFF,densely dotted] (109.5000,161.7000) -- (97.5000,171.7000) --
          (96.5000,167.6000) -- (91.9000,164.6000) -- (91.9000,161.1000) --
          (86.8000,152.3000) -- (83.5000,151.6000) -- (84.2000,148.5000) --
          (80.8000,148.9000) -- (78.9000,152.2000) -- (80.4000,144.8000) --
          (75.1000,148.4000) -- (74.5000,144.4000) -- (64.5000,148.5000) --
          (62.9000,145.7000) -- (57.8000,149.9000) -- (57.5000,156.9000) --
          (54.0000,157.4000) -- (54.2000,160.5000) -- (58.2000,165.7000) --
          (55.4000,175.1000) -- (58.2000,185.4000) -- (55.9000,190.1000) --
          (55.9000,190.1000) -- (70.0000,193.8000) -- (78.8000,190.4000) --
          (93.3000,197.7000) -- (96.6000,196.7000) -- (98.7000,203.4000) --
          (105.6000,195.0000) -- (109.6000,195.9000) -- (111.3000,199.1000) --
          (114.5000,197.8000) -- (120.0000,189.9000) -- (130.1000,187.0000) --
          (131.9000,176.9000) -- (131.8000,170.6000) -- (131.4000,170.3000) --
          (131.7000,170.8000) -- (129.1000,168.7000) -- (129.0000,168.5000) --
          (125.8000,168.2000) -- (124.1000,165.5000) -- (124.1000,163.7000) --
          (123.9000,163.7000) -- (123.9000,165.8000) -- (119.3000,166.2000) --
          (118.2000,162.7000) -- (114.7000,164.3000) -- (113.6000,161.4000) --
          (109.5000,161.7000) -- cycle;

        % path4724
        \path[draw,color=cFFFFFF,densely dotted] (127.1000,163.7000) -- (124.1000,163.7000) --
          (124.1000,165.5000) -- (125.8000,168.2000) -- (129.0000,168.5000) --
          (127.1000,163.7000) -- cycle;

        % path4726
        \path[draw,color=cFFFFFF,densely dotted] (131.4000,170.3000) -- (131.8000,170.6000) --
          (131.9000,176.9000) -- (130.1000,187.0000) -- (120.0000,189.9000) --
          (114.5000,197.8000) -- (115.7000,202.7000) -- (119.7000,202.8000) --
          (115.8000,203.9000) -- (116.4000,207.1000) -- (122.1000,210.2000) --
          (123.4000,213.0000) -- (121.6000,215.9000) -- (124.3000,217.9000) --
          (123.2000,221.5000) -- (120.5000,223.8000) -- (123.6000,225.6000) --
          (120.3000,226.4000) -- (121.4000,232.8000) -- (127.2000,229.0000) --
          (142.0000,228.5000) -- (143.8000,225.4000) -- (153.9000,221.0000) --
          (161.5000,224.9000) -- (162.1000,222.6000) -- (166.1000,212.9000) --
          (172.3000,210.4000) -- (170.1000,190.9000) -- (171.9000,188.3000) --
          (172.0000,174.6000) -- (165.8000,172.1000) -- (157.9000,177.4000) --
          (154.9000,176.4000) -- (149.5000,165.6000) -- (136.6000,165.7000) --
          (136.9000,159.4000) -- (133.3000,160.2000) -- (128.8000,164.5000) --
          (131.4000,170.3000) -- cycle;

        % path4728
        \path[draw,color=cFFFFFF,densely dotted] (54.2000,160.5000) -- (54.0000,157.4000) --
          (53.8000,157.4000) -- (48.4000,154.0000) -- (45.2000,154.7000) --
          (44.5000,158.5000) -- (42.0000,155.5000) -- (40.2000,159.1000) --
          (40.1000,151.9000) -- (23.9000,156.6000) -- (22.2000,153.6000) --
          (10.8000,159.9000) -- (13.7000,161.4000) -- (4.3000,160.4000) --
          (2.1000,163.1000) -- (0.6000,173.2000) -- (2.6000,175.8000) --
          (5.6000,174.5000) -- (8.6000,175.9000) -- (22.4000,171.7000) --
          (18.2000,173.7000) -- (17.6000,177.5000) -- (24.3000,175.9000) --
          (23.8000,179.2000) -- (27.1000,180.6000) -- (23.6000,181.3000) --
          (30.6000,184.3000) -- (23.9000,182.7000) -- (22.4000,179.9000) --
          (18.2000,181.1000) -- (11.7000,179.8000) -- (9.8000,177.1000) --
          (8.1000,181.2000) -- (10.8000,187.0000) -- (12.8000,184.0000) --
          (15.9000,184.2000) -- (22.0000,187.8000) -- (23.0000,191.4000) --
          (21.2000,194.2000) -- (18.4000,192.7000) -- (4.3000,194.1000) --
          (2.6000,196.9000) -- (13.3000,200.9000) -- (17.5000,207.9000) --
          (16.2000,214.0000) -- (24.5000,214.7000) -- (28.6000,210.1000) --
          (28.6000,209.6000) -- (28.6000,209.6000) -- (27.7000,206.3000) --
          (28.9000,206.1000) -- (29.2000,202.7000) -- (30.8000,205.9000) --
          (28.9000,206.1000) -- (28.6000,209.6000) -- (31.9000,211.8000) --
          (37.1000,210.0000) -- (42.4000,216.6000) -- (45.5000,215.9000) --
          (45.5000,212.7000) -- (46.0000,215.9000) -- (48.9000,214.9000) --
          (48.2000,218.0000) -- (51.4000,218.8000) -- (55.6000,217.1000) --
          (55.9000,213.6000) -- (59.0000,214.6000) -- (61.8000,211.4000) --
          (61.7000,205.3000) -- (52.6000,203.8000) -- (48.8000,197.7000) --
          (48.2000,193.7000) -- (55.9000,190.1000) -- (58.2000,185.4000) --
          (55.4000,175.1000) -- (58.2000,165.7000) -- (54.2000,160.5000) -- cycle;

        % path4730
        \path[draw,color=cFFFFFF,densely dotted] (78.8000,190.4000) -- (70.0000,193.8000) --
          (55.9000,190.1000) -- (55.9000,190.1000) -- (48.2000,193.7000) --
          (48.8000,197.7000) -- (52.6000,203.8000) -- (61.7000,205.3000) --
          (61.8000,211.4000) -- (59.0000,214.6000) -- (55.9000,213.6000) --
          (56.7000,222.3000) -- (59.5000,224.0000) -- (62.5000,222.5000) --
          (62.0000,215.9000) -- (62.7000,219.3000) -- (66.0000,220.0000) --
          (63.6000,222.4000) -- (65.3000,225.3000) -- (69.4000,227.1000) --
          (72.4000,225.7000) -- (69.7000,227.1000) -- (70.0000,230.3000) --
          (72.3000,233.4000) -- (71.5000,237.6000) -- (73.9000,240.2000) --
          (72.5000,237.0000) -- (75.6000,234.3000) -- (79.2000,233.4000) --
          (81.9000,235.5000) -- (80.5000,228.9000) -- (84.0000,234.3000) --
          (85.7000,231.5000) -- (89.8000,232.1000) -- (92.2000,234.9000) --
          (89.9000,237.6000) -- (82.9000,236.7000) -- (88.3000,241.0000) --
          (102.2000,239.8000) -- (105.3000,241.6000) -- (101.9000,241.8000) --
          (103.7000,244.7000) -- (120.9000,239.6000) -- (121.4000,232.8000) --
          (120.3000,226.4000) -- (123.6000,225.6000) -- (120.5000,223.8000) --
          (123.2000,221.5000) -- (124.3000,217.9000) -- (121.6000,215.9000) --
          (123.4000,213.0000) -- (122.1000,210.2000) -- (116.4000,207.1000) --
          (115.8000,203.9000) -- (119.7000,202.8000) -- (115.7000,202.7000) --
          (114.5000,197.8000) -- (111.3000,199.1000) -- (109.6000,195.9000) --
          (105.6000,195.0000) -- (98.7000,203.4000) -- (96.6000,196.7000) --
          (93.3000,197.7000) -- (78.8000,190.4000) -- cycle;

        % path4732
        \path[draw,color=cFFFFFF,densely dotted] (127.2000,229.0000) -- (121.4000,232.8000) --
          (120.9000,239.6000) -- (103.7000,244.7000) -- (98.7000,249.5000) --
          (101.3000,256.5000) -- (108.1000,257.6000) -- (110.4000,259.9000) --
          (116.0000,255.8000) -- (124.1000,255.1000) -- (116.5000,258.0000) --
          (115.8000,265.4000) -- (112.4000,266.9000) -- (119.0000,269.3000) --
          (124.4000,274.9000) -- (131.2000,281.9000) -- (144.3000,287.5000) --
          (144.6000,276.8000) -- (148.4000,276.5000) -- (148.4000,280.1000) --
          (151.0000,282.0000) -- (153.7000,279.6000) -- (153.2000,276.3000) --
          (156.7000,275.4000) -- (158.4000,272.7000) -- (163.8000,276.6000) --
          (163.4000,273.3000) -- (160.3000,271.7000) -- (163.4000,266.7000) --
          (162.5000,263.6000) -- (158.5000,258.3000) -- (155.1000,257.4000) --
          (158.3000,254.9000) -- (174.4000,252.5000) -- (174.1000,245.9000) --
          (164.5000,242.2000) -- (166.6000,239.5000) -- (169.8000,239.9000) --
          (161.5000,224.9000) -- (153.9000,221.0000) -- (143.8000,225.4000) --
          (142.0000,228.5000) -- (127.2000,229.0000) -- cycle;

        % path4734
        \path[draw,color=cFFFFFF,densely dotted] (212.5000,173.4000) -- (209.5000,173.1000) --
          (208.7000,176.2000) -- (198.5000,176.2000) -- (190.9000,179.8000) --
          (188.1000,177.9000) -- (185.4000,179.7000) -- (181.7000,175.6000) --
          (181.6000,175.6000) -- (172.0000,174.6000) -- (171.9000,188.3000) --
          (170.1000,190.9000) -- (172.3000,210.4000) -- (166.1000,212.9000) --
          (162.1000,222.6000) -- (173.8000,226.5000) -- (176.2000,224.3000) --
          (186.0000,228.4000) -- (189.4000,228.9000) -- (196.0000,225.7000) --
          (201.7000,227.7000) -- (200.0000,221.5000) -- (206.2000,212.8000) --
          (204.6000,208.9000) -- (210.2000,203.8000) -- (208.8000,200.5000) --
          (213.8000,196.1000) -- (213.5000,188.3000) -- (218.3000,185.2000) --
          (218.4000,185.1000) -- (218.4000,181.8000) -- (214.7000,180.1000) --
          (212.5000,173.4000) -- cycle;

        % path4736
        \path[draw,color=cFFFFFF,densely dotted] (173.8000,226.5000) -- (162.1000,222.6000) --
          (161.5000,224.9000) -- (169.8000,239.9000) -- (166.6000,239.5000) --
          (164.5000,242.2000) -- (174.1000,245.9000) -- (174.4000,252.5000) --
          (158.3000,254.9000) -- (155.1000,257.4000) -- (158.5000,258.3000) --
          (162.5000,263.6000) -- (163.4000,266.7000) -- (160.3000,271.7000) --
          (163.4000,273.3000) -- (163.8000,276.6000) -- (175.8000,280.9000) --
          (189.6000,280.0000) -- (193.7000,274.7000) -- (200.0000,273.8000) --
          (210.9000,273.5000) -- (213.4000,276.2000) -- (220.9000,269.4000) --
          (220.8000,269.4000) -- (227.3000,252.2000) -- (229.2000,241.8000) --
          (230.0000,239.0000) -- (221.7000,239.0000) -- (215.9000,235.3000) --
          (212.4000,236.7000) -- (210.1000,234.5000) -- (209.9000,230.6000) --
          (202.4000,230.7000) -- (201.8000,227.7000) -- (201.7000,227.7000) --
          (196.0000,225.7000) -- (189.4000,228.9000) -- (186.0000,228.4000) --
          (176.2000,224.3000) -- (173.8000,226.5000) -- cycle;

        % path4738
        \path[draw,color=cFFFFFF,densely dotted] (200.0000,273.8000) -- (193.7000,274.7000) --
          (189.6000,280.0000) -- (175.8000,280.9000) -- (178.3000,287.3000) --
          (183.0000,291.9000) -- (188.4000,312.1000) -- (188.1000,320.4000) --
          (190.9000,322.6000) -- (180.7000,327.6000) -- (180.7000,331.8000) --
          (187.3000,339.1000) -- (206.9000,346.7000) -- (210.9000,351.9000) --
          (214.0000,350.1000) -- (214.3000,346.8000) -- (220.7000,343.9000) --
          (224.8000,344.4000) -- (225.6000,343.6000) -- (225.7000,343.6000) --
          (226.7000,340.4000) -- (221.6000,337.0000) -- (225.2000,327.5000) --
          (218.1000,327.3000) -- (218.1000,324.1000) -- (215.8000,321.9000) --
          (215.1000,315.4000) -- (219.0000,309.2000) -- (217.3000,305.8000) --
          (214.3000,306.7000) -- (219.3000,290.9000) -- (216.3000,289.6000) --
          (217.2000,286.2000) -- (213.4000,276.2000) -- (210.9000,273.5000) --
          (200.0000,273.8000) -- cycle;

        % path4740
        \path[draw,color=cFFFFFF,densely dotted] (230.6000,276.3000) -- (227.3000,276.6000) --
          (227.0000,273.0000) -- (220.9000,269.4000) -- (213.4000,276.2000) --
          (217.2000,286.2000) -- (216.3000,289.6000) -- (219.3000,290.9000) --
          (214.3000,306.7000) -- (217.3000,305.8000) -- (219.0000,309.2000) --
          (215.1000,315.4000) -- (215.8000,321.9000) -- (218.1000,324.1000) --
          (218.1000,327.3000) -- (225.2000,327.5000) -- (221.6000,337.0000) --
          (226.7000,340.4000) -- (225.7000,343.6000) -- (235.9000,346.9000) --
          (239.2000,346.0000) -- (237.6000,343.3000) -- (240.1000,341.2000) --
          (244.7000,345.6000) -- (249.0000,344.0000) -- (250.5000,341.2000) --
          (255.8000,342.0000) -- (255.8000,338.1000) -- (259.2000,332.4000) --
          (265.6000,329.4000) -- (266.9000,326.0000) -- (270.7000,327.0000) --
          (272.1000,324.3000) -- (273.1000,324.6000) -- (274.8000,321.5000) --
          (271.9000,320.1000) -- (269.0000,313.9000) -- (265.7000,313.9000) --
          (260.6000,309.1000) -- (261.4000,302.5000) -- (259.0000,299.1000) --
          (251.1000,287.9000) -- (250.8000,284.4000) -- (244.8000,281.0000) --
          (246.4000,284.3000) -- (235.2000,284.8000) -- (232.0000,283.2000) --
          (230.6000,276.3000) -- cycle;

        % path4742
        \path[draw,color=cFFFFFF,densely dotted] (144.6000,276.8000) -- (144.3000,287.5000) --
          (131.2000,281.9000) -- (124.4000,274.9000) -- (115.7000,284.2000) --
          (116.1000,289.2000) -- (127.8000,302.9000) -- (130.2000,310.8000) --
          (139.3000,318.3000) -- (144.7000,319.6000) -- (146.5000,323.5000) --
          (153.1000,324.8000) -- (158.4000,329.1000) -- (159.2000,326.0000) --
          (162.4000,326.5000) -- (172.3000,322.7000) -- (170.8000,326.0000) --
          (178.0000,325.7000) -- (180.7000,327.6000) -- (190.9000,322.6000) --
          (188.1000,320.4000) -- (188.4000,312.1000) -- (183.0000,291.9000) --
          (178.3000,287.3000) -- (175.8000,280.9000) -- (163.8000,276.6000) --
          (158.4000,272.7000) -- (156.7000,275.4000) -- (153.2000,276.3000) --
          (153.7000,279.6000) -- (151.0000,282.0000) -- (148.4000,280.1000) --
          (148.4000,276.5000) -- (144.6000,276.8000) -- cycle;

        % path4744
        \path[draw,color=cFFFFFF,densely dotted] (178.0000,325.7000) -- (170.8000,326.0000) --
          (172.3000,322.7000) -- (162.4000,326.5000) -- (157.3000,337.3000) --
          (160.6000,338.4000) -- (165.2000,347.3000) -- (162.1000,347.9000) --
          (163.9000,353.9000) -- (160.6000,360.3000) -- (155.2000,364.0000) --
          (156.7000,368.8000) -- (165.5000,374.3000) -- (177.2000,386.9000) --
          (179.5000,395.2000) -- (179.5000,395.3000) -- (186.1000,394.3000) --
          (186.9000,397.6000) -- (194.1000,399.5000) -- (194.7000,405.9000) --
          (201.6000,410.8000) -- (212.1000,411.9000) -- (214.4000,406.3000) --
          (214.5000,405.8000) -- (214.5000,405.8000) -- (207.4000,398.6000) --
          (200.7000,397.3000) -- (202.6000,394.0000) -- (200.1000,391.7000) --
          (202.7000,389.8000) -- (201.0000,386.8000) -- (203.0000,383.9000) --
          (200.2000,382.1000) -- (199.9000,378.7000) -- (194.3000,375.2000) --
          (196.8000,372.9000) -- (193.2000,366.5000) -- (202.3000,362.8000) --
          (204.8000,365.1000) -- (207.9000,364.9000) -- (210.9000,351.9000) --
          (206.9000,346.7000) -- (187.3000,339.1000) -- (180.7000,331.8000) --
          (180.7000,327.6000) -- (178.0000,325.7000) -- cycle;

        % path4746
        \path[draw,color=cFFFFFF,densely dotted] (214.0000,350.1000) -- (210.9000,351.9000) --
          (207.9000,364.9000) -- (204.8000,365.1000) -- (202.3000,362.8000) --
          (193.2000,366.5000) -- (196.8000,372.9000) -- (194.3000,375.2000) --
          (199.9000,378.7000) -- (200.2000,382.1000) -- (203.0000,383.9000) --
          (201.0000,386.8000) -- (202.7000,389.8000) -- (200.1000,391.7000) --
          (202.6000,394.0000) -- (200.7000,397.3000) -- (207.4000,398.6000) --
          (214.5000,405.8000) -- (214.5000,405.8000) -- (216.8000,403.6000) --
          (222.9000,402.8000) -- (227.5000,396.3000) -- (228.4000,389.9000) --
          (237.2000,383.3000) -- (239.6000,377.3000) -- (245.9000,370.5000) --
          (245.9000,370.4000) -- (249.7000,366.1000) -- (252.8000,365.5000) --
          (257.3000,356.5000) -- (260.2000,355.3000) -- (260.9000,351.7000) --
          (255.6000,347.7000) -- (255.8000,342.0000) -- (250.5000,341.2000) --
          (249.0000,344.0000) -- (244.7000,345.6000) -- (240.1000,341.2000) --
          (237.6000,343.3000) -- (239.2000,346.0000) -- (235.9000,346.9000) --
          (225.7000,343.6000) -- (225.6000,343.6000) -- (224.8000,344.4000) --
          (220.7000,343.9000) -- (214.3000,346.8000) -- (214.0000,350.1000) -- cycle;

        % path4748
        \path[draw,color=cFFFFFF,densely dotted] (201.6000,410.8000) -- (194.7000,405.9000) --
          (194.1000,399.5000) -- (186.9000,397.6000) -- (186.1000,394.3000) --
          (179.5000,395.3000) -- (180.2000,398.4000) -- (177.8000,402.1000) --
          (175.0000,393.4000) -- (162.9000,381.6000) -- (161.9000,378.4000) --
          (158.6000,384.9000) -- (151.6000,441.2000) -- (156.1000,432.4000) --
          (161.1000,438.2000) -- (161.4000,441.4000) -- (154.2000,440.4000) --
          (150.6000,448.2000) -- (150.7000,453.5000) -- (159.5000,449.6000) --
          (163.7000,451.1000) -- (162.1000,457.0000) -- (181.4000,456.2000) --
          (180.9000,459.5000) -- (193.3000,468.0000) -- (194.0000,473.3000) --
          (201.2000,473.3000) -- (203.0000,469.3000) -- (205.3000,472.3000) --
          (210.4000,467.8000) -- (208.7000,465.1000) -- (212.1000,462.5000) --
          (212.7000,452.5000) -- (222.3000,442.7000) -- (219.0000,441.2000) --
          (220.2000,437.6000) -- (223.3000,438.5000) -- (224.6000,435.7000) --
          (228.1000,435.9000) -- (227.0000,434.4000) -- (228.0000,431.1000) --
          (225.4000,428.2000) -- (223.0000,431.8000) -- (212.0000,429.3000) --
          (214.3000,427.0000) -- (214.5000,420.6000) -- (217.7000,414.4000) --
          (216.4000,411.3000) -- (212.1000,411.9000) -- (201.6000,410.8000) -- cycle;

        % path4750
        \path[draw,color=cFFFFFF,densely dotted] (228.4000,389.9000) -- (227.5000,396.3000) --
          (222.9000,402.8000) -- (216.8000,403.6000) -- (214.5000,405.8000) --
          (214.5000,405.8000) -- (214.4000,406.3000) -- (212.1000,411.9000) --
          (216.4000,411.3000) -- (217.7000,414.4000) -- (214.5000,420.6000) --
          (214.3000,427.0000) -- (212.0000,429.3000) -- (223.0000,431.8000) --
          (225.4000,428.2000) -- (228.0000,431.1000) -- (227.0000,434.4000) --
          (228.1000,435.9000) -- (231.2000,443.0000) -- (241.2000,442.5000) --
          (244.4000,440.2000) -- (246.8000,442.2000) -- (253.7000,441.0000) --
          (255.4000,447.9000) -- (261.4000,444.8000) -- (266.2000,449.5000) --
          (271.4000,442.3000) -- (278.3000,438.4000) -- (277.5000,434.4000) --
          (282.6000,430.8000) -- (284.2000,424.8000) -- (284.9000,419.2000) --
          (281.1000,411.3000) -- (277.4000,410.6000) -- (277.2000,407.2000) --
          (274.6000,405.5000) -- (276.9000,403.4000) -- (274.9000,400.1000) --
          (278.9000,395.0000) -- (275.8000,389.6000) -- (269.3000,387.2000) --
          (271.5000,384.2000) -- (263.5000,377.9000) -- (252.9000,379.1000) --
          (252.9000,373.8000) -- (245.9000,370.4000) -- (245.9000,370.5000) --
          (239.6000,377.3000) -- (237.2000,383.3000) -- (228.4000,389.9000) -- cycle;

        % path4752
        \path[draw,color=cFFFFFF,densely dotted] (163.7000,451.1000) -- (159.5000,449.6000) --
          (150.7000,453.5000) -- (138.3000,509.7000) -- (134.1000,517.6000) --
          (139.8000,520.3000) -- (152.1000,517.5000) -- (153.5000,520.7000) --
          (156.5000,519.1000) -- (159.3000,520.8000) -- (161.2000,518.3000) --
          (172.2000,516.0000) -- (175.0000,518.6000) -- (177.7000,516.9000) --
          (180.0000,519.1000) -- (185.5000,517.0000) -- (189.1000,518.3000) --
          (198.8000,516.6000) -- (197.0000,513.5000) -- (201.8000,501.1000) --
          (201.2000,493.9000) -- (211.7000,490.9000) -- (212.4000,494.2000) --
          (215.3000,492.8000) -- (215.6000,489.5000) -- (214.9000,486.6000) --
          (218.9000,480.7000) -- (205.8000,477.5000) -- (205.3000,472.3000) --
          (203.0000,469.3000) -- (201.2000,473.3000) -- (194.0000,473.3000) --
          (193.3000,468.0000) -- (180.9000,459.5000) -- (181.4000,456.2000) --
          (162.1000,457.0000) -- (163.7000,451.1000) -- cycle;

        % path4754
        \path[draw,color=cFFFFFF,densely dotted] (220.2000,437.6000) -- (219.0000,441.2000) --
          (222.3000,442.7000) -- (212.7000,452.5000) -- (212.1000,462.5000) --
          (208.7000,465.1000) -- (210.4000,467.8000) -- (205.3000,472.3000) --
          (205.8000,477.5000) -- (218.9000,480.7000) -- (214.9000,486.6000) --
          (215.6000,489.5000) -- (215.6000,489.5000) -- (219.8000,490.2000) --
          (221.4000,487.4000) -- (230.7000,488.2000) -- (236.3000,484.9000) --
          (246.0000,486.1000) -- (249.0000,484.6000) -- (253.0000,479.3000) --
          (256.2000,479.6000) -- (254.7000,476.5000) -- (259.6000,470.3000) --
          (256.9000,468.8000) -- (256.9000,465.5000) -- (265.4000,463.3000) --
          (262.4000,453.0000) -- (266.2000,449.5000) -- (261.4000,444.8000) --
          (255.4000,447.9000) -- (253.7000,441.0000) -- (246.8000,442.2000) --
          (244.4000,440.2000) -- (241.2000,442.5000) -- (231.2000,443.0000) --
          (228.1000,435.9000) -- (224.6000,435.7000) -- (223.3000,438.5000) --
          (220.2000,437.6000) -- cycle;

        % path4756
        \path[draw,color=cFFFFFF,densely dotted] (417.2000,266.1000) -- (413.0000,264.8000) --
          (411.6000,258.0000) -- (405.4000,258.5000) -- (403.8000,255.3000) --
          (400.7000,256.3000) -- (399.6000,252.2000) -- (393.1000,253.2000) --
          (386.9000,249.9000) -- (382.6000,243.6000) -- (377.4000,248.3000) --
          (370.7000,248.3000) -- (362.3000,243.0000) -- (358.5000,243.0000) --
          (358.5000,242.9000) -- (355.4000,243.7000) -- (353.7000,246.4000) --
          (355.9000,253.7000) -- (354.0000,260.1000) -- (357.0000,262.5000) --
          (360.5000,271.0000) -- (360.8000,277.7000) -- (363.4000,279.9000) --
          (363.1000,286.5000) -- (361.6000,296.4000) -- (368.1000,303.6000) --
          (375.2000,303.7000) -- (378.4000,301.8000) -- (383.3000,306.0000) --
          (390.0000,299.2000) -- (397.8000,302.4000) -- (410.5000,295.9000) --
          (409.7000,292.8000) -- (411.9000,290.3000) -- (407.4000,282.6000) --
          (409.7000,280.6000) -- (409.4000,274.0000) -- (412.6000,272.9000) --
          (412.5000,271.3000) -- (415.5000,271.0000) -- (416.8000,270.7000) --
          (416.8000,269.7000) -- (417.2000,266.1000) -- cycle;

        % path4758
        \path[draw,color=cFFFFFF,densely dotted] (355.9000,253.7000) -- (353.7000,246.4000) --
          (343.5000,248.2000) -- (339.2000,242.9000) -- (324.0000,239.4000) --
          (318.4000,242.0000) -- (324.2000,250.6000) -- (317.6000,255.2000) --
          (319.9000,261.8000) -- (309.0000,264.1000) -- (308.3000,267.2000) --
          (305.1000,267.1000) -- (302.1000,273.3000) -- (315.0000,275.1000) --
          (317.6000,288.2000) -- (314.8000,293.9000) -- (317.4000,296.5000) --
          (315.9000,299.5000) -- (321.1000,308.3000) -- (322.1000,318.3000) --
          (320.4000,322.3000) -- (325.9000,322.5000) -- (325.9000,322.4000) --
          (329.8000,315.8000) -- (339.7000,314.7000) -- (341.6000,312.0000) --
          (339.7000,306.0000) -- (342.2000,303.9000) -- (351.7000,300.6000) --
          (357.0000,296.4000) -- (361.6000,296.4000) -- (363.1000,286.5000) --
          (363.4000,279.9000) -- (360.8000,277.7000) -- (360.5000,271.0000) --
          (357.0000,262.5000) -- (354.0000,260.1000) -- (355.9000,253.7000) -- cycle;

        % path4760
        \path[draw,color=cFFFFFF,densely dotted] (361.6000,296.4000) -- (357.0000,296.4000) --
          (351.7000,300.6000) -- (342.2000,303.9000) -- (339.7000,306.0000) --
          (341.6000,312.0000) -- (339.7000,314.7000) -- (329.8000,315.8000) --
          (325.9000,322.4000) -- (325.9000,322.5000) -- (327.2000,326.8000) .. controls
          (330.0000,328.7000) and (332.7000,330.5000) .. (335.5000,332.3000) --
          (339.5000,341.7000) -- (342.4000,343.1000) -- (354.9000,334.0000) --
          (357.5000,336.0000) -- (356.1000,339.2000) -- (361.1000,344.2000) --
          (364.8000,343.7000) -- (366.6000,346.5000) -- (381.4000,347.0000) --
          (382.8000,350.6000) -- (390.0000,351.1000) -- (394.1000,355.8000) --
          (400.2000,350.5000) -- (398.5000,336.5000) -- (399.1000,333.1000) --
          (403.5000,331.7000) -- (407.6000,326.5000) -- (408.2000,320.4000) --
          (406.3000,317.7000) -- (397.5000,314.4000) -- (390.0000,299.2000) --
          (383.3000,306.0000) -- (378.4000,301.8000) -- (375.2000,303.7000) --
          (368.1000,303.6000) -- (361.6000,296.4000) -- cycle;

        % path4762
        \path[draw,color=cFFFFFF,densely dotted] (339.0000,346.3000) -- (339.5000,341.7000) --
          (335.5000,332.3000) .. controls (332.7000,330.5000) and (330.0000,328.7000) ..
          (327.2000,326.8000) -- (325.9000,322.5000) -- (320.4000,322.3000) --
          (317.4000,323.2000) -- (302.6000,320.2000) -- (300.4000,323.0000) --
          (287.5000,324.1000) -- (284.4000,327.4000) -- (285.7000,330.3000) --
          (282.3000,336.3000) -- (289.8000,347.3000) -- (291.7000,353.1000) --
          (290.8000,357.0000) -- (294.0000,358.6000) -- (294.1000,361.9000) --
          (301.3000,362.4000) -- (306.6000,365.7000) -- (307.3000,372.4000) --
          (315.2000,368.6000) -- (327.8000,374.4000) -- (330.3000,371.5000) --
          (333.4000,372.4000) -- (336.1000,369.7000) -- (331.1000,363.5000) --
          (341.5000,353.9000) -- (339.0000,346.3000) -- cycle;

        % path4764
        \path[draw,color=cFFFFFF,densely dotted] (339.5000,341.7000) -- (339.0000,346.3000) --
          (341.5000,353.9000) -- (331.1000,363.5000) -- (336.1000,369.7000) --
          (337.4000,375.9000) -- (335.5000,378.7000) -- (336.9000,387.6000) --
          (345.3000,391.4000) -- (347.0000,394.3000) -- (354.5000,394.6000) --
          (361.8000,400.7000) -- (365.9000,396.0000) -- (365.9000,396.0000) --
          (376.7000,392.4000) -- (386.2000,392.5000) -- (392.1000,396.1000) --
          (404.6000,395.3000) -- (407.9000,387.3000) -- (404.9000,378.8000) --
          (399.7000,374.6000) -- (394.5000,366.0000) -- (396.9000,358.9000) --
          (394.1000,355.8000) -- (390.0000,351.1000) -- (382.8000,350.6000) --
          (381.4000,347.0000) -- (366.6000,346.5000) -- (364.8000,343.7000) --
          (361.1000,344.2000) -- (356.1000,339.2000) -- (357.5000,336.0000) --
          (354.9000,334.0000) -- (342.4000,343.1000) -- (339.5000,341.7000) -- cycle;

        % path4766
        \path[draw,color=cFFFFFF,densely dotted] (305.1000,267.1000) -- (299.3000,263.2000) --
          (292.2000,264.2000) -- (290.4000,267.1000) -- (287.7000,265.7000) --
          (280.9000,269.3000) -- (282.0000,274.8000) -- (279.2000,277.1000) --
          (276.3000,281.1000) -- (273.7000,279.2000) -- (267.9000,282.2000) --
          (264.3000,295.6000) -- (265.4000,298.6000) -- (259.0000,299.1000) --
          (261.4000,302.5000) -- (260.6000,309.1000) -- (265.7000,313.9000) --
          (269.0000,313.9000) -- (271.9000,320.1000) -- (274.8000,321.5000) --
          (273.1000,324.6000) -- (275.9000,325.9000) -- (282.4000,324.7000) --
          (284.4000,327.4000) -- (287.5000,324.1000) -- (300.4000,323.0000) --
          (302.6000,320.2000) -- (317.4000,323.2000) -- (320.4000,322.3000) --
          (322.1000,318.3000) -- (321.1000,308.3000) -- (315.9000,299.5000) --
          (317.4000,296.5000) -- (314.8000,293.9000) -- (317.6000,288.2000) --
          (315.0000,275.1000) -- (302.1000,273.3000) -- (305.1000,267.1000) -- cycle;

        % path4768
        \path[draw,color=cFFFFFF,densely dotted] (275.9000,325.9000) -- (273.1000,324.6000) --
          (272.1000,324.3000) -- (270.7000,327.0000) -- (266.9000,326.0000) --
          (265.6000,329.4000) -- (259.2000,332.4000) -- (255.8000,338.1000) --
          (255.8000,342.0000) -- (255.6000,347.7000) -- (260.9000,351.7000) --
          (260.2000,355.3000) -- (257.3000,356.5000) -- (252.8000,365.5000) --
          (249.7000,366.1000) -- (245.9000,370.4000) -- (252.9000,373.8000) --
          (252.9000,379.1000) -- (263.5000,377.9000) -- (271.5000,384.2000) --
          (269.3000,387.2000) -- (275.8000,389.6000) -- (277.5000,386.7000) --
          (280.3000,387.9000) -- (285.6000,383.5000) -- (292.1000,382.1000) --
          (297.4000,376.8000) -- (307.3000,372.4000) -- (306.6000,365.7000) --
          (301.3000,362.4000) -- (294.1000,361.9000) -- (294.0000,358.6000) --
          (290.8000,357.0000) -- (291.7000,353.1000) -- (289.8000,347.3000) --
          (282.3000,336.3000) -- (285.7000,330.3000) -- (284.4000,327.4000) --
          (282.4000,324.7000) -- (275.9000,325.9000) -- cycle;

        % path4770
        \path[draw,color=cFFFFFF,densely dotted] (330.3000,371.5000) -- (327.8000,374.4000) --
          (315.2000,368.6000) -- (307.3000,372.4000) -- (297.4000,376.8000) --
          (292.1000,382.1000) -- (285.6000,383.5000) -- (280.3000,387.9000) --
          (277.5000,386.7000) -- (275.8000,389.6000) -- (278.9000,395.0000) --
          (274.9000,400.1000) -- (276.9000,403.4000) -- (274.6000,405.5000) --
          (277.2000,407.2000) -- (277.4000,410.6000) -- (281.1000,411.3000) --
          (284.9000,419.2000) -- (294.0000,418.3000) -- (301.6000,425.1000) --
          (307.5000,422.2000) -- (315.1000,422.3000) -- (315.1000,422.3000) --
          (318.8000,420.2000) -- (317.1000,416.4000) -- (320.6000,415.2000) --
          (320.9000,411.3000) -- (323.5000,409.1000) -- (321.4000,406.6000) --
          (330.0000,396.0000) -- (330.7000,392.1000) -- (337.7000,394.3000) --
          (336.9000,387.6000) -- (335.5000,378.7000) -- (337.4000,375.9000) --
          (336.1000,369.7000) -- (333.4000,372.4000) -- (330.3000,371.5000) -- cycle;

        % path4772
        \path[draw,color=cFFFFFF,densely dotted] (330.7000,392.1000) -- (330.0000,396.0000) --
          (321.4000,406.6000) -- (323.5000,409.1000) -- (320.9000,411.3000) --
          (320.6000,415.2000) -- (317.1000,416.4000) -- (318.8000,420.2000) --
          (315.1000,422.3000) -- (315.1000,422.3000) -- (316.2000,428.5000) --
          (320.4000,435.2000) -- (318.5000,441.7000) -- (322.1000,447.3000) --
          (328.0000,443.7000) -- (335.1000,445.4000) -- (339.4000,439.9000) --
          (342.8000,429.8000) -- (348.1000,425.1000) -- (354.0000,429.5000) --
          (355.4000,434.3000) -- (358.0000,436.0000) -- (357.5000,439.4000) --
          (360.4000,445.2000) -- (368.1000,427.3000) -- (370.1000,429.9000) --
          (377.7000,423.5000) -- (378.9000,422.4000) -- (375.0000,413.4000) --
          (378.7000,413.2000) -- (375.8000,411.9000) -- (372.1000,402.1000) --
          (366.0000,400.5000) -- (365.9000,396.0000) -- (365.9000,396.0000) --
          (361.8000,400.7000) -- (354.5000,394.6000) -- (347.0000,394.3000) --
          (345.3000,391.4000) -- (336.9000,387.6000) -- (337.7000,394.3000) --
          (330.7000,392.1000) -- cycle;

        % path4774
        \path[draw,color=cFFFFFF,densely dotted] (282.6000,430.8000) -- (277.5000,434.4000) --
          (278.3000,438.4000) -- (271.4000,442.3000) -- (266.2000,449.5000) --
          (262.4000,453.0000) -- (265.4000,463.3000) -- (268.9000,462.7000) --
          (266.4000,465.3000) -- (271.5000,469.7000) -- (274.6000,470.9000) --
          (277.3000,469.3000) -- (277.8000,473.6000) -- (287.6000,470.9000) --
          (290.6000,473.5000) -- (299.0000,467.2000) -- (305.9000,466.4000) --
          (305.0000,456.1000) -- (313.6000,450.5000) -- (320.3000,449.0000) --
          (322.1000,447.3000) -- (322.1000,447.3000) -- (318.5000,441.7000) --
          (320.4000,435.2000) -- (316.2000,428.5000) -- (315.1000,422.3000) --
          (307.5000,422.2000) -- (301.6000,425.1000) -- (294.0000,418.3000) --
          (284.9000,419.2000) -- (284.2000,424.8000) -- (282.6000,430.8000) -- cycle;

        % path4776
        \path[draw,color=cFFFFFF,densely dotted] (256.9000,465.5000) -- (256.9000,468.8000) --
          (259.6000,470.3000) -- (254.7000,476.5000) -- (256.2000,479.6000) --
          (253.0000,479.3000) -- (249.0000,484.6000) -- (254.9000,486.5000) --
          (251.3000,491.7000) -- (256.9000,496.3000) -- (259.2000,504.0000) --
          (270.4000,503.0000) -- (272.2000,505.6000) -- (278.8000,502.8000) --
          (276.3000,500.2000) -- (289.4000,495.3000) -- (289.4000,495.3000) --
          (289.9000,492.2000) -- (295.4000,488.6000) -- (296.5000,485.6000) --
          (294.4000,483.1000) -- (297.9000,481.9000) -- (300.8000,483.1000) --
          (311.2000,479.5000) -- (306.3000,475.5000) -- (309.0000,472.8000) --
          (305.5000,469.7000) -- (306.0000,466.4000) -- (305.9000,466.4000) --
          (299.0000,467.2000) -- (290.6000,473.5000) -- (287.6000,470.9000) --
          (277.8000,473.6000) -- (277.3000,469.3000) -- (274.6000,470.9000) --
          (271.5000,469.7000) -- (266.4000,465.3000) -- (268.9000,462.7000) --
          (265.4000,463.3000) -- (256.9000,465.5000) -- cycle;

        % path4778
        \path[draw,color=cFFFFFF,densely dotted] (328.0000,443.7000) -- (322.1000,447.3000) --
          (322.1000,447.3000) -- (320.3000,449.0000) -- (313.6000,450.5000) --
          (305.0000,456.1000) -- (305.9000,466.4000) -- (306.0000,466.4000) --
          (305.5000,469.7000) -- (309.0000,472.8000) -- (306.3000,475.5000) --
          (311.2000,479.5000) -- (318.1000,476.3000) -- (329.6000,482.9000) --
          (335.1000,487.6000) -- (339.8000,500.6000) -- (344.6000,506.3000) --
          (357.1000,507.8000) -- (358.5000,510.9000) -- (364.9000,511.0000) --
          (365.1000,501.2000) -- (372.1000,502.6000) -- (375.4000,496.7000) --
          (379.7000,495.5000) -- (379.3000,493.9000) -- (384.2000,488.0000) --
          (374.7000,483.4000) -- (377.8000,482.1000) -- (380.1000,477.7000) --
          (369.5000,472.7000) -- (367.8000,470.0000) -- (368.4000,460.2000) --
          (365.2000,450.3000) -- (360.4000,445.3000) -- (360.4000,445.2000) --
          (357.5000,439.4000) -- (358.0000,436.0000) -- (355.4000,434.3000) --
          (354.0000,429.5000) -- (348.1000,425.1000) -- (342.8000,429.8000) --
          (339.4000,439.9000) -- (335.1000,445.4000) -- (328.0000,443.7000) -- cycle;

        % path4780
        \path[draw,color=cFFFFFF,densely dotted] (381.4000,424.3000) -- (378.9000,422.4000) --
          (377.7000,423.5000) -- (370.1000,429.9000) -- (368.1000,427.3000) --
          (360.4000,445.2000) -- (360.4000,445.3000) -- (365.2000,450.3000) --
          (368.4000,460.2000) -- (367.8000,470.0000) -- (369.5000,472.7000) --
          (380.1000,477.7000) -- (388.9000,481.6000) -- (393.2000,481.0000) --
          (393.4000,477.2000) -- (403.2000,479.8000) -- (410.2000,477.7000) --
          (408.6000,474.4000) -- (410.0000,471.1000) -- (405.5000,462.5000) --
          (409.6000,459.5000) -- (410.8000,457.2000) -- (406.1000,447.9000) --
          (403.8000,437.8000) -- (393.6000,429.1000) -- (390.5000,428.8000) --
          (390.3000,432.1000) -- (383.1000,430.7000) -- (381.4000,424.3000) -- cycle;

        % path4782
        \path[draw,color=cFFFFFF,densely dotted] (409.9000,502.5000) -- (409.7000,499.2000) --
          (402.1000,497.6000) -- (403.1000,494.6000) -- (401.0000,491.6000) --
          (396.6000,492.3000) -- (391.9000,496.7000) -- (391.7000,499.9000) --
          (385.8000,497.1000) -- (383.3000,498.9000) -- (382.9000,495.7000) --
          (379.7000,495.6000) -- (379.7000,495.5000) -- (375.4000,496.7000) --
          (372.1000,502.6000) -- (365.1000,501.2000) -- (364.9000,511.0000) --
          (358.5000,510.9000) .. controls (356.0000,512.5000) and (353.5000,514.1000) ..
          (350.9000,515.8000) -- (344.1000,513.4000) -- (344.9000,524.2000) --
          (343.9000,527.6000) -- (340.1000,529.9000) -- (339.5000,533.0000) --
          (343.5000,538.7000) -- (346.3000,536.8000) -- (347.8000,540.1000) --
          (351.3000,539.2000) -- (355.5000,533.2000) -- (356.7000,536.5000) --
          (372.3000,541.7000) -- (374.2000,544.0000) -- (380.3000,539.5000) --
          (387.9000,539.5000) -- (395.3000,531.3000) -- (407.6000,522.4000) --
          (417.1000,519.7000) -- (417.2000,519.8000) -- (421.8000,513.6000) --
          (420.2000,510.3000) -- (413.3000,502.8000) -- (409.9000,502.5000) -- cycle;

        % path4784
        \path[draw,color=cFFFFFF,densely dotted] (428.3000,519.4000) -- (424.0000,526.1000) --
          (440.0000,526.6000) -- (442.3000,529.3000) -- (441.7000,532.4000) --
          (445.8000,533.4000) -- (456.7000,532.2000) -- (454.3000,529.6000) --
          (455.7000,526.8000) -- (458.8000,527.0000) -- (463.9000,533.2000) --
          (478.8000,531.5000) -- (483.2000,541.3000) -- (494.1000,544.0000) --
          (497.2000,543.1000) -- (497.3000,543.2000) -- (497.8000,539.5000) --
          (500.6000,537.3000) -- (498.1000,530.2000) -- (499.4000,527.2000) --
          (502.5000,526.7000) -- (496.2000,516.5000) -- (498.5000,510.6000) --
          (502.2000,509.6000) -- (500.0000,505.3000) -- (500.0000,505.3000) --
          (492.5000,510.0000) -- (488.9000,510.3000) -- (475.4000,505.0000) --
          (471.5000,505.7000) -- (466.4000,503.1000) -- (460.3000,496.7000) --
          (448.6000,493.4000) -- (444.2000,498.1000) -- (442.0000,510.0000) --
          (438.4000,509.0000) -- (433.6000,514.4000) -- (434.8000,517.4000) --
          (428.3000,519.4000) -- cycle(462.5000,523.9000) -- (463.1000,520.2000) --
          (467.7000,520.2000) -- (470.0000,524.7000) -- (473.7000,523.2000) --
          (471.3000,527.5000) -- (467.7000,528.7000) -- (462.5000,523.9000) -- cycle;

        % path4786
        \path[fill=cE3ECF6] (463.1000,520.2000) -- (462.5000,523.9000) --
          (467.7000,528.7000) -- (471.3000,527.5000) -- (473.7000,523.2000) --
          (470.0000,524.7000) -- (467.7000,520.2000) -- (463.1000,520.2000) -- cycle;

        % path4788
        \path[draw,color=cFFFFFF,densely dotted] (344.6000,506.3000) -- (339.8000,500.6000) --
          (335.1000,487.6000) -- (329.6000,482.9000) -- (318.1000,476.3000) --
          (311.2000,479.5000) -- (300.8000,483.1000) -- (297.9000,481.9000) --
          (294.4000,483.1000) -- (296.5000,485.6000) -- (295.4000,488.6000) --
          (289.9000,492.2000) -- (289.4000,495.3000) -- (289.4000,495.3000) --
          (291.0000,502.4000) -- (293.9000,503.8000) -- (294.1000,507.0000) --
          (296.8000,509.3000) -- (295.9000,515.3000) -- (312.0000,525.8000) --
          (313.1000,529.0000) -- (315.1000,531.6000) -- (322.2000,532.6000) --
          (323.4000,528.9000) -- (326.7000,528.3000) -- (340.1000,529.9000) --
          (343.9000,527.6000) -- (344.9000,524.2000) -- (344.1000,513.4000) --
          (350.9000,515.8000) .. controls (353.5000,514.1000) and (356.0000,512.5000) ..
          (358.5000,510.9000) -- (357.1000,507.8000) -- (344.6000,506.3000) -- cycle;

        % path4790
        \path[draw,color=cFFFFFF,densely dotted] (339.5000,533.0000) -- (340.1000,529.9000) --
          (326.7000,528.3000) -- (323.4000,528.9000) -- (322.2000,532.6000) --
          (315.1000,531.6000) -- (313.1000,529.0000) -- (312.6000,529.9000) --
          (306.7000,531.8000) -- (305.2000,528.9000) -- (295.7000,540.1000) --
          (298.2000,546.3000) -- (308.0000,550.6000) -- (310.4000,556.5000) --
          (310.6000,560.0000) -- (308.2000,562.5000) -- (310.9000,565.3000) --
          (309.6000,568.7000) -- (305.9000,569.4000) -- (306.9000,575.8000) --
          (310.7000,577.5000) -- (314.3000,576.1000) -- (319.8000,582.4000) --
          (329.5000,578.0000) -- (327.9000,571.3000) -- (330.9000,569.7000) --
          (346.1000,570.8000) -- (355.4000,564.7000) -- (364.3000,570.1000) --
          (364.8000,559.1000) -- (371.2000,547.1000) -- (374.2000,544.0000) --
          (372.3000,541.7000) -- (356.7000,536.5000) -- (355.5000,533.2000) --
          (351.3000,539.2000) -- (347.8000,540.1000) -- (346.3000,536.8000) --
          (343.5000,538.7000) -- (339.5000,533.0000) -- cycle;

        % path4792
        \path[draw,color=cFFFFFF,densely dotted] (361.6000,595.4000) -- (371.3000,597.9000) --
          (364.8000,589.9000) -- (364.3000,570.1000) -- (355.4000,564.7000) --
          (346.1000,570.8000) -- (330.9000,569.7000) -- (327.9000,571.3000) --
          (329.5000,578.0000) -- (319.8000,582.4000) -- (311.5000,582.6000) --
          (310.0000,585.5000) -- (300.1000,588.5000) -- (298.0000,593.9000) --
          (307.8000,596.9000) -- (310.5000,603.2000) -- (314.1000,603.5000) --
          (319.1000,599.0000) -- (322.6000,598.8000) -- (333.3000,601.2000) --
          (338.4000,605.2000) -- (345.5000,604.7000) -- (345.5000,601.4000) --
          (354.6000,596.6000) -- (361.6000,595.4000) -- cycle;

        % path4794
        \path[draw,color=cFFFFFF,densely dotted] (294.1000,507.0000) -- (293.9000,503.8000) --
          (291.0000,502.4000) -- (289.4000,495.3000) -- (276.3000,500.2000) --
          (278.8000,502.8000) -- (272.2000,505.6000) -- (270.4000,503.0000) --
          (259.2000,504.0000) -- (269.4000,519.0000) -- (262.1000,525.8000) --
          (261.3000,532.3000) -- (248.6000,529.5000) -- (244.6000,535.5000) --
          (241.1000,536.7000) -- (233.2000,549.3000) -- (238.3000,553.7000) --
          (237.7000,557.7000) -- (241.1000,557.4000) -- (242.1000,560.4000) --
          (239.1000,566.3000) -- (232.9000,570.0000) -- (233.6000,578.7000) --
          (240.7000,579.1000) -- (243.2000,576.7000) -- (242.8000,569.3000) --
          (245.6000,567.8000) -- (253.2000,570.4000) -- (252.1000,567.2000) --
          (256.4000,561.2000) -- (259.6000,560.2000) -- (259.6000,553.5000) --
          (263.9000,549.2000) -- (267.0000,547.9000) -- (274.0000,552.5000) --
          (273.6000,546.2000) -- (279.7000,544.3000) -- (276.2000,539.0000) --
          (279.4000,537.7000) -- (283.3000,543.9000) -- (286.4000,543.4000) --
          (287.1000,540.2000) -- (293.5000,542.7000) -- (295.7000,540.1000) --
          (305.2000,528.9000) -- (306.7000,531.8000) -- (312.6000,529.9000) --
          (313.1000,529.0000) -- (312.0000,525.8000) -- (295.9000,515.3000) --
          (296.8000,509.3000) -- (294.1000,507.0000) -- cycle;

        % path4796
        \path[draw,color=cFFFFFF,densely dotted] (256.9000,496.3000) -- (251.3000,491.7000) --
          (254.9000,486.5000) -- (249.0000,484.6000) -- (246.0000,486.1000) --
          (236.3000,484.9000) -- (230.7000,488.2000) -- (221.4000,487.4000) --
          (219.8000,490.2000) -- (215.6000,489.5000) -- (215.6000,489.5000) --
          (215.3000,492.8000) -- (212.4000,494.2000) -- (211.7000,490.9000) --
          (201.2000,493.9000) -- (201.8000,501.1000) -- (197.0000,513.5000) --
          (198.8000,516.6000) -- (206.2000,516.9000) -- (206.2000,517.0000) --
          (210.6000,516.4000) -- (212.9000,520.3000) -- (216.5000,521.8000) --
          (219.4000,531.6000) -- (225.6000,530.9000) -- (227.2000,534.2000) --
          (241.1000,536.7000) -- (244.6000,535.5000) -- (248.6000,529.5000) --
          (261.3000,532.3000) -- (262.1000,525.8000) -- (269.4000,519.0000) --
          (259.2000,504.0000) -- (256.9000,496.3000) -- cycle;

        % path4798
        \path[draw,color=cFFFFFF,densely dotted] (279.4000,537.7000) -- (276.2000,539.0000) --
          (279.7000,544.3000) -- (273.6000,546.2000) -- (274.0000,552.5000) --
          (267.0000,547.9000) -- (263.9000,549.2000) -- (259.6000,553.5000) --
          (259.6000,560.2000) -- (256.4000,561.2000) -- (252.1000,567.2000) --
          (253.2000,570.4000) -- (264.3000,573.3000) -- (269.8000,578.2000) --
          (276.9000,577.8000) -- (283.3000,586.2000) -- (284.5000,583.0000) --
          (288.2000,582.8000) -- (300.1000,588.5000) -- (300.1000,588.5000) --
          (310.0000,585.5000) -- (311.5000,582.6000) -- (319.8000,582.4000) --
          (314.3000,576.1000) -- (310.7000,577.5000) -- (306.9000,575.8000) --
          (305.9000,569.4000) -- (309.6000,568.7000) -- (310.9000,565.3000) --
          (308.2000,562.5000) -- (310.6000,560.0000) -- (310.4000,556.5000) --
          (308.0000,550.6000) -- (298.2000,546.3000) -- (295.7000,540.1000) --
          (293.5000,542.7000) -- (287.1000,540.2000) -- (286.4000,543.4000) --
          (283.3000,543.9000) -- (279.4000,537.7000) -- cycle;

        % path4800
        \path[draw,color=cFFFFFF,densely dotted] (237.7000,557.7000) -- (238.3000,553.7000) --
          (233.2000,549.3000) -- (241.1000,536.7000) -- (227.2000,534.2000) --
          (225.6000,530.9000) -- (219.4000,531.6000) -- (216.5000,521.8000) --
          (212.9000,520.3000) -- (210.6000,516.4000) -- (206.2000,517.0000) --
          (209.0000,523.5000) -- (207.4000,526.6000) -- (210.6000,526.6000) --
          (211.1000,534.7000) -- (208.1000,536.3000) -- (209.0000,539.8000) --
          (200.0000,550.1000) -- (200.4000,553.5000) -- (196.7000,554.4000) --
          (194.9000,557.5000) -- (193.5000,567.2000) -- (193.5000,567.3000) --
          (206.0000,578.4000) -- (219.8000,576.5000) -- (222.6000,578.9000) --
          (233.6000,578.7000) -- (232.9000,570.0000) -- (239.1000,566.3000) --
          (242.1000,560.4000) -- (241.1000,557.4000) -- (237.7000,557.7000) -- cycle;

        % path4802
        \path[draw,color=cFFFFFF,densely dotted] (198.8000,516.6000) -- (189.1000,518.3000) --
          (185.5000,517.0000) -- (180.0000,519.1000) -- (177.7000,516.9000) --
          (175.0000,518.6000) -- (172.2000,516.0000) -- (161.2000,518.3000) --
          (159.3000,520.8000) -- (156.5000,519.1000) -- (153.5000,520.7000) --
          (152.1000,517.5000) -- (139.8000,520.3000) -- (134.1000,517.6000) --
          (134.0000,517.7000) -- (128.8000,524.7000) -- (121.8000,529.9000) --
          (128.3000,533.0000) -- (128.8000,536.5000) -- (131.3000,534.2000) --
          (138.8000,535.7000) -- (140.4000,538.8000) -- (138.3000,545.5000) --
          (135.3000,548.1000) -- (136.8000,551.3000) -- (141.1000,552.3000) --
          (141.7000,548.2000) -- (145.0000,547.2000) -- (145.2000,551.0000) --
          (151.8000,554.5000) -- (162.7000,558.7000) -- (169.7000,558.4000) --
          (172.6000,563.6000) -- (181.4000,570.8000) -- (184.1000,569.0000) --
          (186.7000,570.6000) -- (193.3000,567.2000) -- (193.5000,567.2000) --
          (194.9000,557.5000) -- (196.7000,554.4000) -- (200.4000,553.5000) --
          (200.0000,550.1000) -- (209.0000,539.8000) -- (208.1000,536.3000) --
          (211.1000,534.7000) -- (210.6000,526.6000) -- (207.4000,526.6000) --
          (209.0000,523.5000) -- (206.2000,517.0000) -- (206.2000,516.9000) --
          (198.8000,516.6000) -- cycle(203.5000,537.7000) -- (205.9000,536.5000) --
          (206.0000,539.5000) -- (203.5000,537.7000) -- cycle;

        % path4804
        \path[draw,color=cFFFFFF,densely dotted] (205.9000,536.5000) -- (203.5000,537.7000) --
          (206.0000,539.5000) -- (205.9000,536.5000) -- cycle;

        % path4806
        \path[draw,color=cFFFFFF,densely dotted] (550.1000,501.9000) -- (543.9000,498.7000) --
          (533.0000,499.1000) -- (531.6000,503.0000) -- (525.0000,499.0000) --
          (513.6000,508.6000) -- (509.1000,503.1000) -- (506.1000,505.1000) --
          (500.2000,504.9000) -- (500.0000,505.3000) -- (502.2000,509.6000) --
          (498.5000,510.6000) -- (496.2000,516.5000) -- (502.5000,526.7000) --
          (499.4000,527.2000) -- (498.1000,530.2000) -- (500.6000,537.3000) --
          (497.8000,539.5000) -- (497.3000,543.2000) -- (504.2000,547.2000) --
          (503.9000,550.4000) -- (513.9000,547.5000) -- (520.5000,549.3000) --
          (521.8000,552.5000) -- (524.1000,546.2000) -- (532.4000,547.7000) --
          (532.5000,544.4000) -- (538.9000,543.0000) -- (541.9000,540.2000) --
          (546.1000,540.7000) -- (548.6000,534.6000) -- (544.6000,532.7000) --
          (551.6000,523.6000) -- (557.6000,522.8000) -- (559.9000,518.8000) --
          (557.4000,514.3000) -- (558.3000,511.1000) -- (552.0000,508.1000) --
          (550.1000,501.9000) -- cycle;

        % path4808
        \path[draw,color=cFFFFFF,densely dotted] (154.5000,335.3000) -- (148.4000,331.5000) --
          (146.2000,332.1000) -- (146.3000,332.2000) -- (146.2000,332.2000) --
          (151.2000,336.7000) -- (154.5000,335.3000) -- cycle;

        % path4810
        \path[draw,color=cFFFFFF,densely dotted] (144.6000,332.7000) -- (145.3000,332.4000) --
          (146.1000,332.3000) -- (143.5000,329.9000) -- (144.6000,332.7000) -- cycle;

        % path4812
        \path[draw,color=cFFFFFF,densely dotted] (150.6000,345.8000) -- (148.7000,348.9000) --
          (155.9000,361.4000) -- (157.4000,357.8000) -- (156.3000,349.5000) --
          (150.6000,345.8000) -- cycle;

        % path4814
        \path[draw,color=cFFFFFF,densely dotted] (112.3000,277.7000) -- (111.2000,274.2000) --
          (109.1000,276.8000) -- (115.9000,283.3000) -- (115.7000,279.8000) --
          (112.3000,277.7000) -- cycle;

        % path4816
        \path[draw,color=cFFFFFF,densely dotted] (68.8000,248.0000) -- (65.6000,248.4000) --
          (66.5000,252.2000) -- (70.1000,253.2000) -- (74.5000,251.9000) --
          (68.8000,248.0000) -- cycle;

        % path4818
        \path[draw,color=cFFFFFF,densely dotted] (569.9000,566.5000) -- (568.0000,556.2000) --
          (564.8000,555.0000) -- (562.1000,557.3000) -- (562.6000,576.9000) --
          (553.0000,575.4000) -- (549.0000,580.8000) -- (537.0000,585.6000) --
          (536.1000,588.8000) -- (532.9000,589.1000) -- (530.4000,592.3000) --
          (530.7000,596.2000) -- (526.3000,601.8000) -- (526.3000,602.0000) --
          (541.6000,604.5000) -- (543.0000,608.3000) -- (549.7000,612.1000) --
          (556.4000,618.7000) -- (557.1000,622.3000) -- (560.0000,623.6000) --
          (564.1000,635.9000) -- (568.5000,636.5000) -- (572.1000,634.6000) --
          (572.0000,634.3000) -- (572.1000,627.3000) -- (578.1000,614.0000) --
          (573.9000,584.8000) -- (569.3000,578.4000) -- (569.9000,566.5000) -- cycle;

        % path4820
        \path[draw,color=cFFFFFF,densely dotted] (526.3000,602.0000) -- (525.7000,605.1000) --
          (528.6000,604.0000) -- (532.9000,608.7000) -- (527.1000,612.9000) --
          (528.3000,619.1000) -- (537.1000,624.0000) -- (534.2000,625.8000) --
          (530.9000,635.2000) -- (539.0000,632.4000) -- (540.0000,638.8000) --
          (537.0000,646.7000) -- (547.7000,648.8000) -- (542.2000,652.5000) --
          (542.0000,655.9000) -- (547.3000,660.8000) -- (559.9000,664.7000) --
          (561.3000,667.8000) -- (564.7000,669.2000) -- (571.3000,652.6000) --
          (567.6000,652.3000) -- (570.5000,650.1000) -- (572.9000,644.0000) --
          (572.1000,634.6000) -- (568.5000,636.5000) -- (564.1000,635.9000) --
          (560.0000,623.6000) -- (557.1000,622.3000) -- (556.4000,618.7000) --
          (549.7000,612.1000) -- (543.0000,608.3000) -- (541.6000,604.5000) --
          (526.3000,602.0000) -- cycle;

        % line4822
        \path[draw=cBCBD43,line join=round,line cap=round,very thin,densely dotted]
          (201.8000,227.7000) -- (201.7000,227.7000);

        % polyline4824
        \path[draw=c999999,line join=round,line cap=round,very thin,densely dotted]
          (450.0000,163.8000) -- (449.5000,159.2000) -- (445.9000,154.1000) --
          (447.2000,147.5000) -- (450.6000,146.4000) -- (451.3000,141.9000) --
          (448.4000,140.4000) -- (449.5000,135.9000) -- (447.0000,125.7000);

        % path4826
        \path[draw=c999999,line join=round,line cap=round,very thin,densely dotted]
          (435.8000,174.8000) -- (432.1000,173.5000) -- (429.3000,175.2000) .. controls
          (427.1000,174.4000) and (425.1000,173.5000) .. (422.9000,172.7000) --
          (420.0000,170.4000) -- (419.6000,163.6000) -- (416.1000,163.0000) --
          (408.8000,164.0000) -- (398.2000,172.5000) -- (398.2000,176.2000) --
          (388.1000,174.9000) -- (384.4000,169.9000);

        % polyline4828
        \path[draw=c999999,line join=round,line cap=round,very thin,densely dotted]
          (450.0000,163.8000) -- (440.4000,166.3000) -- (442.9000,169.0000) --
          (440.3000,171.6000) -- (441.1000,174.8000) -- (435.8000,174.8000);

        % polyline4830
        \path[draw=c999999,line join=round,line cap=round,very thin,densely dotted]
          (472.8000,181.2000) -- (453.8000,169.6000) -- (450.0000,163.8000);

        % polyline4832
        \path[draw=c999999,line join=round,line cap=round,very thin,densely dotted]
          (438.4000,210.2000) -- (442.4000,211.0000) -- (443.0000,214.0000) --
          (445.9000,215.2000) -- (445.5000,218.4000) -- (448.8000,218.7000) --
          (450.8000,221.2000) -- (452.0000,224.7000) -- (448.8000,227.4000) --
          (451.4000,232.8000) -- (461.8000,235.0000) -- (465.1000,236.8000) --
          (465.1000,240.0000) -- (470.5000,237.8000) -- (471.9000,235.0000) --
          (479.2000,230.7000) -- (481.7000,232.7000) -- (485.2000,231.5000) --
          (485.5000,222.2000) -- (487.9000,219.4000) -- (491.1000,219.9000) --
          (493.6000,216.3000) -- (493.4000,214.6000) -- (493.4000,214.6000) --
          (491.9000,211.8000) -- (488.9000,213.0000) -- (488.7000,209.4000) --
          (482.5000,204.2000) -- (484.5000,194.6000) -- (470.1000,183.9000) --
          (472.8000,181.2000);

        % polyline4834
        \path[draw=c999999,line join=round,line cap=round,very thin,densely dotted]
          (435.8000,174.8000) -- (435.1000,178.6000) -- (437.8000,184.5000) --
          (444.1000,187.8000) -- (444.7000,198.9000) -- (444.5000,202.0000) --
          (438.3000,204.8000) -- (438.4000,210.2000);

        % polyline4836
        \path[draw=c999999,line join=round,line cap=round,very thin,densely dotted]
          (438.4000,210.2000) -- (438.4000,210.2000) -- (434.1000,209.6000) --
          (428.9000,214.0000) -- (422.0000,215.1000) -- (419.5000,217.3000);

        % polyline4838
        \path[draw=c999999,line join=round,line cap=round,very thin,densely dotted]
          (486.2000,175.4000) -- (484.0000,178.0000) -- (480.6000,177.4000) --
          (472.8000,181.2000);

        % polyline4840
        \path[draw=c999999,line join=round,line cap=round,very thin,densely dotted]
          (537.4000,220.2000) -- (528.0000,212.7000) -- (522.3000,215.2000) --
          (516.8000,211.1000) -- (510.4000,213.0000) -- (502.5000,208.2000) --
          (493.4000,214.6000);

        % polyline4842
        \path[draw=c999999,line join=round,line cap=round,very thin,densely dotted]
          (569.3000,132.4000) -- (569.3000,132.5000) -- (567.8000,137.2000) --
          (566.2000,140.1000) -- (556.1000,140.4000) -- (546.0000,135.2000) --
          (545.0000,132.1000) -- (539.5000,144.2000) -- (545.6000,147.0000) --
          (543.8000,149.8000) -- (546.4000,151.6000) -- (548.0000,148.8000) --
          (551.1000,149.2000) -- (556.1000,153.0000) -- (552.7000,168.0000) --
          (550.2000,169.9000) -- (546.3000,169.6000);

        % polyline4844
        \path[draw=c999999,line join=round,line cap=round,very thin,densely dotted]
          (570.8000,196.6000) -- (566.2000,194.4000) -- (565.8000,191.4000) --
          (556.3000,185.0000) -- (552.8000,184.8000);

        % polyline4846
        \path[draw=c999999,line join=round,line cap=round,very thin,densely dotted]
          (548.4000,171.0000) -- (547.0000,182.3000) -- (552.8000,184.8000);

        % polyline4848
        \path[draw=c999999,line join=round,line cap=round,very thin,densely dotted]
          (552.8000,184.8000) -- (551.6000,185.8000) -- (546.9000,201.1000) --
          (541.6000,209.8000) -- (541.5000,216.8000) -- (538.4000,219.4000);

        % polyline4850
        \path[draw=c999999,line join=round,line cap=round,very thin,densely dotted]
          (537.4000,220.2000) -- (534.8000,226.5000) -- (537.6000,237.6000);

        % polyline4852
        \path[draw=c999999,line join=round,line cap=round,very thin,densely dotted]
          (537.6000,237.6000) -- (537.6000,237.6000) -- (544.4000,240.6000) --
          (542.9000,243.4000) -- (544.2000,245.9000) -- (544.4000,246.1000) --
          (548.6000,241.4000) -- (548.6000,241.2000) -- (553.0000,240.7000) --
          (550.6000,235.0000) -- (547.5000,235.2000) -- (547.0000,225.1000) --
          (538.4000,219.4000);

        % line4854
        \path[draw=c999999,line join=round,line cap=round,very thin,densely dotted]
          (538.4000,219.4000) -- (537.4000,220.2000);

        % polyline4856
        \path[draw=c999999,line join=round,line cap=round,very thin,densely dotted]
          (537.6000,237.6000) -- (537.5000,237.6000) -- (530.8000,236.8000) --
          (530.3000,239.9000) -- (526.7000,240.4000) -- (524.6000,242.8000) --
          (521.5000,241.9000) -- (515.2000,244.0000) -- (511.9000,249.3000) --
          (486.8000,260.8000);

        % polyline4858
        \path[draw=c999999,line join=round,line cap=round,very thin,densely dotted]
          (544.3000,360.6000) -- (544.1000,361.0000) -- (541.6000,362.7000) --
          (534.8000,360.2000) -- (532.3000,358.1000) -- (532.9000,354.7000) --
          (530.1000,352.9000) -- (519.6000,368.6000) -- (514.4000,364.6000) --
          (510.8000,363.6000) -- (507.8000,365.4000) -- (504.5000,363.5000) --
          (503.1000,360.7000) -- (500.0000,359.7000) -- (497.0000,352.2000);

        % polyline4860
        \path[draw=c999999,line join=round,line cap=round,very thin,densely dotted]
          (486.8000,260.8000) -- (480.7000,260.7000) -- (478.1000,258.3000);

        % polyline4862
        \path[draw=c999999,line join=round,line cap=round,very thin,densely dotted]
          (509.4000,308.5000) -- (506.0000,305.0000) -- (510.0000,296.6000) --
          (512.4000,294.0000) -- (507.7000,289.5000) -- (504.4000,288.9000) --
          (498.0000,277.6000) -- (494.5000,278.5000) -- (492.8000,275.6000) --
          (490.6000,278.1000) -- (489.3000,275.2000) -- (492.6000,269.4000) --
          (486.8000,260.8000);

        % polyline4864
        \path[draw=c999999,line join=round,line cap=round,very thin,densely dotted]
          (478.1000,258.3000) -- (473.7000,249.2000) -- (470.8000,247.5000) --
          (476.1000,243.0000) -- (475.9000,239.4000) -- (473.6000,236.0000) --
          (470.5000,237.8000);

        % polyline4866
        \path[draw=c999999,line join=round,line cap=round,very thin,densely dotted]
          (478.1000,258.3000) -- (476.8000,259.8000) -- (474.8000,269.1000) --
          (467.8000,276.6000) -- (466.6000,281.2000);

        % polyline4868
        \path[draw=c999999,line join=round,line cap=round,very thin,densely dotted]
          (466.6000,281.2000) -- (466.6000,281.2000) -- (449.7000,282.8000) --
          (439.8000,287.6000) -- (433.6000,280.5000) -- (427.4000,278.4000) --
          (425.9000,275.4000) -- (419.5000,273.5000) -- (416.8000,270.7000);

        % polyline4870
        \path[draw=c999999,line join=round,line cap=round,very thin,densely dotted]
          (470.6000,318.0000) -- (473.7000,316.8000) -- (472.8000,310.8000) --
          (476.0000,305.2000) -- (473.5000,295.5000) -- (470.7000,293.7000) --
          (476.3000,289.5000) -- (467.1000,284.6000) -- (466.6000,281.2000);

        % polyline4872
        \path[draw=c999999,line join=round,line cap=round,very thin,densely dotted]
          (470.6000,318.0000) -- (465.2000,314.3000) -- (459.1000,315.7000) --
          (452.5000,314.1000) -- (445.9000,337.5000);

        % polyline4874
        \path[draw=c999999,line join=round,line cap=round,very thin,densely dotted]
          (506.6000,318.7000) -- (501.3000,326.1000) -- (495.2000,329.7000) --
          (490.8000,329.9000) -- (490.3000,326.8000) -- (486.6000,325.0000) --
          (482.1000,329.9000) -- (478.8000,330.1000) -- (478.7000,326.7000) --
          (475.4000,326.7000) -- (470.6000,318.0000);

        % polyline4876
        \path[draw=c999999,line join=round,line cap=round,very thin,densely dotted]
          (462.2000,362.1000) -- (455.0000,362.7000) -- (450.9000,356.3000) --
          (447.7000,356.9000) -- (444.2000,346.5000) -- (445.9000,337.5000);

        % polyline4878
        \path[draw=c999999,line join=round,line cap=round,very thin,densely dotted]
          (445.9000,337.5000) -- (439.5000,328.9000) -- (437.7000,331.5000) --
          (434.9000,329.8000) -- (432.4000,331.7000) -- (430.0000,329.6000) --
          (427.3000,331.8000) -- (427.0000,335.1000);

        % polyline4880
        \path[draw=c999999,line join=round,line cap=round,very thin,densely dotted]
          (497.0000,352.2000) -- (497.0000,352.1000) -- (496.1000,341.7000) --
          (499.4000,341.7000) -- (502.5000,338.4000);

        % polyline4882
        \path[draw=c999999,line join=round,line cap=round,very thin,densely dotted]
          (488.0000,374.9000) -- (495.3000,386.8000) -- (502.2000,387.0000) --
          (502.6000,383.8000) -- (513.2000,386.2000) -- (516.2000,393.3000) --
          (513.6000,399.4000) -- (514.8000,405.5000) -- (520.5000,407.2000);

        % polyline4884
        \path[draw=c999999,line join=round,line cap=round,very thin,densely dotted]
          (497.0000,352.2000) -- (494.4000,367.4000) -- (488.0000,374.9000);

        % polyline4886
        \path[draw=c999999,line join=round,line cap=round,very thin,densely dotted]
          (538.4000,407.6000) -- (538.3000,407.6000) -- (531.5000,408.9000) --
          (529.3000,411.1000) -- (520.5000,407.2000);

        % polyline4888
        \path[draw=c999999,line join=round,line cap=round,very thin,densely dotted]
          (520.5000,407.2000) -- (518.8000,411.6000) -- (520.8000,416.1000) --
          (525.7000,419.6000) -- (525.3000,426.2000) -- (521.8000,425.0000) --
          (507.7000,429.3000) -- (506.6000,432.4000) -- (500.5000,433.8000) --
          (499.4000,437.3000);

        % polyline4890
        \path[draw=c999999,line join=round,line cap=round,very thin,densely dotted]
          (446.3000,387.4000) -- (451.3000,383.0000) -- (449.2000,380.2000) --
          (452.0000,377.4000) -- (459.7000,375.7000) -- (462.6000,370.1000) --
          (462.2000,362.1000);

        % polyline4892
        \path[draw=c999999,line join=round,line cap=round,very thin,densely dotted]
          (488.0000,374.9000) -- (475.5000,357.9000) -- (472.3000,358.6000) --
          (469.0000,364.4000) -- (462.2000,362.1000);

        % polyline4894
        \path[draw=c999999,line join=round,line cap=round,very thin,densely dotted]
          (427.0000,335.1000) -- (427.0000,335.1000) -- (429.3000,338.5000) --
          (423.9000,342.3000) -- (422.1000,352.0000) -- (427.9000,359.6000) --
          (426.0000,366.1000) -- (428.5000,368.0000) -- (427.1000,370.8000) --
          (428.4000,374.1000) -- (433.6000,378.8000) -- (440.2000,381.2000) --
          (441.2000,384.6000) -- (446.3000,387.4000);

        % polyline4896
        \path[draw=c999999,line join=round,line cap=round,very thin,densely dotted]
          (433.4000,403.1000) -- (437.2000,402.1000) -- (440.9000,396.7000) --
          (446.4000,393.6000) -- (446.5000,388.6000) -- (446.3000,387.4000) --
          (446.3000,387.4000);

        % line4898
        \path[draw=c999999,line join=round,line cap=round,very thin,densely dotted]
          (448.8000,398.3000) -- (446.4000,393.6000);

        % polyline4900
        \path[draw=c999999,line join=round,line cap=round,very thin,densely dotted]
          (486.2000,175.4000) -- (492.6000,176.1000) -- (492.5000,179.4000) --
          (497.1000,184.8000) -- (503.6000,184.0000) -- (504.8000,180.9000) --
          (510.8000,178.7000) -- (513.5000,180.6000) -- (520.3000,179.5000) --
          (526.2000,175.3000) -- (528.0000,178.2000) -- (534.8000,180.1000) --
          (546.7000,171.5000) -- (548.4000,171.0000) -- (546.3000,169.6000);

        % polyline4902
        \path[draw=c999999,line join=round,line cap=round,very thin,densely dotted]
          (433.4000,403.1000) -- (432.3000,407.7000) -- (432.3000,411.1000) --
          (429.4000,409.8000) -- (427.8000,415.7000) -- (424.8000,418.2000) --
          (425.6000,421.4000) -- (421.7000,421.8000) -- (417.9000,428.5000) --
          (411.3000,430.5000) -- (403.8000,437.8000);

        % polyline4904
        \path[draw=c999999,line join=round,line cap=round,very thin,densely dotted]
          (443.3000,465.3000) -- (445.6000,449.2000) -- (449.0000,442.6000) --
          (448.0000,434.6000) -- (451.9000,429.5000) -- (453.5000,422.6000) --
          (449.3000,409.9000) -- (448.8000,398.3000);

        % polyline4906
        \path[draw=c999999,line join=round,line cap=round,very thin,densely dotted]
          (499.4000,437.3000) -- (492.8000,436.3000) -- (486.7000,431.9000) --
          (483.4000,432.5000) -- (482.4000,412.8000) -- (479.0000,414.7000) --
          (469.8000,412.6000) -- (466.8000,413.7000) -- (468.0000,406.7000) --
          (464.3000,397.8000) -- (458.8000,394.9000) -- (448.8000,398.3000);

        % polyline4908
        \path[draw=c999999,line join=round,line cap=round,very thin,densely dotted]
          (448.6000,493.4000) -- (453.7000,487.9000) -- (446.6000,480.7000) --
          (446.4000,474.5000) -- (443.4000,469.5000);

        % line4910
        \path[draw=c999999,line join=round,line cap=round,very thin,densely dotted]
          (443.3000,465.3000) -- (443.4000,469.5000);

        % polyline4912
        \path[draw=c999999,line join=round,line cap=round,very thin,densely dotted]
          (443.4000,469.5000) -- (436.7000,465.7000) -- (433.6000,465.7000) --
          (430.7000,468.2000) -- (430.6000,464.6000) -- (422.7000,469.6000) --
          (413.9000,463.9000) -- (410.8000,457.2000);

        % polyline4914
        \path[draw=c999999,line join=round,line cap=round,very thin,densely dotted]
          (486.1000,478.8000) -- (480.3000,476.3000) -- (480.2000,473.0000) --
          (469.7000,472.1000) -- (469.1000,465.7000) -- (466.3000,468.1000) --
          (463.6000,466.5000) -- (453.8000,470.9000) -- (449.9000,466.1000) --
          (443.3000,465.3000);

        % polyline4916
        \path[draw=c999999,line join=round,line cap=round,very thin,densely dotted]
          (459.1000,458.0000) -- (464.2000,461.1000) -- (461.5000,465.1000) --
          (457.0000,466.4000) -- (454.7000,463.9000) -- (456.1000,460.5000) --
          (459.1000,458.0000);

        % polyline4918
        \path[draw=c999999,line join=round,line cap=round,very thin,densely dotted]
          (499.4000,437.3000) -- (499.0000,441.0000) -- (492.5000,441.4000) --
          (489.7000,449.0000) -- (493.1000,451.5000) -- (490.3000,454.4000) --
          (485.6000,452.8000) -- (482.0000,458.4000) -- (483.7000,461.5000) --
          (494.9000,467.8000) -- (494.7000,473.1000);

        % polyline4920
        \path[draw=c999999,line join=round,line cap=round,very thin,densely dotted]
          (555.8000,437.8000) -- (555.6000,438.1000) -- (548.9000,441.8000) --
          (538.7000,453.3000) -- (526.3000,449.5000) -- (522.1000,454.2000) --
          (521.3000,458.5000) -- (517.4000,453.4000) -- (513.6000,454.3000) --
          (507.7000,458.8000) -- (506.1000,468.5000) -- (501.7000,467.2000) --
          (505.6000,472.7000) -- (501.4000,471.5000) -- (494.7000,473.1000);

        % polyline4922
        \path[draw=c999999,line join=round,line cap=round,very thin,densely dotted]
          (494.7000,473.1000) -- (493.6000,473.4000) -- (490.9000,473.5000) --
          (486.1000,478.8000);

        % polyline4924
        \path[draw=c999999,line join=round,line cap=round,very thin,densely dotted]
          (554.1000,458.6000) -- (547.2000,466.3000) -- (545.0000,472.3000) --
          (549.8000,481.9000) -- (557.6000,490.6000) -- (551.7000,489.8000) --
          (545.9000,494.1000) -- (547.3000,497.0000) -- (543.9000,498.7000);

        % polyline4926
        \path[draw=c999999,line join=round,line cap=round,very thin,densely dotted]
          (316.9000,5.1000) -- (324.4000,19.5000) -- (330.6000,22.4000) --
          (329.8000,25.6000) -- (334.4000,30.3000) -- (347.5000,30.8000) --
          (349.4000,28.3000) -- (350.8000,31.2000) -- (349.7000,38.0000) --
          (356.3000,39.1000) -- (360.7000,45.0000) -- (357.2000,46.8000) --
          (361.5000,52.7000) -- (360.8000,55.8000) -- (364.1000,56.2000) --
          (366.2000,58.8000) -- (362.2000,70.9000);

        % polyline4928
        \path[draw=c999999,line join=round,line cap=round,very thin,densely dotted]
          (362.1000,62.7000) -- (361.3000,65.6000) -- (358.2000,66.4000) --
          (362.1000,62.7000);

        % polyline4930
        \path[draw=c999999,line join=round,line cap=round,very thin,densely dotted]
          (363.7000,94.9000) -- (361.1000,87.0000) -- (365.4000,77.9000) --
          (365.9000,73.7000);

        % line4932
        \path[draw=c999999,line join=round,line cap=round,very thin,densely dotted]
          (365.9000,73.7000) -- (362.2000,70.9000);

        % polyline4934
        \path[draw=c999999,line join=round,line cap=round,very thin,densely dotted]
          (365.9000,73.7000) -- (380.0000,73.1000) -- (382.9000,70.6000) --
          (385.6000,72.5000) -- (391.4000,70.5000) -- (409.0000,75.2000) --
          (409.0000,75.3000);

        % polyline4936
        \path[draw=c999999,line join=round,line cap=round,very thin,densely dotted]
          (362.2000,70.9000) -- (352.6000,72.9000) -- (351.6000,69.2000) --
          (349.0000,71.2000) -- (345.8000,66.0000) -- (344.9000,69.1000) --
          (335.5000,65.2000) -- (332.7000,68.7000) -- (330.8000,65.1000) --
          (333.7000,58.7000) -- (320.3000,61.5000) -- (312.2000,55.1000) --
          (311.2000,51.7000) -- (297.4000,50.5000);

        % polyline4938
        \path[draw=c999999,line join=round,line cap=round,very thin,densely dotted]
          (303.8000,91.4000) -- (303.8000,91.5000) -- (308.4000,95.6000) --
          (312.0000,94.4000) -- (328.1000,96.7000) -- (342.5000,104.4000) --
          (352.1000,96.6000) -- (363.7000,94.9000);

        % polyline4940
        \path[draw=c999999,line join=round,line cap=round,very thin,densely dotted]
          (361.9000,135.7000) -- (364.4000,133.9000) -- (363.8000,130.6000) --
          (360.5000,129.4000) -- (357.9000,131.1000) -- (357.0000,127.9000) --
          (360.4000,128.1000) -- (357.2000,122.9000) -- (360.1000,121.1000) --
          (362.8000,114.7000) -- (365.7000,113.4000) -- (363.2000,111.4000) --
          (364.1000,106.4000) -- (363.7000,94.9000);

        % polyline4942
        \path[draw=c999999,line join=round,line cap=round,very thin,densely dotted]
          (405.9000,115.1000) -- (405.5000,99.1000) -- (408.6000,98.7000) --
          (414.3000,90.6000) -- (412.7000,87.2000) -- (414.3000,83.3000) --
          (413.3000,76.9000);

        % polyline4944
        \path[draw=c999999,line join=round,line cap=round,very thin,densely dotted]
          (526.3000,602.0000) -- (541.6000,604.5000) -- (543.0000,608.3000) --
          (549.7000,612.1000) -- (556.4000,618.7000) -- (557.1000,622.3000) --
          (560.0000,623.6000) -- (564.1000,635.9000) -- (568.5000,636.5000) --
          (572.1000,634.6000);

        % polyline4946
        \path[draw=c999999,line join=round,line cap=round,very thin,densely dotted]
          (300.4000,115.5000) -- (285.7000,111.7000) -- (280.3000,119.7000) --
          (271.1000,122.5000) -- (269.5000,125.4000) -- (263.5000,122.8000) --
          (261.2000,120.4000) -- (263.2000,117.7000) -- (257.7000,114.4000) --
          (253.9000,115.2000) -- (245.4000,109.6000) -- (238.0000,112.5000) --
          (236.4000,113.1000) -- (236.4000,113.2000) -- (236.4000,113.3000) --
          (236.8000,123.1000) -- (240.0000,123.1000) -- (237.5000,125.4000) --
          (240.5000,135.9000) -- (238.9000,139.3000) -- (241.6000,143.9000) --
          (240.8000,146.4000);

        % polyline4948
        \path[draw=c999999,line join=round,line cap=round,very thin,densely dotted]
          (285.8000,69.9000) -- (285.9000,70.0000) -- (299.9000,82.0000) --
          (303.8000,91.4000);

        % polyline4950
        \path[draw=c999999,line join=round,line cap=round,very thin,densely dotted]
          (300.4000,115.5000) -- (301.9000,106.5000) -- (300.4000,96.7000) --
          (303.8000,91.4000);

        % polyline4952
        \path[draw=c999999,line join=round,line cap=round,very thin,densely dotted]
          (340.0000,138.5000) -- (327.9000,131.3000) -- (325.3000,133.1000) --
          (315.0000,130.6000) -- (307.9000,132.7000) -- (301.6000,131.5000) --
          (299.9000,127.9000);

        % polyline4954
        \path[draw=c999999,line join=round,line cap=round,very thin,densely dotted]
          (290.5000,147.9000) -- (288.6000,141.4000) -- (291.1000,138.4000) --
          (295.4000,138.5000);

        % polyline4956
        \path[draw=c999999,line join=round,line cap=round,very thin,densely dotted]
          (322.3000,150.3000) -- (322.3000,150.4000) -- (316.2000,143.5000) --
          (306.0000,139.4000) -- (295.4000,138.5000);

        % line4958
        \path[draw=c999999,line join=round,line cap=round,very thin,densely dotted]
          (295.4000,138.5000) -- (299.9000,127.9000);

        % polyline4960
        \path[draw=c999999,line join=round,line cap=round,very thin,densely dotted]
          (299.9000,127.9000) -- (304.3000,124.9000) -- (300.4000,115.5000);

        % polyline4962
        \path[draw=c999999,line join=round,line cap=round,very thin,densely dotted]
          (258.8000,166.1000) -- (261.4000,163.2000) -- (268.2000,162.0000) --
          (274.1000,158.4000) -- (277.3000,160.2000) -- (284.1000,158.7000) --
          (283.8000,155.6000) -- (290.5000,147.9000);

        % polyline4964
        \path[draw=c999999,line join=round,line cap=round,very thin,densely dotted]
          (309.5000,181.5000) -- (306.0000,182.3000) -- (303.5000,179.8000) --
          (302.1000,173.5000) -- (295.2000,167.5000) -- (294.6000,155.3000) --
          (290.5000,147.9000);

        % path4966
        \path[draw=c999999,line join=round,line cap=round,very thin,densely dotted]
          (338.3000,143.4000) .. controls (337.5000,143.9000) and (336.7000,144.5000) ..
          (336.0000,145.0000) .. controls (335.0000,145.7000) and (334.0000,146.4000) ..
          (333.0000,147.1000) -- (330.1000,145.9000) -- (326.4000,147.4000) --
          (322.4000,150.3000) -- (322.3000,150.3000);

        % polyline4968
        \path[draw=c999999,line join=round,line cap=round,very thin,densely dotted]
          (322.3000,150.3000) -- (320.1000,153.8000) -- (323.6000,159.6000);

        % polyline4970
        \path[draw=c999999,line join=round,line cap=round,very thin,densely dotted]
          (339.3000,165.4000) -- (336.7000,162.7000) -- (333.6000,163.7000) --
          (323.7000,159.6000) -- (323.6000,159.6000);

        % polyline4972
        \path[draw=c999999,line join=round,line cap=round,very thin,densely dotted]
          (323.6000,159.6000) -- (315.7000,165.7000) -- (314.0000,168.6000) --
          (314.7000,171.8000) -- (309.5000,181.5000);

        % polyline4974
        \path[draw=c999999,line join=round,line cap=round,very thin,densely dotted]
          (240.8000,146.4000) -- (241.0000,150.0000) -- (248.1000,150.6000) --
          (256.9000,161.2000) -- (254.9000,163.8000) -- (258.8000,166.1000);

        % polyline4976
        \path[draw=c999999,line join=round,line cap=round,very thin,densely dotted]
          (257.4000,199.1000) -- (257.4000,199.1000) -- (256.2000,190.6000) --
          (264.6000,184.9000) -- (264.1000,179.7000) -- (264.2000,174.6000) --
          (258.9000,169.8000) -- (258.8000,166.1000);

        % polyline4978
        \path[draw=c999999,line join=round,line cap=round,very thin,densely dotted]
          (240.8000,146.4000) -- (234.5000,145.3000) -- (226.6000,147.1000) --
          (218.6000,153.2000) -- (215.3000,153.6000) -- (208.8000,152.3000) --
          (206.0000,154.3000) -- (200.4000,150.9000) -- (193.3000,153.5000) --
          (190.2000,152.1000) -- (189.2000,155.0000) -- (183.2000,158.3000);

        % polyline4980
        \path[draw=c999999,line join=round,line cap=round,very thin,densely dotted]
          (259.3000,205.4000) -- (257.1000,207.7000) -- (259.0000,210.3000) --
          (259.4000,216.5000) -- (252.4000,228.4000) -- (248.8000,229.6000) --
          (248.1000,233.5000) -- (248.1000,233.6000) -- (239.5000,238.6000) --
          (236.3000,238.3000) -- (236.9000,241.9000) -- (230.0000,239.0000);

        % polyline4982
        \path[draw=c999999,line join=round,line cap=round,very thin,densely dotted]
          (546.3000,169.6000) -- (544.0000,166.6000) -- (527.9000,160.5000) --
          (521.9000,156.3000) -- (512.4000,154.6000) -- (509.3000,148.8000) --
          (509.8000,145.6000) -- (496.9000,141.1000) -- (491.9000,132.7000) --
          (495.2000,132.5000) -- (494.1000,126.2000) -- (495.8000,122.8000) --
          (494.5000,118.8000) -- (491.7000,117.0000) -- (489.0000,106.4000);

        % polyline4984
        \path[draw=c999999,line join=round,line cap=round,very thin,densely dotted]
          (259.3000,205.4000) -- (259.8000,203.3000) -- (257.4000,199.1000);

        % polyline4986
        \path[draw=c999999,line join=round,line cap=round,very thin,densely dotted]
          (257.4000,199.1000) -- (257.4000,199.1000) -- (253.8000,198.3000) --
          (250.3000,194.3000) -- (247.1000,195.5000) -- (238.9000,189.9000) --
          (237.2000,180.4000) -- (234.1000,178.7000) -- (221.7000,185.3000) --
          (218.3000,185.2000);

        % polyline4988
        \path[draw=c999999,line join=round,line cap=round,very thin,densely dotted]
          (280.9000,269.3000) -- (276.2000,261.8000) -- (273.2000,263.2000) --
          (270.6000,261.1000) -- (271.9000,250.9000) -- (268.1000,245.1000) --
          (268.9000,242.0000) -- (262.1000,238.3000) -- (259.6000,240.3000) --
          (259.9000,234.2000) -- (248.1000,233.6000);

        % polyline4990
        \path[draw=c999999,line join=round,line cap=round,very thin,densely dotted]
          (331.6000,191.1000) -- (327.4000,190.5000) -- (324.3000,192.4000) --
          (322.6000,189.4000) -- (320.4000,192.6000) -- (312.7000,193.4000) --
          (312.6000,193.4000) -- (309.5000,181.5000);

        % polyline4992
        \path[draw=c999999,line join=round,line cap=round,very thin,densely dotted]
          (312.7000,193.4000) -- (312.1000,195.1000) -- (306.7000,207.3000) --
          (297.2000,207.9000) -- (290.4000,211.1000) -- (290.7000,214.2000) --
          (290.6000,214.2000);

        % polyline4994
        \path[draw=c999999,line join=round,line cap=round,very thin,densely dotted]
          (290.6000,214.2000) -- (293.4000,222.3000) -- (291.5000,225.7000) --
          (293.4000,232.1000) -- (296.8000,230.7000) -- (299.4000,232.3000) --
          (300.7000,236.5000) -- (303.9000,237.2000) -- (306.0000,234.7000) --
          (323.1000,236.4000) -- (324.0000,239.4000) -- (318.4000,242.0000) --
          (324.2000,250.6000) -- (317.6000,255.2000) -- (319.9000,261.8000) --
          (309.0000,264.1000) -- (308.3000,267.2000) -- (305.1000,267.1000);

        % polyline4996
        \path[draw=c999999,line join=round,line cap=round,very thin,densely dotted]
          (447.0000,125.7000) -- (424.5000,123.6000) -- (413.4000,116.2000) --
          (405.9000,115.1000);

        % polyline4998
        \path[draw=c999999,line join=round,line cap=round,very thin,densely dotted]
          (381.0000,153.9000) -- (389.6000,142.4000) -- (386.8000,141.0000) --
          (387.8000,138.0000) -- (385.9000,135.2000) -- (392.0000,132.7000) --
          (388.4000,123.7000) -- (389.8000,120.7000) -- (398.8000,116.2000) --
          (404.8000,118.3000) -- (405.9000,115.1000);

        % polyline5000
        \path[draw=c999999,line join=round,line cap=round,very thin,densely dotted]
          (381.0000,153.9000) -- (370.6000,147.1000) -- (366.2000,142.5000) --
          (366.9000,139.3000) -- (364.9000,136.5000) -- (361.9000,135.7000);

        % polyline5002
        \path[draw=c999999,line join=round,line cap=round,very thin,densely dotted]
          (340.0000,138.5000) -- (340.3000,140.8000) -- (338.4000,143.4000) --
          (338.3000,143.4000);

        % polyline5004
        \path[draw=c999999,line join=round,line cap=round,very thin,densely dotted]
          (361.9000,135.7000) -- (352.9000,139.0000) -- (350.0000,137.2000) --
          (347.8000,139.6000) -- (340.0000,138.5000);

        % polyline5006
        \path[draw=c999999,line join=round,line cap=round,very thin,densely dotted]
          (338.3000,143.4000) -- (340.2000,149.2000) -- (338.7000,152.0000) --
          (340.2000,157.4000) -- (340.2000,157.4000) -- (340.5000,160.6000) --
          (339.3000,165.4000);

        % polyline5008
        \path[draw=c999999,line join=round,line cap=round,very thin,densely dotted]
          (339.3000,165.4000) -- (339.8000,166.3000) -- (335.4000,185.8000) --
          (331.6000,191.1000);

        % polyline5010
        \path[draw=c999999,line join=round,line cap=round,very thin,densely dotted]
          (378.2000,185.9000) -- (377.8000,176.3000) -- (381.1000,175.4000) --
          (384.4000,169.9000);

        % polyline5012
        \path[draw=c999999,line join=round,line cap=round,very thin,densely dotted]
          (384.4000,169.9000) -- (381.5000,168.0000) -- (377.6000,158.6000) --
          (380.5000,156.1000) -- (381.0000,153.9000);

        % polyline5014
        \path[draw=c999999,line join=round,line cap=round,very thin,densely dotted]
          (356.3000,201.9000) -- (360.7000,195.8000) -- (361.0000,189.3000) --
          (378.2000,185.9000);

        % polyline5016
        \path[draw=c999999,line join=round,line cap=round,very thin,densely dotted]
          (356.3000,201.9000) -- (343.8000,204.9000) -- (333.5000,204.1000) --
          (336.4000,202.4000) -- (337.0000,199.3000) -- (332.3000,194.5000) --
          (331.6000,191.1000);

        % polyline5018
        \path[draw=c999999,line join=round,line cap=round,very thin,densely dotted]
          (358.5000,242.9000) -- (356.0000,235.1000) -- (352.7000,232.0000) --
          (358.6000,228.2000) -- (360.4000,225.6000) -- (359.1000,222.3000) --
          (364.3000,216.8000) -- (363.9000,210.3000) -- (356.3000,201.9000);

        % polyline5020
        \path[draw=c999999,line join=round,line cap=round,very thin,densely dotted]
          (419.5000,217.3000) -- (416.2000,215.7000) -- (402.1000,217.5000) --
          (396.4000,205.2000) -- (392.4000,200.6000) -- (390.2000,203.1000) --
          (387.3000,193.8000) -- (382.3000,187.5000) -- (378.2000,185.9000);

        % polyline5022
        \path[draw=c999999,line join=round,line cap=round,very thin,densely dotted]
          (411.6000,258.0000) -- (411.7000,258.0000) -- (410.3000,255.0000) --
          (411.8000,245.9000) -- (419.3000,234.2000) -- (417.3000,231.7000) --
          (420.5000,229.3000) -- (420.9000,225.1000) -- (417.4000,220.7000) --
          (419.5000,217.3000);

        % polyline5024
        \path[draw=c999999,line join=round,line cap=round,very thin,densely dotted]
          (129.1000,168.6000) -- (129.0000,168.5000) -- (125.8000,168.2000) --
          (124.1000,165.5000) -- (124.1000,163.7000) -- (124.1000,163.5000) --
          (127.3000,163.5000) -- (129.1000,168.6000);

        % polyline5026
        \path[draw=c999999,line join=round,line cap=round,very thin,densely dotted]
          (129.1000,168.6000) -- (131.4000,170.3000) -- (131.8000,170.6000) --
          (131.9000,176.9000) -- (130.1000,187.0000) -- (120.0000,189.9000) --
          (114.5000,197.8000);

        % polyline5028
        \path[draw=c999999,line join=round,line cap=round,very thin,densely dotted]
          (55.9000,190.1000) -- (58.2000,185.4000) -- (55.4000,175.1000) --
          (58.2000,165.7000) -- (54.2000,160.5000) -- (54.0000,157.4000);

        % polyline5030
        \path[draw=c999999,line join=round,line cap=round,very thin,densely dotted]
          (114.5000,197.8000) -- (111.3000,199.1000) -- (109.6000,195.9000) --
          (105.6000,195.0000) -- (98.7000,203.4000) -- (96.6000,196.7000) --
          (93.3000,197.7000) -- (78.8000,190.4000) -- (70.0000,193.8000) --
          (55.9000,190.1000) -- (55.9000,190.1000);

        % polyline5032
        \path[draw=c999999,line join=round,line cap=round,very thin,densely dotted]
          (55.9000,190.1000) -- (48.2000,193.7000) -- (48.8000,197.7000) --
          (52.6000,203.8000) -- (61.7000,205.3000) -- (61.8000,211.4000) --
          (59.0000,214.6000) -- (55.9000,213.6000);

        % polyline5034
        \path[draw=c999999,line join=round,line cap=round,very thin,densely dotted]
          (206.2000,517.0000) -- (206.2000,516.9000) -- (198.8000,516.6000);

        % polyline5036
        \path[draw=c999999,line join=round,line cap=round,very thin,densely dotted]
          (121.4000,232.8000) -- (120.3000,226.4000) -- (123.6000,225.6000) --
          (120.5000,223.8000) -- (123.2000,221.5000) -- (124.3000,217.9000) --
          (121.6000,215.9000) -- (123.4000,213.0000) -- (122.1000,210.2000) --
          (116.4000,207.1000) -- (115.8000,203.9000) -- (119.7000,202.8000) --
          (115.7000,202.7000) -- (114.5000,197.8000);

        % polyline5038
        \path[draw=c999999,line join=round,line cap=round,very thin,densely dotted]
          (161.5000,224.9000) -- (153.9000,221.0000) -- (143.8000,225.4000) --
          (142.0000,228.5000) -- (127.2000,229.0000) -- (121.4000,232.8000);

        % polyline5040
        \path[draw=c999999,line join=round,line cap=round,very thin,densely dotted]
          (181.7000,175.6000) -- (185.7000,170.6000) -- (187.8000,163.0000) --
          (183.2000,158.3000);

        % polyline5042
        \path[draw=c999999,line join=round,line cap=round,very thin,densely dotted]
          (218.3000,185.2000) -- (218.4000,185.1000) -- (218.4000,181.8000) --
          (214.7000,180.1000) -- (212.5000,173.4000) -- (209.5000,173.1000) --
          (208.7000,176.2000) -- (198.5000,176.2000) -- (190.9000,179.8000) --
          (188.1000,177.9000) -- (185.4000,179.7000) -- (181.7000,175.6000);

        % polyline5044
        \path[draw=c999999,line join=round,line cap=round,very thin,densely dotted]
          (172.3000,116.1000) -- (172.3000,116.2000) -- (173.0000,123.6000) --
          (182.4000,127.1000) -- (180.2000,130.0000) -- (182.7000,139.2000) --
          (173.6000,144.7000) -- (175.3000,147.3000) -- (170.3000,153.2000) --
          (172.3000,156.0000) -- (183.2000,158.3000);

        % polyline5046
        \path[draw=c999999,line join=round,line cap=round,very thin,densely dotted]
          (149.5000,165.6000) -- (154.9000,176.4000) -- (157.9000,177.4000) --
          (165.8000,172.1000) -- (172.0000,174.6000);

        % polyline5048
        \path[draw=c999999,line join=round,line cap=round,very thin,densely dotted]
          (181.7000,175.6000) -- (181.6000,175.6000) -- (172.0000,174.6000);

        % polyline5050
        \path[draw=c999999,line join=round,line cap=round,very thin,densely dotted]
          (172.0000,174.6000) -- (171.9000,188.3000) -- (170.1000,190.9000) --
          (172.3000,210.4000) -- (166.1000,212.9000) -- (162.1000,222.6000);

        % polyline5052
        \path[draw=c999999,line join=round,line cap=round,very thin,densely dotted]
          (198.8000,516.6000) -- (189.1000,518.3000) -- (185.5000,517.0000) --
          (180.0000,519.1000) -- (177.7000,516.9000) -- (175.0000,518.6000) --
          (172.2000,516.0000) -- (161.2000,518.3000) -- (159.3000,520.8000) --
          (156.5000,519.1000) -- (153.5000,520.7000) -- (152.1000,517.5000) --
          (139.8000,520.3000) -- (134.1000,517.6000);

        % line5054
        \path[draw=c999999,line join=round,line cap=round,very thin,densely dotted]
          (162.1000,222.6000) -- (161.5000,224.9000);

        % polyline5056
        \path[draw=c999999,line join=round,line cap=round,very thin,densely dotted]
          (230.0000,239.0000) -- (221.7000,239.0000) -- (215.9000,235.3000) --
          (212.4000,236.7000) -- (210.1000,234.5000) -- (209.9000,230.6000) --
          (202.4000,230.7000) -- (201.8000,227.7000);

        % polyline5058
        \path[draw=c999999,line join=round,line cap=round,very thin,densely dotted]
          (469.9000,106.9000) -- (471.3000,112.9000) -- (477.4000,110.6000) --
          (482.3000,116.1000) -- (483.7000,119.1000) -- (482.0000,125.7000) --
          (487.5000,138.6000) -- (486.5000,141.7000) -- (484.0000,157.1000) --
          (485.4000,166.8000) -- (483.6000,170.0000) -- (486.4000,172.3000) --
          (486.2000,175.4000);

        % polyline5060
        \path[draw=c999999,line join=round,line cap=round,very thin,densely dotted]
          (230.0000,239.0000) -- (229.2000,241.8000) -- (227.3000,252.2000) --
          (220.8000,269.4000) -- (220.9000,269.4000) -- (227.0000,273.0000) --
          (227.3000,276.6000) -- (230.6000,276.3000) -- (232.0000,283.2000) --
          (235.2000,284.8000) -- (246.4000,284.3000) -- (244.8000,281.0000) --
          (250.8000,284.4000) -- (251.1000,287.9000) -- (259.0000,299.1000);

        % polyline5062
        \path[draw=c999999,line join=round,line cap=round,very thin,densely dotted]
          (213.4000,276.2000) -- (210.9000,273.5000) -- (200.0000,273.8000) --
          (193.7000,274.7000) -- (189.6000,280.0000) -- (175.8000,280.9000);

        % line5064
        \path[draw=c999999,line join=round,line cap=round,very thin,densely dotted]
          (220.9000,269.4000) -- (213.4000,276.2000);

        % polyline5066
        \path[draw=c999999,line join=round,line cap=round,very thin,densely dotted]
          (225.7000,343.6000) -- (226.7000,340.4000) -- (221.6000,337.0000) --
          (225.2000,327.5000) -- (218.1000,327.3000) -- (218.1000,324.1000) --
          (215.8000,321.9000) -- (215.1000,315.4000) -- (219.0000,309.2000) --
          (217.3000,305.8000) -- (214.3000,306.7000) -- (219.3000,290.9000) --
          (216.3000,289.6000) -- (217.2000,286.2000) -- (213.4000,276.2000);

        % polyline5068
        \path[draw=c999999,line join=round,line cap=round,very thin,densely dotted]
          (163.8000,276.6000) -- (163.4000,273.3000) -- (160.3000,271.7000) --
          (163.4000,266.7000) -- (162.5000,263.6000) -- (158.5000,258.3000) --
          (155.1000,257.4000) -- (158.3000,254.9000) -- (174.4000,252.5000) --
          (174.1000,245.9000) -- (164.5000,242.2000) -- (166.6000,239.5000) --
          (169.8000,239.9000) -- (161.5000,224.9000);

        % polyline5070
        \path[draw=c999999,line join=round,line cap=round,very thin,densely dotted]
          (124.4000,274.9000) -- (131.2000,281.9000) -- (144.3000,287.5000) --
          (144.6000,276.8000) -- (148.4000,276.5000) -- (148.4000,280.1000) --
          (151.0000,282.0000) -- (153.7000,279.6000) -- (153.2000,276.3000) --
          (156.7000,275.4000) -- (158.4000,272.7000) -- (163.8000,276.6000);

        % polyline5072
        \path[draw=c999999,line join=round,line cap=round,very thin,densely dotted]
          (175.8000,280.9000) -- (178.3000,287.3000) -- (183.0000,291.9000) --
          (188.4000,312.1000) -- (188.1000,320.4000) -- (190.9000,322.6000) --
          (180.7000,327.6000);

        % line5074
        \path[draw=c999999,line join=round,line cap=round,very thin,densely dotted]
          (175.8000,280.9000) -- (163.8000,276.6000);

        % polyline5076
        \path[draw=c999999,line join=round,line cap=round,very thin,densely dotted]
          (180.7000,327.6000) -- (178.0000,325.7000) -- (170.8000,326.0000) --
          (172.3000,322.7000) -- (162.4000,326.5000);

        % polyline5078
        \path[draw=c999999,line join=round,line cap=round,very thin,densely dotted]
          (497.3000,543.2000) -- (497.8000,539.5000) -- (500.6000,537.3000) --
          (498.1000,530.2000) -- (499.4000,527.2000) -- (502.5000,526.7000) --
          (496.2000,516.5000) -- (498.5000,510.6000) -- (502.2000,509.6000) --
          (500.0000,505.3000);

        % polyline5080
        \path[draw=c999999,line join=round,line cap=round,very thin,densely dotted]
          (146.1000,332.3000) -- (146.2000,332.2000) -- (146.3000,332.2000) --
          (146.2000,332.1000) -- (143.5000,329.7000) -- (143.5000,329.9000) --
          (144.6000,332.7000) -- (145.3000,332.4000);

        % line5082
        \path[draw=c999999,line join=round,line cap=round,very thin,densely dotted]
          (500.2000,504.9000) -- (500.0000,505.3000);

        % polyline5084
        \path[draw=c999999,line join=round,line cap=round,very thin,densely dotted]
          (500.0000,505.3000) -- (500.0000,505.3000) -- (492.5000,510.0000) --
          (488.9000,510.3000) -- (475.4000,505.0000) -- (471.5000,505.7000) --
          (466.4000,503.1000) -- (460.3000,496.7000) -- (448.6000,493.4000);

        % polyline5086
        \path[draw=c999999,line join=round,line cap=round,very thin,densely dotted]
          (180.7000,327.6000) -- (180.7000,331.8000) -- (187.3000,339.1000) --
          (206.9000,346.7000) -- (210.9000,351.9000);

        % polyline5088
        \path[draw=c999999,line join=round,line cap=round,very thin,densely dotted]
          (210.9000,351.9000) -- (214.0000,350.1000) -- (214.3000,346.8000) --
          (220.7000,343.9000) -- (224.8000,344.4000) -- (225.6000,343.6000) --
          (225.7000,343.6000);

        % polyline5090
        \path[draw=c999999,line join=round,line cap=round,very thin,densely dotted]
          (486.1000,478.8000) -- (488.6000,484.9000) -- (487.2000,488.3000) --
          (491.4000,494.2000) -- (489.5000,497.0000) -- (493.3000,498.1000) --
          (500.2000,504.9000);

        % polyline5092
        \path[draw=c999999,line join=round,line cap=round,very thin,densely dotted]
          (103.7000,244.7000) -- (120.9000,239.6000) -- (121.4000,232.8000);

        % polyline5094
        \path[draw=c999999,line join=round,line cap=round,very thin,densely dotted]
          (543.9000,498.7000) -- (533.0000,499.1000) -- (531.6000,503.0000) --
          (525.0000,499.0000) -- (513.6000,508.6000) -- (509.1000,503.1000) --
          (506.1000,505.1000) -- (500.2000,504.9000);

        % polyline5096
        \path[draw=c999999,line join=round,line cap=round,very thin,densely dotted]
          (210.9000,351.9000) -- (207.9000,364.9000) -- (204.8000,365.1000) --
          (202.3000,362.8000) -- (193.2000,366.5000) -- (196.8000,372.9000) --
          (194.3000,375.2000) -- (199.9000,378.7000) -- (200.2000,382.1000) --
          (203.0000,383.9000) -- (201.0000,386.8000) -- (202.7000,389.8000) --
          (200.1000,391.7000) -- (202.6000,394.0000) -- (200.7000,397.3000) --
          (207.4000,398.6000) -- (214.5000,405.8000) -- (214.5000,405.8000);

        % polyline5098
        \path[draw=c999999,line join=round,line cap=round,very thin,densely dotted]
          (212.1000,411.9000) -- (201.6000,410.8000) -- (194.7000,405.9000) --
          (194.1000,399.5000) -- (186.9000,397.6000) -- (186.1000,394.3000) --
          (179.5000,395.3000);

        % polyline5100
        \path[draw=c999999,line join=round,line cap=round,very thin,densely dotted]
          (214.5000,405.8000) -- (214.5000,405.8000) -- (216.8000,403.6000) --
          (222.9000,402.8000) -- (227.5000,396.3000) -- (228.4000,389.9000) --
          (237.2000,383.3000) -- (239.6000,377.3000) -- (245.9000,370.5000) --
          (245.9000,370.4000) -- (252.9000,373.8000) -- (252.9000,379.1000) --
          (263.5000,377.9000) -- (271.5000,384.2000) -- (269.3000,387.2000) --
          (275.8000,389.6000);

        % polyline5102
        \path[draw=c999999,line join=round,line cap=round,very thin,densely dotted]
          (214.5000,405.8000) -- (214.4000,406.3000) -- (212.1000,411.9000);

        % polyline5104
        \path[draw=c999999,line join=round,line cap=round,very thin,densely dotted]
          (228.1000,435.9000) -- (227.0000,434.4000) -- (228.0000,431.1000) --
          (225.4000,428.2000) -- (223.0000,431.8000) -- (212.0000,429.3000) --
          (214.3000,427.0000) -- (214.5000,420.6000) -- (217.7000,414.4000) --
          (216.4000,411.3000) -- (212.1000,411.9000);

        % polyline5106
        \path[draw=c999999,line join=round,line cap=round,very thin,densely dotted]
          (150.7000,453.5000) -- (159.5000,449.6000) -- (163.7000,451.1000) --
          (162.1000,457.0000) -- (181.4000,456.2000) -- (180.9000,459.5000) --
          (193.3000,468.0000) -- (194.0000,473.3000) -- (201.2000,473.3000) --
          (203.0000,469.3000) -- (205.3000,472.3000);

        % polyline5108
        \path[draw=c999999,line join=round,line cap=round,very thin,densely dotted]
          (205.3000,472.3000) -- (210.4000,467.8000) -- (208.7000,465.1000) --
          (212.1000,462.5000) -- (212.7000,452.5000) -- (222.3000,442.7000) --
          (219.0000,441.2000) -- (220.2000,437.6000) -- (223.3000,438.5000) --
          (224.6000,435.7000) -- (228.1000,435.9000);

        % polyline5110
        \path[draw=c999999,line join=round,line cap=round,very thin,densely dotted]
          (266.2000,449.5000) -- (261.4000,444.8000) -- (255.4000,447.9000) --
          (253.7000,441.0000) -- (246.8000,442.2000) -- (244.4000,440.2000) --
          (241.2000,442.5000) -- (231.2000,443.0000) -- (228.1000,435.9000);

        % polyline5112
        \path[draw=c999999,line join=round,line cap=round,very thin,densely dotted]
          (215.6000,489.5000) -- (214.9000,486.6000) -- (218.9000,480.7000) --
          (205.8000,477.5000) -- (205.3000,472.3000);

        % polyline5114
        \path[draw=c999999,line join=round,line cap=round,very thin,densely dotted]
          (416.8000,270.7000) -- (416.8000,269.7000) -- (417.2000,266.1000) --
          (413.0000,264.8000) -- (411.6000,258.0000);

        % polyline5116
        \path[draw=c999999,line join=round,line cap=round,very thin,densely dotted]
          (411.6000,258.0000) -- (405.4000,258.5000) -- (403.8000,255.3000) --
          (400.7000,256.3000) -- (399.6000,252.2000) -- (393.1000,253.2000) --
          (386.9000,249.9000) -- (382.6000,243.6000) -- (377.4000,248.3000) --
          (370.7000,248.3000) -- (362.3000,243.0000) -- (358.5000,243.0000) --
          (358.5000,242.9000);

        % line5118
        \path[draw=c999999,line join=round,line cap=round,very thin,densely dotted]
          (415.5000,271.0000) -- (416.8000,270.7000);

        % polyline5120
        \path[draw=c999999,line join=round,line cap=round,very thin,densely dotted]
          (465.7000,98.9000) -- (463.7000,100.2000) -- (461.4000,103.2000) --
          (455.0000,100.9000) -- (452.8000,103.2000) -- (453.8000,113.0000) --
          (450.2000,118.7000) -- (451.7000,121.5000) -- (447.0000,125.6000) --
          (447.0000,125.7000);

        % polyline5122
        \path[draw=c999999,line join=round,line cap=round,very thin,densely dotted]
          (415.5000,271.0000) -- (412.6000,272.9000) -- (412.5000,271.3000) --
          (415.5000,271.0000);

        % polyline5124
        \path[draw=c999999,line join=round,line cap=round,very thin,densely dotted]
          (390.0000,299.2000) -- (397.8000,302.4000) -- (410.5000,295.9000) --
          (409.7000,292.8000) -- (411.9000,290.3000) -- (407.4000,282.6000) --
          (409.7000,280.6000) -- (409.4000,274.0000) -- (412.6000,272.9000);

        % polyline5126
        \path[draw=c999999,line join=round,line cap=round,very thin,densely dotted]
          (353.7000,246.4000) -- (355.9000,253.7000) -- (354.0000,260.1000) --
          (357.0000,262.5000) -- (360.5000,271.0000) -- (360.8000,277.7000) --
          (363.4000,279.9000) -- (363.1000,286.5000) -- (361.6000,296.4000);

        % polyline5128
        \path[draw=c999999,line join=round,line cap=round,very thin,densely dotted]
          (358.5000,242.9000) -- (355.4000,243.7000) -- (353.7000,246.4000);

        % polyline5130
        \path[draw=c999999,line join=round,line cap=round,very thin,densely dotted]
          (353.7000,246.4000) -- (343.5000,248.2000) -- (339.2000,242.9000) --
          (324.0000,239.4000);

        % path5132
        \path[draw=c999999,line join=round,line cap=round,very thin,densely dotted]
          (339.5000,341.7000) -- (335.5000,332.3000) .. controls (332.7000,330.5000) and
          (330.0000,328.7000) .. (327.2000,326.8000) -- (325.9000,322.5000) --
          (325.9000,322.4000) -- (329.8000,315.8000) -- (339.7000,314.7000) --
          (341.6000,312.0000) -- (339.7000,306.0000) -- (342.2000,303.9000) --
          (351.7000,300.6000) -- (357.0000,296.4000) -- (361.6000,296.4000);

        % polyline5134
        \path[draw=c999999,line join=round,line cap=round,very thin,densely dotted]
          (361.6000,296.4000) -- (368.1000,303.6000) -- (375.2000,303.7000) --
          (378.4000,301.8000) -- (383.3000,306.0000) -- (390.0000,299.2000);

        % polyline5136
        \path[draw=c999999,line join=round,line cap=round,very thin,densely dotted]
          (339.5000,341.7000) -- (339.0000,346.3000) -- (341.5000,353.9000) --
          (331.1000,363.5000) -- (336.1000,369.7000);

        % polyline5138
        \path[draw=c999999,line join=round,line cap=round,very thin,densely dotted]
          (394.1000,355.8000) -- (390.0000,351.1000) -- (382.8000,350.6000) --
          (381.4000,347.0000) -- (366.6000,346.5000) -- (364.8000,343.7000) --
          (361.1000,344.2000) -- (356.1000,339.2000) -- (357.5000,336.0000) --
          (354.9000,334.0000) -- (342.4000,343.1000) -- (339.5000,341.7000);

        % polyline5140
        \path[draw=c999999,line join=round,line cap=round,very thin,densely dotted]
          (403.5000,331.7000) -- (407.6000,326.5000) -- (408.2000,320.4000) --
          (406.3000,317.7000) -- (397.5000,314.4000) -- (390.0000,299.2000);

        % polyline5142
        \path[draw=c999999,line join=round,line cap=round,very thin,densely dotted]
          (394.1000,355.8000) -- (400.2000,350.5000) -- (398.5000,336.5000) --
          (399.1000,333.1000) -- (403.5000,331.7000);

        % polyline5144
        \path[draw=c999999,line join=round,line cap=round,very thin,densely dotted]
          (427.0000,335.1000) -- (420.6000,339.5000) -- (418.2000,337.1000) --
          (409.3000,338.7000) -- (403.5000,335.1000) -- (403.5000,331.7000);

        % polyline5146
        \path[draw=c999999,line join=round,line cap=round,very thin,densely dotted]
          (290.6000,214.2000) -- (286.9000,212.1000) -- (280.8000,216.0000) --
          (266.4000,203.9000) -- (259.3000,205.4000);

        % polyline5148
        \path[draw=c999999,line join=round,line cap=round,very thin,densely dotted]
          (305.1000,267.1000) -- (299.3000,263.2000) -- (292.2000,264.2000) --
          (290.4000,267.1000) -- (287.7000,265.7000) -- (280.9000,269.3000);

        % polyline5150
        \path[draw=c999999,line join=round,line cap=round,very thin,densely dotted]
          (320.4000,322.3000) -- (322.1000,318.3000) -- (321.1000,308.3000) --
          (315.9000,299.5000) -- (317.4000,296.5000) -- (314.8000,293.9000) --
          (317.6000,288.2000) -- (315.0000,275.1000) -- (302.1000,273.3000) --
          (305.1000,267.1000);

        % polyline5152
        \path[draw=c999999,line join=round,line cap=round,very thin,densely dotted]
          (280.9000,269.3000) -- (282.0000,274.8000) -- (279.2000,277.1000) --
          (276.3000,281.1000) -- (273.7000,279.2000) -- (267.9000,282.2000) --
          (264.3000,295.6000) -- (265.4000,298.6000) -- (259.0000,299.1000);

        % polyline5154
        \path[draw=c999999,line join=round,line cap=round,very thin,densely dotted]
          (259.0000,299.1000) -- (261.4000,302.5000) -- (260.6000,309.1000) --
          (265.7000,313.9000) -- (269.0000,313.9000) -- (271.9000,320.1000) --
          (274.8000,321.5000) -- (273.1000,324.6000);

        % polyline5156
        \path[draw=c999999,line join=round,line cap=round,very thin,densely dotted]
          (273.1000,324.6000) -- (275.9000,325.9000) -- (282.4000,324.7000) --
          (284.4000,327.4000);

        % polyline5158
        \path[draw=c999999,line join=round,line cap=round,very thin,densely dotted]
          (273.1000,324.6000) -- (272.1000,324.3000) -- (270.7000,327.0000) --
          (266.9000,326.0000) -- (265.6000,329.4000) -- (259.2000,332.4000) --
          (255.8000,338.1000) -- (255.8000,342.0000);

        % polyline5160
        \path[draw=c999999,line join=round,line cap=round,very thin,densely dotted]
          (255.8000,342.0000) -- (250.5000,341.2000) -- (249.0000,344.0000) --
          (244.7000,345.6000) -- (240.1000,341.2000) -- (237.6000,343.3000) --
          (239.2000,346.0000) -- (235.9000,346.9000) -- (225.7000,343.6000);

        % polyline5162
        \path[draw=c999999,line join=round,line cap=round,very thin,densely dotted]
          (245.9000,370.4000) -- (249.7000,366.1000) -- (252.8000,365.5000) --
          (257.3000,356.5000) -- (260.2000,355.3000) -- (260.9000,351.7000) --
          (255.6000,347.7000) -- (255.8000,342.0000);

        % line5164
        \path[draw=c999999,line join=round,line cap=round,very thin,densely dotted]
          (320.4000,322.3000) -- (325.9000,322.5000);

        % polyline5166
        \path[draw=c999999,line join=round,line cap=round,very thin,densely dotted]
          (284.4000,327.4000) -- (287.5000,324.1000) -- (300.4000,323.0000) --
          (302.6000,320.2000) -- (317.4000,323.2000) -- (320.4000,322.3000);

        % polyline5168
        \path[draw=c999999,line join=round,line cap=round,very thin,densely dotted]
          (307.3000,372.4000) -- (306.6000,365.7000) -- (301.3000,362.4000) --
          (294.1000,361.9000) -- (294.0000,358.6000) -- (290.8000,357.0000) --
          (291.7000,353.1000) -- (289.8000,347.3000) -- (282.3000,336.3000) --
          (285.7000,330.3000) -- (284.4000,327.4000);

        % polyline5170
        \path[draw=c999999,line join=round,line cap=round,very thin,densely dotted]
          (336.1000,369.7000) -- (333.4000,372.4000) -- (330.3000,371.5000) --
          (327.8000,374.4000) -- (315.2000,368.6000) -- (307.3000,372.4000);

        % polyline5172
        \path[draw=c999999,line join=round,line cap=round,very thin,densely dotted]
          (307.3000,372.4000) -- (297.4000,376.8000) -- (292.1000,382.1000) --
          (285.6000,383.5000) -- (280.3000,387.9000) -- (277.5000,386.7000) --
          (275.8000,389.6000);

        % polyline5174
        \path[draw=c999999,line join=round,line cap=round,very thin,densely dotted]
          (336.9000,387.6000) -- (337.7000,394.3000) -- (330.7000,392.1000) --
          (330.0000,396.0000) -- (321.4000,406.6000) -- (323.5000,409.1000) --
          (320.9000,411.3000) -- (320.6000,415.2000) -- (317.1000,416.4000) --
          (318.8000,420.2000) -- (315.1000,422.3000) -- (315.1000,422.3000);

        % polyline5176
        \path[draw=c999999,line join=round,line cap=round,very thin,densely dotted]
          (275.8000,389.6000) -- (278.9000,395.0000) -- (274.9000,400.1000) --
          (276.9000,403.4000) -- (274.6000,405.5000) -- (277.2000,407.2000) --
          (277.4000,410.6000) -- (281.1000,411.3000) -- (284.9000,419.2000);

        % polyline5178
        \path[draw=c999999,line join=round,line cap=round,very thin,densely dotted]
          (284.9000,419.2000) -- (284.2000,424.8000) -- (282.6000,430.8000) --
          (277.5000,434.4000) -- (278.3000,438.4000) -- (271.4000,442.3000) --
          (266.2000,449.5000);

        % polyline5180
        \path[draw=c999999,line join=round,line cap=round,very thin,densely dotted]
          (265.4000,463.3000) -- (256.9000,465.5000) -- (256.9000,468.8000) --
          (259.6000,470.3000) -- (254.7000,476.5000) -- (256.2000,479.6000) --
          (253.0000,479.3000) -- (249.0000,484.6000);

        % polyline5182
        \path[draw=c999999,line join=round,line cap=round,very thin,densely dotted]
          (265.4000,463.3000) -- (268.9000,462.7000) -- (266.4000,465.3000) --
          (271.5000,469.7000) -- (274.6000,470.9000) -- (277.3000,469.3000) --
          (277.8000,473.6000) -- (287.6000,470.9000) -- (290.6000,473.5000) --
          (299.0000,467.2000) -- (305.9000,466.4000);

        % polyline5184
        \path[draw=c999999,line join=round,line cap=round,very thin,densely dotted]
          (266.2000,449.5000) -- (262.4000,453.0000) -- (265.4000,463.3000);

        % polyline5186
        \path[draw=c999999,line join=round,line cap=round,very thin,densely dotted]
          (315.1000,422.3000) -- (307.5000,422.2000) -- (301.6000,425.1000) --
          (294.0000,418.3000) -- (284.9000,419.2000);

        % polyline5188
        \path[draw=c999999,line join=round,line cap=round,very thin,densely dotted]
          (322.1000,447.3000) -- (318.5000,441.7000) -- (320.4000,435.2000) --
          (316.2000,428.5000) -- (315.1000,422.3000);

        % polyline5190
        \path[draw=c999999,line join=round,line cap=round,very thin,densely dotted]
          (360.4000,445.2000) -- (357.5000,439.4000) -- (358.0000,436.0000) --
          (355.4000,434.3000) -- (354.0000,429.5000) -- (348.1000,425.1000) --
          (342.8000,429.8000) -- (339.4000,439.9000) -- (335.1000,445.4000) --
          (328.0000,443.7000) -- (322.1000,447.3000);

        % polyline5192
        \path[draw=c999999,line join=round,line cap=round,very thin,densely dotted]
          (305.9000,466.4000) -- (305.0000,456.1000) -- (313.6000,450.5000) --
          (320.3000,449.0000) -- (322.1000,447.3000) -- (322.1000,447.3000);

        % polyline5194
        \path[draw=c999999,line join=round,line cap=round,very thin,densely dotted]
          (305.9000,466.4000) -- (306.0000,466.4000) -- (305.5000,469.7000) --
          (309.0000,472.8000) -- (306.3000,475.5000) -- (311.2000,479.5000);

        % polyline5196
        \path[draw=c999999,line join=round,line cap=round,very thin,densely dotted]
          (404.6000,395.3000) -- (407.9000,387.3000) -- (404.9000,378.8000) --
          (399.7000,374.6000) -- (394.5000,366.0000) -- (396.9000,358.9000) --
          (394.1000,355.8000);

        % polyline5198
        \path[draw=c999999,line join=round,line cap=round,very thin,densely dotted]
          (404.6000,395.3000) -- (392.1000,396.1000) -- (386.2000,392.5000) --
          (376.7000,392.4000) -- (365.9000,396.0000);

        % polyline5200
        \path[draw=c999999,line join=round,line cap=round,very thin,densely dotted]
          (433.4000,403.1000) -- (430.9000,400.4000) -- (427.9000,401.5000) --
          (427.1000,395.3000) -- (420.4000,392.8000) -- (413.4000,396.2000) --
          (410.2000,396.6000) -- (407.7000,394.3000) -- (404.6000,395.3000);

        % polyline5202
        \path[draw=c999999,line join=round,line cap=round,very thin,densely dotted]
          (336.1000,369.7000) -- (337.4000,375.9000) -- (335.5000,378.7000) --
          (336.9000,387.6000);

        % polyline5204
        \path[draw=c999999,line join=round,line cap=round,very thin,densely dotted]
          (336.9000,387.6000) -- (345.3000,391.4000) -- (347.0000,394.3000) --
          (354.5000,394.6000) -- (361.8000,400.7000) -- (365.9000,396.0000) --
          (365.9000,396.0000);

        % polyline5206
        \path[draw=c999999,line join=round,line cap=round,very thin,densely dotted]
          (365.9000,396.0000) -- (366.0000,400.5000) -- (372.1000,402.1000) --
          (375.8000,411.9000) -- (378.7000,413.2000) -- (375.0000,413.4000) --
          (378.9000,422.4000);

        % polyline5208
        \path[draw=c999999,line join=round,line cap=round,very thin,densely dotted]
          (403.8000,437.8000) -- (393.6000,429.1000) -- (390.5000,428.8000) --
          (390.3000,432.1000) -- (383.1000,430.7000) -- (381.4000,424.3000) --
          (378.9000,422.4000) -- (377.7000,423.5000) -- (370.1000,429.9000) --
          (368.1000,427.3000) -- (360.4000,445.2000);

        % polyline5210
        \path[draw=c999999,line join=round,line cap=round,very thin,densely dotted]
          (162.1000,222.6000) -- (173.8000,226.5000) -- (176.2000,224.3000) --
          (186.0000,228.4000) -- (189.4000,228.9000) -- (196.0000,225.7000) --
          (201.7000,227.7000) -- (200.0000,221.5000) -- (206.2000,212.8000) --
          (204.6000,208.9000) -- (210.2000,203.8000) -- (208.8000,200.5000) --
          (213.8000,196.1000) -- (213.5000,188.3000) -- (218.3000,185.2000);

        % polyline5212
        \path[draw=c999999,line join=round,line cap=round,very thin,densely dotted]
          (380.1000,477.7000) -- (369.5000,472.7000) -- (367.8000,470.0000) --
          (368.4000,460.2000) -- (365.2000,450.3000) -- (360.4000,445.3000) --
          (360.4000,445.2000);

        % polyline5214
        \path[draw=c999999,line join=round,line cap=round,very thin,densely dotted]
          (410.8000,457.2000) -- (406.1000,447.9000) -- (403.8000,437.8000);

        % polyline5216
        \path[draw=c999999,line join=round,line cap=round,very thin,densely dotted]
          (410.8000,457.2000) -- (409.6000,459.5000) -- (405.5000,462.5000) --
          (410.0000,471.1000) -- (408.6000,474.4000) -- (410.2000,477.7000) --
          (403.2000,479.8000) -- (393.4000,477.2000) -- (393.2000,481.0000) --
          (388.9000,481.6000) -- (380.1000,477.7000);

        % polyline5218
        \path[draw=c999999,line join=round,line cap=round,very thin,densely dotted]
          (379.7000,495.5000) -- (379.7000,495.6000) -- (382.9000,495.7000) --
          (383.3000,498.9000) -- (385.8000,497.1000) -- (391.7000,499.9000) --
          (391.9000,496.7000) -- (396.6000,492.3000) -- (401.0000,491.6000) --
          (403.1000,494.6000) -- (402.1000,497.6000) -- (409.7000,499.2000) --
          (409.9000,502.5000) -- (413.3000,502.8000) -- (420.2000,510.3000) --
          (421.8000,513.6000) -- (417.2000,519.8000) -- (417.2000,519.9000);

        % polyline5220
        \path[draw=c999999,line join=round,line cap=round,very thin,densely dotted]
          (543.9000,498.7000) -- (550.1000,501.9000) -- (552.0000,508.1000) --
          (558.3000,511.1000) -- (557.4000,514.3000) -- (559.9000,518.8000);

        % polyline5222
        \path[draw=c999999,line join=round,line cap=round,very thin,densely dotted]
          (424.0000,526.1000) -- (428.3000,519.4000) -- (434.8000,517.4000) --
          (433.6000,514.4000) -- (438.4000,509.0000) -- (442.0000,510.0000) --
          (444.2000,498.1000) -- (448.6000,493.4000);

        % polyline5224
        \path[draw=c999999,line join=round,line cap=round,very thin,densely dotted]
          (380.1000,477.7000) -- (377.8000,482.1000) -- (374.7000,483.4000) --
          (384.2000,488.0000) -- (379.3000,493.9000) -- (379.7000,495.5000);

        % polyline5226
        \path[draw=c999999,line join=round,line cap=round,very thin,densely dotted]
          (358.5000,510.9000) -- (364.9000,511.0000) -- (365.1000,501.2000) --
          (372.1000,502.6000) -- (375.4000,496.7000) -- (379.7000,495.5000);

        % polyline5228
        \path[draw=c999999,line join=round,line cap=round,very thin,densely dotted]
          (358.5000,510.9000) -- (357.1000,507.8000) -- (344.6000,506.3000) --
          (339.8000,500.6000) -- (335.1000,487.6000) -- (329.6000,482.9000) --
          (318.1000,476.3000) -- (311.2000,479.5000);

        % path5230
        \path[draw=c999999,line join=round,line cap=round,very thin,densely dotted]
          (340.1000,529.9000) -- (343.9000,527.6000) -- (344.9000,524.2000) --
          (344.1000,513.4000) -- (350.9000,515.8000) .. controls (353.5000,514.1000) and
          (356.0000,512.5000) .. (358.5000,510.9000);

        % polyline5232
        \path[draw=c999999,line join=round,line cap=round,very thin,densely dotted]
          (374.2000,544.0000) -- (372.3000,541.7000) -- (356.7000,536.5000) --
          (355.5000,533.2000) -- (351.3000,539.2000) -- (347.8000,540.1000) --
          (346.3000,536.8000) -- (343.5000,538.7000) -- (339.5000,533.0000) --
          (340.1000,529.9000);

        % polyline5234
        \path[draw=c999999,line join=round,line cap=round,very thin,densely dotted]
          (205.9000,536.5000) -- (206.0000,539.5000) -- (203.5000,537.7000) --
          (205.9000,536.5000);

        % polyline5236
        \path[draw=c999999,line join=round,line cap=round,very thin,densely dotted]
          (364.3000,570.1000) -- (355.4000,564.7000) -- (346.1000,570.8000) --
          (330.9000,569.7000) -- (327.9000,571.3000) -- (329.5000,578.0000) --
          (319.8000,582.4000);

        % polyline5238
        \path[draw=c999999,line join=round,line cap=round,very thin,densely dotted]
          (311.2000,479.5000) -- (300.8000,483.1000) -- (297.9000,481.9000) --
          (294.4000,483.1000) -- (296.5000,485.6000) -- (295.4000,488.6000) --
          (289.9000,492.2000) -- (289.4000,495.3000) -- (289.4000,495.3000);

        % polyline5240
        \path[draw=c999999,line join=round,line cap=round,very thin,densely dotted]
          (313.1000,529.0000) -- (312.0000,525.8000) -- (295.9000,515.3000) --
          (296.8000,509.3000) -- (294.1000,507.0000) -- (293.9000,503.8000) --
          (291.0000,502.4000) -- (289.4000,495.3000);

        % polyline5242
        \path[draw=c999999,line join=round,line cap=round,very thin,densely dotted]
          (295.7000,540.1000) -- (305.2000,528.9000) -- (306.7000,531.8000) --
          (312.6000,529.9000) -- (313.1000,529.0000);

        % polyline5244
        \path[draw=c999999,line join=round,line cap=round,very thin,densely dotted]
          (313.1000,529.0000) -- (315.1000,531.6000) -- (322.2000,532.6000) --
          (323.4000,528.9000) -- (326.7000,528.3000) -- (340.1000,529.9000);

        % polyline5246
        \path[draw=c999999,line join=round,line cap=round,very thin,densely dotted]
          (289.4000,495.3000) -- (276.3000,500.2000) -- (278.8000,502.8000) --
          (272.2000,505.6000) -- (270.4000,503.0000) -- (259.2000,504.0000);

        % polyline5248
        \path[draw=c999999,line join=round,line cap=round,very thin,densely dotted]
          (259.2000,504.0000) -- (256.9000,496.3000) -- (251.3000,491.7000) --
          (254.9000,486.5000) -- (249.0000,484.6000);

        % polyline5250
        \path[draw=c999999,line join=round,line cap=round,very thin,densely dotted]
          (249.0000,484.6000) -- (246.0000,486.1000) -- (236.3000,484.9000) --
          (230.7000,488.2000) -- (221.4000,487.4000) -- (219.8000,490.2000) --
          (215.6000,489.5000) -- (215.6000,489.5000);

        % polyline5252
        \path[draw=c999999,line join=round,line cap=round,very thin,densely dotted]
          (259.2000,504.0000) -- (269.4000,519.0000) -- (262.1000,525.8000) --
          (261.3000,532.3000) -- (248.6000,529.5000) -- (244.6000,535.5000) --
          (241.1000,536.7000);

        % polyline5254
        \path[draw=c999999,line join=round,line cap=round,very thin,densely dotted]
          (253.2000,570.4000) -- (252.1000,567.2000) -- (256.4000,561.2000) --
          (259.6000,560.2000) -- (259.6000,553.5000) -- (263.9000,549.2000) --
          (267.0000,547.9000) -- (274.0000,552.5000) -- (273.6000,546.2000) --
          (279.7000,544.3000) -- (276.2000,539.0000) -- (279.4000,537.7000) --
          (283.3000,543.9000) -- (286.4000,543.4000) -- (287.1000,540.2000) --
          (293.5000,542.7000) -- (295.7000,540.1000);

        % polyline5256
        \path[draw=c999999,line join=round,line cap=round,very thin,densely dotted]
          (206.2000,517.0000) -- (210.6000,516.4000) -- (212.9000,520.3000) --
          (216.5000,521.8000) -- (219.4000,531.6000) -- (225.6000,530.9000) --
          (227.2000,534.2000) -- (241.1000,536.7000) -- (233.2000,549.3000) --
          (238.3000,553.7000) -- (237.7000,557.7000) -- (241.1000,557.4000) --
          (242.1000,560.4000) -- (239.1000,566.3000) -- (232.9000,570.0000) --
          (233.6000,578.7000);

        % polyline5258
        \path[draw=c999999,line join=round,line cap=round,very thin,densely dotted]
          (319.8000,582.4000) -- (314.3000,576.1000) -- (310.7000,577.5000) --
          (306.9000,575.8000) -- (305.9000,569.4000) -- (309.6000,568.7000) --
          (310.9000,565.3000) -- (308.2000,562.5000) -- (310.6000,560.0000) --
          (310.4000,556.5000) -- (308.0000,550.6000) -- (298.2000,546.3000) --
          (295.7000,540.1000);

        % polyline5260
        \path[draw=c999999,line join=round,line cap=round,very thin,densely dotted]
          (319.8000,582.4000) -- (311.5000,582.6000) -- (310.0000,585.5000) --
          (300.1000,588.5000);

        % polyline5262
        \path[draw=c999999,line join=round,line cap=round,very thin,densely dotted]
          (198.8000,516.6000) -- (197.0000,513.5000) -- (201.8000,501.1000) --
          (201.2000,493.9000) -- (211.7000,490.9000) -- (212.4000,494.2000) --
          (215.3000,492.8000) -- (215.6000,489.5000);

        % polyline5264
        \path[draw=c999999,line join=round,line cap=round,very thin,densely dotted]
          (193.5000,567.3000) -- (193.5000,567.2000) -- (194.9000,557.5000) --
          (196.7000,554.4000) -- (200.4000,553.5000) -- (200.0000,550.1000) --
          (209.0000,539.8000) -- (208.1000,536.3000) -- (211.1000,534.7000) --
          (210.6000,526.6000) -- (207.4000,526.6000) -- (209.0000,523.5000) --
          (206.2000,517.0000);

        % polyline5266
        \path[draw=c666666,line join=round,line cap=round,very thin,densely dotted]
          (463.1000,520.2000) -- (467.7000,520.2000) -- (470.0000,524.7000) --
          (473.7000,523.2000) -- (471.3000,527.5000) -- (467.7000,528.7000) --
          (462.5000,523.9000) -- (463.1000,520.2000);

        % polyline5268
        \path[draw=c666666,line join=round,line cap=round,thin]
          (586.4000,497.6000) -- (586.9000,489.5000) -- (593.9000,479.8000) --
          (596.1000,474.9000) -- (591.6000,469.0000) -- (578.3000,473.8000) --
          (568.0000,468.3000) -- (561.0000,466.9000) -- (554.1000,458.6000) --
          (553.9000,458.4000) -- (556.0000,452.9000) -- (551.6000,446.8000) --
          (554.1000,444.7000) -- (555.6000,438.1000) -- (555.7000,437.9000) --
          (555.8000,437.8000) -- (557.9000,435.4000) -- (561.9000,435.5000) --
          (557.6000,425.0000) -- (550.8000,424.6000) -- (545.0000,420.9000) --
          (544.0000,413.6000) -- (540.9000,413.4000) -- (538.4000,407.6000) --
          (538.3000,407.5000) -- (560.0000,398.2000) -- (562.2000,394.1000) --
          (560.9000,391.1000) -- (564.1000,385.3000) -- (555.8000,379.6000) --
          (553.4000,373.4000) -- (554.1000,369.9000) -- (547.9000,367.4000) --
          (544.6000,364.4000) -- (544.1000,361.0000) -- (544.1000,360.8000) --
          (544.3000,360.6000) -- (553.6000,351.6000) -- (554.6000,348.1000) --
          (550.9000,342.9000) -- (542.1000,334.6000) -- (544.7000,327.7000) --
          (540.6000,321.6000) -- (541.1000,318.4000) -- (537.3000,317.7000) --
          (527.9000,318.3000) -- (522.3000,323.0000) -- (518.3000,321.4000) --
          (515.2000,327.9000) -- (517.9000,331.4000) -- (510.0000,337.4000) --
          (502.5000,338.4000) -- (502.2000,338.5000) -- (502.2000,338.4000) --
          (502.8000,332.8000) -- (509.1000,328.8000) -- (510.5000,321.0000) --
          (506.6000,318.7000) -- (506.4000,318.6000) -- (509.4000,308.5000) --
          (510.4000,305.0000) -- (523.5000,292.9000) -- (522.6000,282.0000) --
          (532.3000,275.4000) -- (545.4000,259.1000) -- (545.2000,255.9000) --
          (547.8000,253.6000) -- (542.9000,249.5000) -- (544.0000,246.2000) --
          (544.2000,245.9000) -- (548.3000,241.4000) -- (548.6000,241.4000) --
          (552.6000,240.9000) -- (555.9000,241.6000) -- (555.0000,244.9000) --
          (563.6000,245.0000) -- (566.6000,243.9000) -- (567.1000,240.4000) --
          (570.4000,240.3000) -- (570.6000,236.8000) -- (573.1000,234.5000) --
          (569.9000,225.8000) -- (572.9000,208.5000) -- (570.7000,196.9000) --
          (570.8000,196.6000) -- (577.4000,181.3000) -- (579.0000,163.7000) --
          (590.8000,148.4000) -- (595.2000,136.4000) -- (585.6000,133.0000) --
          (569.3000,132.4000) -- (569.1000,132.4000) -- (559.7000,124.0000) --
          (552.6000,128.6000) -- (544.8000,127.3000) -- (542.0000,128.7000) --
          (539.5000,124.3000) -- (536.1000,122.7000) -- (532.9000,124.5000) --
          (532.6000,127.7000) -- (529.5000,126.9000) -- (520.5000,116.7000) --
          (518.6000,110.1000) -- (515.3000,107.8000) -- (502.8000,104.9000) --
          (497.9000,108.9000) -- (494.7000,109.0000) -- (489.0000,106.4000) --
          (485.0000,103.0000) -- (477.3000,103.0000) -- (475.9000,106.1000) --
          (469.9000,106.9000) -- (469.6000,107.0000) -- (465.7000,98.9000) --
          (462.1000,99.2000) -- (462.1000,95.6000) -- (455.3000,93.9000) --
          (447.8000,87.3000) -- (441.1000,87.5000) -- (442.2000,80.2000) --
          (439.0000,73.7000) -- (439.0000,69.1000) -- (441.9000,63.4000) --
          (438.9000,61.6000) -- (434.3000,66.5000) -- (432.4000,73.7000) --
          (422.6000,78.1000) -- (416.5000,76.1000) -- (413.3000,76.9000) --
          (413.0000,76.9000) -- (408.8000,75.6000) -- (409.0000,75.3000) --
          (412.6000,68.5000) -- (409.2000,62.5000) -- (411.5000,55.3000) --
          (408.3000,56.0000) -- (403.8000,50.4000) -- (397.0000,51.7000) --
          (390.8000,50.2000) -- (388.4000,52.4000) -- (385.9000,42.2000) --
          (379.8000,40.8000) -- (378.6000,37.7000) -- (375.3000,40.2000) --
          (371.8000,38.9000) -- (368.3000,29.4000) -- (368.8000,26.1000) --
          (364.5000,21.1000) -- (357.9000,21.7000) -- (351.7000,26.0000) --
          (348.3000,24.5000) -- (345.1000,18.7000) -- (341.4000,18.7000) --
          (340.0000,15.7000) -- (341.1000,9.2000) -- (338.3000,2.5000) --
          (335.1000,0.6000);

        % polyline5270
        \path[draw=c666666,line join=round,line cap=round,thin]
          (121.8000,529.9000) -- (128.3000,533.0000) -- (128.8000,536.5000) --
          (131.3000,534.2000) -- (138.8000,535.7000) -- (140.4000,538.8000) --
          (138.3000,545.5000) -- (135.3000,548.1000) -- (136.8000,551.3000) --
          (141.1000,552.3000) -- (141.7000,548.2000) -- (145.0000,547.2000) --
          (145.2000,551.0000) -- (151.8000,554.5000) -- (162.7000,558.7000) --
          (169.7000,558.4000) -- (172.6000,563.6000) -- (181.4000,570.8000) --
          (184.1000,569.0000) -- (186.7000,570.6000) -- (193.3000,567.2000) --
          (193.5000,567.3000) -- (206.0000,578.4000) -- (219.8000,576.5000) --
          (222.6000,578.9000) -- (233.6000,578.7000) -- (240.7000,579.1000) --
          (243.2000,576.7000) -- (242.8000,569.3000) -- (245.6000,567.8000) --
          (253.2000,570.4000) -- (264.3000,573.3000) -- (269.8000,578.2000) --
          (276.9000,577.8000) -- (283.3000,586.2000) -- (284.5000,583.0000) --
          (288.2000,582.8000) -- (300.1000,588.5000) -- (300.1000,588.5000) --
          (298.0000,593.9000) -- (307.8000,596.9000) -- (310.5000,603.2000) --
          (314.1000,603.5000) -- (319.1000,599.0000) -- (322.6000,598.8000) --
          (333.3000,601.2000) -- (338.4000,605.2000) -- (345.5000,604.7000) --
          (345.5000,601.4000) -- (354.6000,596.6000) -- (361.6000,595.4000) --
          (371.3000,597.9000);

        % polyline5272
        \path[draw=c666666,line join=round,line cap=round,thin]
          (371.3000,597.9000) -- (364.8000,589.9000) -- (364.3000,570.1000) --
          (364.8000,559.1000) -- (371.2000,547.1000) -- (374.2000,544.0000) --
          (380.3000,539.5000) -- (387.9000,539.5000) -- (395.3000,531.3000) --
          (407.6000,522.4000) -- (417.1000,519.7000) -- (417.2000,519.9000) --
          (420.4000,525.3000) -- (423.7000,526.1000) -- (424.0000,526.1000) --
          (440.0000,526.6000) -- (442.3000,529.3000) -- (441.7000,532.4000) --
          (445.8000,533.4000) -- (456.7000,532.2000) -- (454.3000,529.6000) --
          (455.7000,526.8000) -- (458.8000,527.0000) -- (463.9000,533.2000) --
          (478.8000,531.5000) -- (483.2000,541.3000) -- (494.1000,544.0000) --
          (497.2000,543.1000) -- (497.3000,543.2000) -- (504.2000,547.2000) --
          (503.9000,550.4000) -- (513.9000,547.5000) -- (520.5000,549.3000) --
          (521.8000,552.5000) -- (524.1000,546.2000) -- (532.4000,547.7000) --
          (532.5000,544.4000) -- (538.9000,543.0000) -- (541.9000,540.2000) --
          (546.1000,540.7000) -- (548.6000,534.6000) -- (544.6000,532.7000) --
          (551.6000,523.6000) -- (557.6000,522.8000) -- (559.9000,518.8000) --
          (562.5000,514.2000) -- (569.0000,511.1000) -- (570.1000,506.5000) --
          (586.4000,497.6000);

        % polyline5274
        \path[draw=c666666,line join=round,line cap=round,thin]
          (65.6000,248.4000) -- (66.5000,252.2000) -- (70.1000,253.2000) --
          (74.5000,251.9000) -- (68.8000,248.0000) -- (65.6000,248.4000);

        % polyline5276
        \path[draw=c666666,line join=round,line cap=round,thin]
          (568.0000,556.2000) -- (564.8000,555.0000) -- (562.1000,557.3000) --
          (562.6000,576.9000) -- (553.0000,575.4000) -- (549.0000,580.8000) --
          (537.0000,585.6000) -- (536.1000,588.8000) -- (532.9000,589.1000) --
          (530.4000,592.3000) -- (530.7000,596.2000) -- (526.3000,601.8000) --
          (526.3000,602.0000) -- (525.7000,605.1000) -- (528.6000,604.0000) --
          (532.9000,608.7000) -- (527.1000,612.9000) -- (528.3000,619.1000) --
          (537.1000,624.0000) -- (534.2000,625.8000) -- (530.9000,635.2000) --
          (539.0000,632.4000) -- (540.0000,638.8000) -- (537.0000,646.7000) --
          (547.7000,648.8000) -- (542.2000,652.5000) -- (542.0000,655.9000) --
          (547.3000,660.8000) -- (559.9000,664.7000) -- (561.3000,667.8000) --
          (564.7000,669.2000) -- (571.3000,652.6000) -- (567.6000,652.3000) --
          (570.5000,650.1000) -- (572.9000,644.0000) -- (572.1000,634.6000) --
          (572.0000,634.3000) -- (572.1000,627.3000) -- (578.1000,614.0000) --
          (573.9000,584.8000) -- (569.3000,578.4000) -- (569.9000,566.5000) --
          (568.0000,556.2000);

        % polyline5278
        \path[draw=c666666,line join=round,line cap=round,thin]
          (111.2000,274.2000) -- (109.1000,276.8000) -- (115.9000,283.3000) --
          (115.7000,279.8000) -- (112.3000,277.7000) -- (111.2000,274.2000);

        % polyline5280
        \path[draw=c666666,line join=round,line cap=round,thin]
          (148.7000,348.9000) -- (155.9000,361.4000) -- (157.4000,357.8000) --
          (156.3000,349.5000) -- (150.6000,345.8000) -- (148.7000,348.9000);

        % polyline5282
        \path[draw=c666666,line join=round,line cap=round,thin]
          (143.3000,329.7000) -- (144.6000,332.8000) -- (145.3000,332.4000) --
          (146.1000,332.3000) -- (143.5000,329.9000) -- (143.3000,329.7000);

        % polyline5284
        \path[draw=c666666,line join=round,line cap=round,thin]
          (148.4000,331.5000) -- (146.2000,332.1000) -- (146.1000,332.2000) --
          (146.2000,332.2000) -- (151.2000,336.7000) -- (154.5000,335.3000) --
          (148.4000,331.5000);

        % polyline5286
        \path[draw=c666666,line join=round,line cap=round,thin]
          (28.6000,209.6000) -- (31.9000,211.8000) -- (37.1000,210.0000) --
          (42.4000,216.6000) -- (45.5000,215.9000) -- (45.5000,212.7000) --
          (46.0000,215.9000) -- (48.9000,214.9000) -- (48.2000,218.0000) --
          (51.4000,218.8000) -- (55.6000,217.1000) -- (55.9000,213.6000) --
          (56.7000,222.3000) -- (59.5000,224.0000) -- (62.5000,222.5000) --
          (62.0000,215.9000) -- (62.7000,219.3000) -- (66.0000,220.0000) --
          (63.6000,222.4000) -- (65.3000,225.3000) -- (69.4000,227.1000) --
          (72.4000,225.7000) -- (69.7000,227.1000) -- (70.0000,230.3000) --
          (72.3000,233.4000) -- (71.5000,237.6000) -- (73.9000,240.2000) --
          (72.5000,237.0000) -- (75.6000,234.3000) -- (79.2000,233.4000) --
          (81.9000,235.5000) -- (80.5000,228.9000) -- (84.0000,234.3000) --
          (85.7000,231.5000) -- (89.8000,232.1000) -- (92.2000,234.9000) --
          (89.9000,237.6000) -- (82.9000,236.7000) -- (88.3000,241.0000) --
          (102.2000,239.8000) -- (105.3000,241.6000) -- (101.9000,241.8000) --
          (103.7000,244.7000) -- (98.7000,249.5000) -- (101.3000,256.5000) --
          (108.1000,257.6000) -- (110.4000,259.9000) -- (116.0000,255.8000) --
          (124.1000,255.1000) -- (116.5000,258.0000) -- (115.8000,265.4000) --
          (112.4000,266.9000) -- (119.0000,269.3000) -- (124.4000,274.9000) --
          (115.7000,284.2000) -- (116.1000,289.2000) -- (127.8000,302.9000) --
          (130.2000,310.8000) -- (139.3000,318.3000) -- (144.7000,319.6000) --
          (146.5000,323.5000) -- (153.1000,324.8000) -- (158.4000,329.1000) --
          (159.2000,326.0000) -- (162.4000,326.5000) -- (157.3000,337.3000) --
          (160.6000,338.4000) -- (165.2000,347.3000) -- (162.1000,347.9000) --
          (163.9000,353.9000) -- (160.6000,360.3000) -- (155.2000,364.0000) --
          (156.7000,368.8000) -- (165.5000,374.3000) -- (177.2000,386.9000) --
          (179.5000,395.2000) -- (179.5000,395.3000) -- (180.2000,398.4000) --
          (177.8000,402.1000) -- (175.0000,393.4000) -- (162.9000,381.6000) --
          (161.9000,378.4000) -- (158.6000,384.9000) -- (151.6000,441.2000) --
          (156.1000,432.4000) -- (161.1000,438.2000) -- (161.4000,441.4000) --
          (154.2000,440.4000) -- (150.6000,448.2000) -- (150.7000,453.5000) --
          (138.3000,509.7000) -- (134.1000,517.6000) -- (134.0000,517.7000) --
          (128.8000,524.7000) -- (121.8000,529.9000);

        % polyline5288
        \path[draw=c666666,line join=round,line cap=round,thin]
          (335.1000,0.6000) -- (316.9000,5.1000) -- (301.5000,10.0000) --
          (295.0000,15.6000) -- (294.9000,35.8000) -- (298.2000,39.7000) --
          (295.1000,37.7000) -- (294.4000,44.4000) -- (294.9000,48.6000) --
          (297.4000,50.5000) -- (293.3000,52.1000) -- (292.9000,55.4000) --
          (296.3000,61.6000) -- (293.1000,60.0000) -- (285.8000,69.9000) --
          (285.6000,70.3000) -- (273.4000,78.8000) -- (248.6000,84.9000) --
          (232.7000,93.6000) -- (226.2000,107.7000) -- (228.1000,110.2000) --
          (238.0000,112.5000) -- (238.2000,112.6000) -- (236.4000,113.2000) --
          (236.3000,113.3000) -- (228.6000,115.0000) -- (223.1000,119.8000) --
          (212.3000,122.9000) -- (202.8000,118.9000) -- (188.8000,117.4000) --
          (178.1000,113.5000) -- (172.3000,116.1000) -- (172.1000,116.2000) --
          (164.9000,104.4000) -- (167.6000,95.5000) -- (166.5000,92.4000) --
          (159.2000,91.7000) -- (156.4000,93.7000) -- (149.0000,94.2000) --
          (136.3000,89.0000) -- (136.1000,92.1000) -- (139.7000,94.2000) --
          (140.3000,99.7000) -- (138.2000,101.9000) -- (141.0000,113.4000) --
          (150.6000,123.7000) -- (149.1000,135.0000) -- (151.3000,138.1000) --
          (150.7000,144.9000) -- (148.2000,151.3000) -- (152.9000,161.9000) --
          (155.8000,163.2000) -- (158.6000,161.4000) -- (157.5000,164.8000) --
          (149.5000,165.6000) -- (136.6000,165.7000) -- (136.9000,159.4000) --
          (133.3000,160.2000) -- (128.8000,164.5000) -- (131.4000,170.3000) --
          (131.7000,170.8000) -- (129.1000,168.7000) -- (129.0000,168.5000) --
          (127.1000,163.7000) -- (124.1000,163.7000) -- (123.9000,163.7000) --
          (123.9000,165.8000) -- (119.3000,166.2000) -- (118.2000,162.7000) --
          (114.7000,164.3000) -- (113.6000,161.4000) -- (109.5000,161.7000) --
          (97.5000,171.7000) -- (96.5000,167.6000) -- (91.9000,164.6000) --
          (91.9000,161.1000) -- (86.8000,152.3000) -- (83.5000,151.6000) --
          (84.2000,148.5000) -- (80.8000,148.9000) -- (78.9000,152.2000) --
          (80.4000,144.8000) -- (75.1000,148.4000) -- (74.5000,144.4000) --
          (64.5000,148.5000) -- (62.9000,145.7000) -- (57.8000,149.9000) --
          (57.5000,156.9000) -- (54.0000,157.4000) -- (53.8000,157.4000) --
          (48.4000,154.0000) -- (45.2000,154.7000) -- (44.5000,158.5000) --
          (42.0000,155.5000) -- (40.2000,159.1000) -- (40.1000,151.9000) --
          (23.9000,156.6000) -- (22.2000,153.6000) -- (10.8000,159.9000) --
          (13.7000,161.4000) -- (4.3000,160.4000) -- (2.1000,163.1000) --
          (0.6000,173.2000) -- (2.6000,175.8000) -- (5.6000,174.5000) --
          (8.6000,175.9000) -- (22.4000,171.7000) -- (18.2000,173.7000) --
          (17.6000,177.5000) -- (24.3000,175.9000) -- (23.8000,179.2000) --
          (27.1000,180.6000) -- (23.6000,181.3000) -- (30.6000,184.3000) --
          (23.9000,182.7000) -- (22.4000,179.9000) -- (18.2000,181.1000) --
          (11.7000,179.8000) -- (9.8000,177.1000) -- (8.1000,181.2000) --
          (10.8000,187.0000) -- (12.8000,184.0000) -- (15.9000,184.2000) --
          (22.0000,187.8000) -- (23.0000,191.4000) -- (21.2000,194.2000) --
          (18.4000,192.7000) -- (4.3000,194.1000) -- (2.6000,196.9000) --
          (13.3000,200.9000) -- (17.5000,207.9000) -- (16.2000,214.0000) --
          (24.5000,214.7000) -- (28.6000,210.1000) -- (28.6000,209.6000);

        % polyline5290
        \path[draw=c666666,line join=round,line cap=round,thin]
          (28.9000,206.1000) -- (30.8000,205.9000) -- (29.2000,202.7000) --
          (28.9000,206.1000);

        % polyline5292
        \path[draw=c666666,line join=round,line cap=round,thin]
          (28.9000,206.1000) -- (27.7000,206.3000) -- (28.6000,209.6000) --
          (28.6000,209.6000);

        % line5294
        \path[draw=c666666,line join=round,line cap=round,thin]
          (28.6000,209.6000) -- (28.9000,206.1000);

        \begin{scope}% g5296
          % path5298
          \path[fill=c666666] (525.8000,548.3000) .. controls (525.8000,548.3000) and
            (525.8000,548.3000) .. (525.7000,548.3000) .. controls (525.6000,548.3000) and
            (525.6000,548.3000) .. (525.6000,548.3000) .. controls (525.6000,548.3000) and
            (525.6000,548.3000) .. (525.5000,548.4000) .. controls (525.5000,548.4000) and
            (525.5000,548.4000) .. (525.4000,548.4000) -- (525.4000,548.5000) .. controls
            (525.4000,548.5000) and (525.3000,548.6000) .. (525.3000,548.7000) .. controls
            (525.3000,548.7000) and (525.3000,548.7000) .. (525.3000,548.8000) .. controls
            (525.3000,548.8000) and (525.3000,548.9000) .. (525.3000,549.0000) .. controls
            (525.4000,549.0000) and (525.4000,549.1000) .. (525.4000,549.1000) .. controls
            (525.5000,549.2000) and (525.5000,549.2000) .. (525.6000,549.2000) .. controls
            (525.6000,549.2000) and (525.6000,549.2000) .. (525.7000,549.2000) --
            (525.8000,549.2000) -- (525.8000,549.2000) .. controls (525.9000,549.2000) and
            (525.9000,549.2000) .. (525.9000,549.2000) -- (525.9000,549.2000) --
            (525.9000,549.0000) .. controls (525.9000,549.0000) and (525.9000,549.0000) ..
            (525.8000,549.1000) .. controls (525.8000,549.1000) and (525.8000,549.1000) ..
            (525.7000,549.1000) .. controls (525.6000,549.1000) and (525.6000,549.1000) ..
            (525.6000,549.0000) .. controls (525.6000,549.0000) and (525.6000,549.0000) ..
            (525.5000,549.0000) -- (525.5000,548.9000) -- (525.5000,548.8000) .. controls
            (525.5000,548.7000) and (525.5000,548.7000) .. (525.5000,548.7000) .. controls
            (525.5000,548.6000) and (525.5000,548.5000) .. (525.5000,548.5000) .. controls
            (525.6000,548.5000) and (525.6000,548.5000) .. (525.6000,548.5000) .. controls
            (525.6000,548.4000) and (525.7000,548.4000) .. (525.7000,548.4000) .. controls
            (525.8000,548.4000) and (525.8000,548.4000) .. (525.8000,548.4000) .. controls
            (525.9000,548.5000) and (525.9000,548.5000) .. (525.9000,548.5000) --
            (525.9000,548.4000) .. controls (525.9000,548.3000) and (525.9000,548.3000) ..
            (525.8000,548.3000) -- cycle;

          % path5300
          \path[fill=c666666] (526.6000,548.5000) -- (526.6000,548.5000) .. controls
            (526.6000,548.6000) and (526.6000,548.6000) .. (526.6000,548.7000) --
            (526.6000,549.2000) -- (526.8000,549.2000) -- (526.8000,548.6000) .. controls
            (526.9000,548.5000) and (526.9000,548.5000) .. (526.9000,548.5000) .. controls
            (527.0000,548.4000) and (527.0000,548.4000) .. (527.1000,548.4000) --
            (527.1000,548.4000) .. controls (527.1000,548.5000) and (527.1000,548.5000) ..
            (527.1000,548.5000) .. controls (527.2000,548.5000) and (527.2000,548.5000) ..
            (527.2000,548.5000) .. controls (527.2000,548.6000) and (527.2000,548.6000) ..
            (527.2000,548.7000) -- (527.2000,549.2000) -- (527.4000,549.2000) --
            (527.4000,548.7000) .. controls (527.4000,548.6000) and (527.4000,548.5000) ..
            (527.4000,548.5000) .. controls (527.3000,548.4000) and (527.3000,548.4000) ..
            (527.3000,548.4000) .. controls (527.3000,548.3000) and (527.2000,548.3000) ..
            (527.1000,548.3000) -- (527.1000,548.3000) .. controls (527.0000,548.3000) and
            (527.0000,548.3000) .. (526.9000,548.4000) .. controls (526.9000,548.4000) and
            (526.8000,548.4000) .. (526.8000,548.5000) .. controls (526.8000,548.4000) and
            (526.7000,548.4000) .. (526.7000,548.4000) .. controls (526.6000,548.3000) and
            (526.6000,548.3000) .. (526.6000,548.3000) .. controls (526.5000,548.3000) and
            (526.5000,548.3000) .. (526.5000,548.3000) .. controls (526.4000,548.3000) and
            (526.4000,548.3000) .. (526.4000,548.3000) .. controls (526.4000,548.4000) and
            (526.3000,548.4000) .. (526.3000,548.4000) .. controls (526.3000,548.4000) and
            (526.3000,548.4000) .. (526.3000,548.5000) -- (526.3000,548.5000) --
            (526.3000,548.3000) -- (526.1000,548.3000) -- (526.1000,549.2000) --
            (526.3000,549.2000) -- (526.3000,548.6000) .. controls (526.3000,548.5000) and
            (526.4000,548.5000) .. (526.4000,548.5000) .. controls (526.4000,548.4000) and
            (526.5000,548.4000) .. (526.5000,548.4000) .. controls (526.6000,548.4000) and
            (526.6000,548.4000) .. (526.6000,548.4000) .. controls (526.6000,548.5000) and
            (526.6000,548.5000) .. (526.6000,548.5000) -- cycle;

          % path5302
          \path[fill=c666666] (527.8000,548.3000) -- (527.8000,548.3000) --
            (527.7000,548.3000) .. controls (527.7000,548.3000) and (527.6000,548.3000) ..
            (527.6000,548.4000) -- (527.6000,548.5000) .. controls (527.7000,548.5000) and
            (527.7000,548.5000) .. (527.8000,548.4000) -- (527.9000,548.4000) .. controls
            (527.9000,548.4000) and (528.0000,548.4000) .. (528.0000,548.5000) .. controls
            (528.1000,548.5000) and (528.1000,548.5000) .. (528.1000,548.5000) --
            (528.1000,548.6000) -- (528.1000,548.7000) -- (528.1000,548.7000) --
            (528.1000,548.7000) -- (528.0000,548.7000) -- (527.9000,548.7000) --
            (527.8000,548.7000) .. controls (527.7000,548.7000) and (527.7000,548.7000) ..
            (527.6000,548.7000) .. controls (527.6000,548.8000) and (527.6000,548.8000) ..
            (527.6000,548.8000) -- (527.6000,548.9000) .. controls (527.6000,549.0000) and
            (527.6000,549.0000) .. (527.6000,549.0000) .. controls (527.6000,549.1000) and
            (527.6000,549.2000) .. (527.6000,549.2000) .. controls (527.7000,549.2000) and
            (527.7000,549.2000) .. (527.8000,549.2000) .. controls (527.8000,549.2000) and
            (527.8000,549.2000) .. (527.9000,549.2000) -- (527.9000,549.2000) .. controls
            (528.0000,549.2000) and (528.0000,549.2000) .. (528.0000,549.2000) --
            (528.1000,549.2000) -- (528.1000,549.1000) -- (528.1000,549.2000) --
            (528.3000,549.2000) -- (528.3000,548.7000) .. controls (528.3000,548.6000) and
            (528.3000,548.6000) .. (528.3000,548.5000) -- (528.2000,548.5000) .. controls
            (528.2000,548.4000) and (528.1000,548.4000) .. (528.1000,548.4000) .. controls
            (528.1000,548.3000) and (528.0000,548.3000) .. (527.9000,548.3000) .. controls
            (527.9000,548.3000) and (527.9000,548.3000) .. (527.8000,548.3000) --
            cycle(528.1000,548.8000) -- (528.1000,549.0000) -- (528.1000,549.0000) ..
            controls (528.1000,549.0000) and (528.1000,549.0000) .. (528.0000,549.0000) --
            (528.0000,549.1000) .. controls (527.9000,549.1000) and (527.9000,549.1000) ..
            (527.9000,549.1000) -- (527.8000,549.1000) -- (527.8000,549.0000) .. controls
            (527.8000,549.0000) and (527.8000,549.0000) .. (527.7000,549.0000) --
            (527.7000,548.9000) .. controls (527.7000,548.9000) and (527.7000,548.8000) ..
            (527.8000,548.8000) -- (527.9000,548.8000) .. controls (528.0000,548.8000) and
            (528.0000,548.8000) .. (528.0000,548.8000) .. controls (528.1000,548.8000) and
            (528.1000,548.8000) .. (528.1000,548.8000) -- cycle;

          % path5304
          \path[fill=c666666] (529.2000,548.6000) .. controls (529.2000,548.5000) and
            (529.1000,548.5000) .. (529.1000,548.4000) .. controls (529.1000,548.4000) and
            (529.1000,548.3000) .. (529.0000,548.3000) -- (528.9000,548.3000) --
            (528.8000,548.3000) .. controls (528.8000,548.3000) and (528.7000,548.3000) ..
            (528.7000,548.4000) -- (528.6000,548.4000) -- (528.6000,548.4000) --
            (528.6000,548.4000) -- (528.6000,548.3000) -- (528.5000,548.3000) --
            (528.5000,549.6000) -- (528.6000,549.6000) -- (528.6000,549.1000) .. controls
            (528.6000,549.2000) and (528.7000,549.2000) .. (528.8000,549.2000) .. controls
            (528.8000,549.2000) and (528.8000,549.2000) .. (528.9000,549.2000) .. controls
            (528.9000,549.2000) and (528.9000,549.2000) .. (529.0000,549.2000) --
            (529.0000,549.2000) .. controls (529.1000,549.2000) and (529.1000,549.2000) ..
            (529.1000,549.2000) .. controls (529.1000,549.1000) and (529.1000,549.0000) ..
            (529.1000,549.0000) .. controls (529.2000,549.0000) and (529.2000,548.9000) ..
            (529.2000,548.9000) .. controls (529.3000,548.9000) and (529.3000,548.8000) ..
            (529.3000,548.7000) .. controls (529.3000,548.7000) and (529.3000,548.7000) ..
            (529.2000,548.6000) -- cycle(528.9000,548.4000) .. controls
            (528.9000,548.4000) and (528.9000,548.4000) .. (529.0000,548.5000) --
            (529.0000,548.5000) .. controls (529.1000,548.5000) and (529.1000,548.6000) ..
            (529.1000,548.7000) -- (529.1000,548.7000) .. controls (529.1000,548.8000) and
            (529.1000,548.9000) .. (529.1000,548.9000) .. controls (529.1000,548.9000) and
            (529.1000,549.0000) .. (529.0000,549.0000) .. controls (529.0000,549.0000) and
            (529.0000,549.0000) .. (528.9000,549.1000) -- (528.9000,549.1000) .. controls
            (528.8000,549.1000) and (528.8000,549.1000) .. (528.7000,549.0000) --
            (528.6000,549.0000) -- (528.6000,548.5000) -- (528.6000,548.5000) --
            (528.7000,548.5000) .. controls (528.7000,548.5000) and (528.8000,548.5000) ..
            (528.8000,548.4000) .. controls (528.8000,548.4000) and (528.8000,548.4000) ..
            (528.9000,548.4000) -- cycle;

        \end{scope}
      \end{scope}
    \end{scope}
    \begin{scope}% g5306
      % path5308
      \path[draw,color=cFFFFFF,densely dotted] (327.8000,150.9000) -- (324.9000,152.7000) --
        (328.3000,156.7000) -- (334.2000,156.6000) -- (332.2000,154.5000) --
        (332.2000,154.4000) -- (331.4000,151.4000) -- (327.8000,150.9000) -- cycle;

      % polyline5310
      \path[draw=c999999,line join=round,line cap=round,very thin,densely dotted]
        (327.8000,150.9000) -- (324.9000,152.7000) -- (328.3000,156.7000);

      % line5312
      \path[draw=c999999,line join=round,line cap=round,very thin,densely dotted]
        (328.3000,156.7000) -- (327.8000,161.3000);

      % polyline5314
      \path[draw=c999999,line join=round,line cap=round,very thin,densely dotted]
        (332.2000,154.4000) -- (331.4000,151.4000) -- (327.8000,150.9000);

      % line5316
      \path[draw=c999999,line join=round,line cap=round,very thin,densely dotted]
        (332.2000,154.4000) -- (332.2000,154.5000);

      % polyline5318
      \path[draw=c999999,line join=round,line cap=round,very thin,densely dotted]
        (332.2000,154.5000) -- (334.2000,156.6000) -- (328.3000,156.7000);

      % line5320
      \path[draw=c999999,line join=round,line cap=round,very thin,densely dotted]
        (340.2000,157.4000) -- (332.2000,154.5000);

      % line5322
      \path[draw=c999999,line join=round,line cap=round,very thin,densely dotted]
        (332.2000,154.5000) -- (332.2000,154.4000);

      % polyline5324
      \path[draw=c999999,line join=round,line cap=round,very thin,densely dotted]
        (326.4000,147.4000) -- (327.9000,149.6000) -- (327.8000,150.9000);

    \end{scope}
    \begin{scope}% g5326
      \begin{scope}% g5328
        % path5330
        \path[draw,color=cFFFFFF,densely dotted] (575.6000,6.8000) -- (575.5000,3.3000) -- (572.5000,3.3000)
          -- (568.7000,1.5000) -- (565.5000,5.8000) -- (562.9000,6.3000) --
          (559.8000,8.8000) -- (555.1000,15.4000) -- (549.1000,15.4000) --
          (546.1000,16.8000) -- (542.6000,16.2000) -- (538.1000,10.3000) --
          (535.6000,10.3000) -- (533.4000,13.8000) -- (529.9000,13.3000) --
          (529.9000,9.3000) -- (528.6000,9.7000) -- (518.0000,10.5000) --
          (517.4000,13.8000) -- (521.8000,16.3000) -- (521.9000,16.3000) --
          (526.0000,16.0000) -- (528.6000,18.9000) -- (528.6000,20.0000) --
          (523.0000,25.4000) -- (523.0000,28.5000) -- (529.8000,28.5000) --
          (536.0000,31.5000) -- (542.0000,37.2000) -- (542.0000,37.5000) --
          (544.5000,39.9000) -- (554.4000,45.0000) -- (557.9000,40.5000) --
          (560.3000,42.8000) -- (567.9000,48.8000) -- (575.5000,56.8000) --
          (575.0000,45.6000) -- (576.5000,43.8000) -- (576.5000,40.6000) --
          (574.1000,38.9000) -- (573.7000,32.2000) -- (576.7000,27.7000) --
          (576.7000,25.4000) -- (580.5000,24.2000) -- (576.7000,18.4000) --
          (577.1000,16.0000) -- (574.0000,13.5000) -- (574.0000,8.3000) --
          (575.6000,6.8000) -- cycle;

        % path5332
        \path[draw,color=cFFFFFF,densely dotted] (557.9000,40.5000) -- (554.4000,45.0000) --
          (544.5000,39.9000) -- (542.0000,37.5000) -- (542.0000,41.4000) --
          (540.0000,47.2000) -- (546.4000,49.9000) -- (550.9000,50.2000) --
          (550.9000,55.7000) -- (549.5000,56.5000) -- (546.7000,52.5000) --
          (541.1000,52.5000) -- (540.5000,54.3000) -- (534.4000,54.3000) --
          (534.4000,55.2000) -- (530.1000,55.2000) -- (526.8000,58.3000) --
          (526.8000,75.5000) -- (527.1000,75.1000) -- (531.4000,75.3000) --
          (533.6000,80.2000) -- (544.2000,83.5000) -- (548.8000,83.5000) --
          (552.4000,79.6000) -- (554.6000,79.2000) -- (556.3000,80.1000) --
          (560.8000,79.9000) -- (561.3000,82.1000) -- (565.2000,87.1000) --
          (569.5000,91.1000) -- (572.5000,91.1000) -- (575.1000,87.1000) --
          (579.0000,84.4000) -- (579.0000,82.1000) -- (576.2000,80.4000) --
          (576.7000,76.7000) -- (578.6000,74.0000) -- (581.1000,73.4000) --
          (582.6000,70.6000) -- (579.1000,66.0000) -- (579.1000,62.5000) --
          (575.5000,56.8000) -- (567.9000,48.8000) -- (560.3000,42.8000) --
          (557.9000,40.5000) -- cycle;

        % path5334
        \path[draw,color=cFFFFFF,densely dotted] (542.0000,41.4000) -- (542.0000,37.5000) --
          (542.0000,37.2000) -- (536.0000,31.5000) -- (529.8000,28.5000) --
          (523.0000,28.5000) -- (523.0000,31.1000) -- (514.6000,37.0000) --
          (509.4000,37.0000) -- (506.0000,41.4000) -- (506.1000,43.9000) --
          (508.6000,45.9000) -- (513.9000,45.9000) -- (513.9000,47.8000) --
          (519.9000,50.2000) -- (524.9000,55.2000) -- (530.1000,55.2000) --
          (534.4000,55.2000) -- (534.4000,54.3000) -- (540.5000,54.3000) --
          (541.1000,52.5000) -- (546.7000,52.5000) -- (549.5000,56.5000) --
          (550.9000,55.7000) -- (550.9000,50.2000) -- (546.4000,49.9000) --
          (540.0000,47.2000) -- (542.0000,41.4000) -- cycle;

        % path5336
        \path[draw,color=cFFFFFF,densely dotted] (521.9000,16.3000) -- (521.9000,16.3000) --
          (520.1000,18.5000) -- (501.7000,28.6000) -- (499.0000,31.5000) --
          (499.0000,36.9000) -- (487.8000,38.9000) -- (487.8000,42.8000) --
          (490.3000,45.0000) -- (489.5000,50.3000) -- (491.6000,54.9000) --
          (502.8000,62.3000) -- (502.8000,63.8000) -- (517.1000,72.9000) --
          (514.4000,76.1000) -- (514.4000,78.6000) -- (520.0000,78.6000) --
          (522.4000,82.5000) -- (524.1000,82.3000) -- (524.1000,78.4000) --
          (526.8000,75.5000) -- (526.8000,58.3000) -- (530.1000,55.2000) --
          (524.9000,55.2000) -- (519.9000,50.2000) -- (513.9000,47.8000) --
          (513.9000,45.9000) -- (508.6000,45.9000) -- (506.1000,43.9000) --
          (506.0000,41.4000) -- (509.4000,37.0000) -- (514.6000,37.0000) --
          (523.0000,31.1000) -- (523.0000,28.5000) -- (523.0000,25.4000) --
          (528.6000,20.0000) -- (528.6000,18.9000) -- (526.0000,16.0000) --
          (521.9000,16.3000) -- cycle;

        % polyline5338
        \path[draw=cCCCCCC,line join=round,line cap=round,very thin,densely dotted]
          (526.8000,75.5000) -- (526.8000,58.3000) -- (530.1000,55.2000);

        % polyline5340
        \path[draw=cCCCCCC,line join=round,line cap=round,very thin,densely dotted]
          (523.0000,28.5000) -- (523.0000,25.4000) -- (528.6000,20.0000) --
          (528.6000,18.9000) -- (526.0000,16.0000) -- (521.9000,16.3000) --
          (521.8000,16.3000);

        % polyline5342
        \path[draw=cCCCCCC,line join=round,line cap=round,very thin,densely dotted]
          (575.5000,56.8000) -- (567.9000,48.8000) -- (560.3000,42.8000) --
          (557.9000,40.5000) -- (554.4000,45.0000) -- (544.5000,39.9000) --
          (542.0000,37.5000);

        % polyline5344
        \path[draw=cCCCCCC,line join=round,line cap=round,very thin,densely dotted]
          (542.0000,37.5000) -- (542.0000,41.4000) -- (540.0000,47.2000) --
          (546.4000,49.9000) -- (550.9000,50.2000) -- (550.9000,55.7000) --
          (549.5000,56.5000) -- (546.7000,52.5000) -- (541.1000,52.5000) --
          (540.5000,54.3000) -- (534.4000,54.3000) -- (534.4000,55.2000) --
          (530.1000,55.2000);

        % polyline5346
        \path[draw=cCCCCCC,line join=round,line cap=round,very thin,densely dotted]
          (542.0000,37.5000) -- (542.0000,37.2000) -- (536.0000,31.5000) --
          (529.8000,28.5000) -- (523.0000,28.5000);

        % polyline5348
        \path[draw=cCCCCCC,line join=round,line cap=round,very thin,densely dotted]
          (530.1000,55.2000) -- (524.9000,55.2000) -- (519.9000,50.2000) --
          (513.9000,47.8000) -- (513.9000,45.9000) -- (508.6000,45.9000) --
          (506.1000,43.9000) -- (506.0000,41.4000) -- (509.4000,37.0000) --
          (514.6000,37.0000) -- (523.0000,31.1000) -- (523.0000,28.5000);

        % polyline5350
        \path[draw=c666666,line join=round,line cap=round,thin]
          (535.6000,10.3000) -- (533.4000,13.8000) -- (529.9000,13.3000) --
          (529.9000,9.3000);

        % polyline5352
        \path[draw=c666666,line join=round,line cap=round,thin]
          (549.1000,15.4000) -- (546.1000,16.8000) -- (542.6000,16.2000) --
          (538.1000,10.3000) -- (535.6000,10.3000);

        % polyline5354
        \path[draw=c666666,line join=round,line cap=round,thin]
          (518.0000,10.5000) -- (517.4000,13.8000) -- (521.8000,16.3000) --
          (521.9000,16.3000) -- (520.1000,18.5000) -- (501.7000,28.6000) --
          (499.0000,31.5000) -- (499.0000,36.9000) -- (487.8000,38.9000) --
          (487.8000,42.8000) -- (490.3000,45.0000) -- (489.5000,50.3000) --
          (491.6000,54.9000) -- (502.8000,62.3000) -- (502.8000,63.8000) --
          (517.1000,72.9000) -- (514.4000,76.1000) -- (514.4000,78.6000) --
          (520.0000,78.6000) -- (522.4000,82.5000) -- (524.1000,82.3000) --
          (524.1000,78.4000) -- (526.8000,75.5000) -- (527.1000,75.1000) --
          (531.4000,75.3000) -- (533.6000,80.2000) -- (544.2000,83.5000) --
          (548.8000,83.5000) -- (552.4000,79.6000) -- (554.6000,79.2000) --
          (556.3000,80.1000) -- (560.8000,79.9000) -- (561.3000,82.1000) --
          (565.2000,87.1000) -- (569.5000,91.1000) -- (572.5000,91.1000) --
          (575.1000,87.1000) -- (579.0000,84.4000) -- (579.0000,82.1000) --
          (576.2000,80.4000) -- (576.7000,76.7000) -- (578.6000,74.0000) --
          (581.1000,73.4000) -- (582.6000,70.6000) -- (579.1000,66.0000) --
          (579.1000,62.5000);

        % line5356
        \path[draw=c666666,line join=round,line cap=round,thin]
          (579.1000,62.5000) -- (575.5000,56.8000);

        % polyline5358
        \path[draw=c666666,line join=round,line cap=round,thin]
          (529.9000,9.3000) -- (528.6000,9.7000) -- (518.0000,10.5000);

        % polyline5360
        \path[draw=c666666,line join=round,line cap=round,thin]
          (575.5000,56.8000) -- (575.0000,45.6000) -- (576.5000,43.8000) --
          (576.5000,40.6000) -- (574.1000,38.9000) -- (573.7000,32.2000) --
          (576.7000,27.7000) -- (576.7000,25.4000) -- (580.5000,24.2000) --
          (576.7000,18.4000) -- (577.1000,16.0000) -- (574.0000,13.5000) --
          (574.0000,8.3000) -- (575.6000,6.8000) -- (575.5000,3.3000) --
          (572.5000,3.3000) -- (568.7000,1.5000) -- (565.5000,5.8000) --
          (562.9000,6.3000) -- (559.8000,8.8000) -- (555.1000,15.4000) --
          (549.1000,15.4000);

      \end{scope}
    \end{scope}
  \end{scope}
\end{scope}
\begin{scope}% regions
  \begin{scope}% g5363
    % polyline5365
    \path[draw=c999999,line join=round,line cap=round,very thin,densely dotted]
      (146.1000,332.3000) -- (146.2000,332.2000) -- (146.3000,332.2000) --
      (146.2000,332.1000) -- (143.5000,329.7000) -- (143.5000,329.9000) --
      (144.6000,332.7000) -- (145.3000,332.4000);

    % polyline5367
    \path[draw=c666666,line join=round,line cap=round,very thin,densely dotted]
      (463.1000,520.2000) -- (467.7000,520.2000) -- (470.0000,524.7000) --
      (473.7000,523.2000) -- (471.3000,527.5000) -- (467.7000,528.7000) --
      (462.5000,523.9000) -- (463.1000,520.2000);

    % polyline5369
    \path[draw=c666666,line join=round,line cap=round,thin]
      (121.8000,529.9000) -- (128.3000,533.0000) -- (128.8000,536.5000) --
      (131.3000,534.2000) -- (138.8000,535.7000) -- (140.4000,538.8000) --
      (138.3000,545.5000) -- (135.3000,548.1000) -- (136.8000,551.3000) --
      (141.1000,552.3000) -- (141.7000,548.2000) -- (145.0000,547.2000) --
      (145.2000,551.0000) -- (151.8000,554.5000) -- (162.7000,558.7000) --
      (169.7000,558.4000) -- (172.6000,563.6000) -- (181.4000,570.8000) --
      (184.1000,569.0000) -- (186.7000,570.6000) -- (193.3000,567.2000) --
      (193.5000,567.3000) -- (206.0000,578.4000) -- (219.8000,576.5000) --
      (222.6000,578.9000) -- (233.6000,578.7000) -- (240.7000,579.1000) --
      (243.2000,576.7000) -- (242.8000,569.3000) -- (245.6000,567.8000) --
      (253.2000,570.4000) -- (264.3000,573.3000) -- (269.8000,578.2000) --
      (276.9000,577.8000) -- (283.3000,586.2000) -- (284.5000,583.0000) --
      (288.2000,582.8000) -- (300.1000,588.5000) -- (300.1000,588.5000) --
      (298.0000,593.9000) -- (307.8000,596.9000) -- (310.5000,603.2000) --
      (314.1000,603.5000) -- (319.1000,599.0000) -- (322.6000,598.8000) --
      (333.3000,601.2000) -- (338.4000,605.2000) -- (345.5000,604.7000) --
      (345.5000,601.4000) -- (354.6000,596.6000) -- (361.6000,595.4000) --
      (371.3000,597.9000);

    % polyline5371
    \path[draw=c666666,line join=round,line cap=round,thin]
      (586.4000,497.6000) -- (586.9000,489.5000) -- (593.9000,479.8000) --
      (596.1000,474.9000) -- (591.6000,469.0000) -- (578.3000,473.8000) --
      (568.0000,468.3000) -- (561.0000,466.9000) -- (554.1000,458.6000) --
      (553.9000,458.4000) -- (556.0000,452.9000) -- (551.6000,446.8000) --
      (554.1000,444.7000) -- (555.6000,438.1000) -- (555.7000,437.9000) --
      (555.8000,437.8000) -- (557.9000,435.4000) -- (561.9000,435.5000) --
      (557.6000,425.0000) -- (550.8000,424.6000) -- (545.0000,420.9000) --
      (544.0000,413.6000) -- (540.9000,413.4000) -- (538.4000,407.6000) --
      (538.3000,407.5000) -- (560.0000,398.2000) -- (562.2000,394.1000) --
      (560.9000,391.1000) -- (564.1000,385.3000) -- (555.8000,379.6000) --
      (553.4000,373.4000) -- (554.1000,369.9000) -- (547.9000,367.4000) --
      (544.6000,364.4000) -- (544.1000,360.8000) -- (544.3000,360.6000) --
      (553.6000,351.6000) -- (554.6000,348.1000) -- (550.9000,342.9000) --
      (542.1000,334.6000) -- (544.7000,327.7000) -- (540.6000,321.6000) --
      (541.1000,318.4000) -- (537.3000,317.7000) -- (527.9000,318.3000) --
      (522.3000,323.0000) -- (518.3000,321.4000) -- (515.2000,327.9000) --
      (517.9000,331.4000) -- (510.0000,337.4000) -- (502.2000,338.5000) --
      (502.2000,338.4000) -- (502.8000,332.8000) -- (509.1000,328.8000) --
      (510.5000,321.0000) -- (506.6000,318.7000) -- (506.4000,318.6000) --
      (510.4000,305.0000) -- (523.5000,292.9000) -- (522.6000,282.0000) --
      (532.3000,275.4000) -- (545.4000,259.1000) -- (545.2000,255.9000) --
      (547.8000,253.6000) -- (542.9000,249.5000) -- (544.0000,246.2000) --
      (548.3000,241.4000) -- (552.6000,240.9000) -- (553.0000,240.9000) --
      (555.9000,241.6000) -- (555.0000,244.9000) -- (563.6000,245.0000) --
      (566.6000,243.9000) -- (567.1000,240.4000) -- (570.4000,240.3000) --
      (570.6000,236.8000) -- (573.1000,234.5000) -- (569.9000,225.8000) --
      (572.9000,208.5000) -- (570.7000,196.9000) -- (577.4000,181.3000) --
      (579.0000,163.7000) -- (590.8000,148.4000) -- (595.2000,136.4000) --
      (585.6000,133.0000) -- (569.3000,132.4000) -- (569.1000,132.4000) --
      (559.7000,124.0000) -- (552.6000,128.6000) -- (544.8000,127.3000) --
      (542.0000,128.7000) -- (539.5000,124.3000) -- (536.1000,122.7000) --
      (532.9000,124.5000) -- (532.6000,127.7000) -- (529.5000,126.9000) --
      (520.5000,116.7000) -- (518.6000,110.1000) -- (515.3000,107.8000) --
      (502.8000,104.9000) -- (497.9000,108.9000) -- (494.7000,109.0000) --
      (489.0000,106.4000) -- (485.0000,103.0000) -- (477.3000,103.0000) --
      (475.9000,106.1000) -- (469.6000,107.0000) -- (465.7000,98.9000) --
      (462.1000,99.2000) -- (462.1000,95.6000) -- (455.3000,93.9000) --
      (447.8000,87.3000) -- (441.1000,87.5000) -- (442.2000,80.2000) --
      (439.0000,73.7000) -- (439.0000,69.1000) -- (441.9000,63.4000) --
      (438.9000,61.6000) -- (434.3000,66.5000) -- (432.4000,73.7000) --
      (422.6000,78.1000) -- (416.5000,76.1000) -- (413.3000,76.9000) --
      (413.0000,76.9000) -- (408.8000,75.6000) -- (409.0000,75.3000) --
      (412.6000,68.5000) -- (409.2000,62.5000) -- (411.5000,55.3000) --
      (408.3000,56.0000) -- (403.8000,50.4000) -- (397.0000,51.7000) --
      (390.8000,50.2000) -- (388.4000,52.4000) -- (385.9000,42.2000) --
      (379.8000,40.8000) -- (378.6000,37.7000) -- (375.3000,40.2000) --
      (371.8000,38.9000) -- (368.3000,29.4000) -- (368.8000,26.1000) --
      (364.5000,21.1000) -- (357.9000,21.7000) -- (351.7000,26.0000) --
      (348.3000,24.5000) -- (345.1000,18.7000) -- (341.4000,18.7000) --
      (340.0000,15.7000) -- (341.1000,9.2000) -- (338.3000,2.5000) --
      (335.1000,0.6000);

    % polyline5373
    \path[draw=c666666,line join=round,line cap=round,thin]
      (562.1000,557.3000) -- (562.6000,576.9000) -- (553.0000,575.4000) --
      (549.0000,580.8000) -- (537.0000,585.6000) -- (536.1000,588.8000) --
      (532.9000,589.1000) -- (530.4000,592.3000) -- (530.7000,596.2000) --
      (526.3000,601.8000) -- (525.7000,605.1000) -- (528.6000,604.0000) --
      (532.9000,608.7000) -- (527.1000,612.9000) -- (528.3000,619.1000) --
      (537.1000,624.0000) -- (534.2000,625.8000) -- (530.9000,635.2000) --
      (539.0000,632.4000) -- (540.0000,638.8000) -- (537.0000,646.7000) --
      (547.7000,648.8000) -- (542.2000,652.5000) -- (542.0000,655.9000) --
      (547.3000,660.8000) -- (559.9000,664.7000) -- (561.3000,667.8000) --
      (564.7000,669.2000) -- (571.3000,652.6000) -- (567.6000,652.3000) --
      (570.5000,650.1000) -- (572.9000,644.0000) -- (572.0000,634.3000) --
      (572.1000,627.3000) -- (578.1000,614.0000) -- (573.9000,584.8000) --
      (569.3000,578.4000) -- (569.9000,566.5000) -- (568.0000,556.2000) --
      (564.8000,555.0000) -- (562.1000,557.3000);

    % polyline5375
    \path[draw=c666666,line join=round,line cap=round,thin]
      (65.6000,248.4000) -- (66.5000,252.2000) -- (70.1000,253.2000) --
      (74.5000,251.9000) -- (68.8000,248.0000) -- (65.6000,248.4000);

    % polyline5377
    \path[draw=c666666,line join=round,line cap=round,thin]
      (112.3000,277.7000) -- (111.2000,274.2000) -- (109.1000,276.8000) --
      (115.9000,283.3000) -- (115.7000,279.8000) -- (112.3000,277.7000);

    % polyline5379
    \path[draw=c666666,line join=round,line cap=round,thin]
      (148.7000,348.9000) -- (155.9000,361.4000) -- (157.4000,357.8000) --
      (156.3000,349.5000) -- (150.6000,345.8000) -- (148.7000,348.9000);

    % polyline5381
    \path[draw=c666666,line join=round,line cap=round,thin]
      (28.9000,206.1000) -- (30.8000,205.9000) -- (29.2000,202.7000) --
      (28.9000,206.1000);

    % polyline5383
    \path[draw=c666666,line join=round,line cap=round,thin]
      (143.3000,329.7000) -- (144.6000,332.8000) -- (145.3000,332.4000) --
      (146.1000,332.3000) -- (143.5000,329.9000) -- (143.3000,329.7000);

    % polyline5385
    \path[draw=c666666,line join=round,line cap=round,thin]
      (148.4000,331.5000) -- (146.2000,332.1000) -- (146.1000,332.2000) --
      (146.2000,332.2000) -- (151.2000,336.7000) -- (154.5000,335.3000) --
      (148.4000,331.5000);

    % polyline5387
    \path[draw=c666666,line join=round,line cap=round,thin]
      (28.9000,206.1000) -- (27.7000,206.3000) -- (28.6000,209.6000) --
      (28.6000,209.6000);

    % polyline5389
    \path[draw=c666666,line join=round,line cap=round,thin]
      (335.1000,0.6000) -- (316.9000,5.1000) -- (301.5000,10.0000) --
      (295.0000,15.6000) -- (294.9000,35.8000) -- (298.2000,39.7000) --
      (295.1000,37.7000) -- (294.4000,44.4000) -- (294.9000,48.6000) --
      (297.4000,50.5000) -- (293.3000,52.1000) -- (292.9000,55.4000) --
      (296.3000,61.6000) -- (293.1000,60.0000) -- (285.8000,69.9000) --
      (285.6000,70.3000) -- (273.4000,78.8000) -- (248.6000,84.9000) --
      (232.7000,93.6000) -- (226.2000,107.7000) -- (228.1000,110.2000) --
      (238.0000,112.5000) -- (238.2000,112.6000) -- (236.4000,113.2000) --
      (236.3000,113.3000) -- (228.6000,115.0000) -- (223.1000,119.8000) --
      (212.3000,122.9000) -- (202.8000,118.9000) -- (188.8000,117.4000) --
      (178.1000,113.5000) -- (172.1000,116.2000) -- (164.9000,104.4000) --
      (167.6000,95.5000) -- (166.5000,92.4000) -- (159.2000,91.7000) --
      (156.4000,93.7000) -- (149.0000,94.2000) -- (136.3000,89.0000) --
      (136.1000,92.1000) -- (139.7000,94.2000) -- (140.3000,99.7000) --
      (138.2000,101.9000) -- (141.0000,113.4000) -- (150.6000,123.7000) --
      (149.1000,135.0000) -- (151.3000,138.1000) -- (150.7000,144.9000) --
      (148.2000,151.3000) -- (152.9000,161.9000) -- (155.8000,163.2000) --
      (158.6000,161.4000) -- (157.5000,164.8000) -- (149.5000,165.6000) --
      (136.6000,165.7000) -- (136.9000,159.4000) -- (133.3000,160.2000) --
      (128.8000,164.5000) -- (131.7000,170.8000) -- (129.1000,168.7000) --
      (127.1000,163.7000) -- (123.9000,163.7000) -- (123.9000,165.8000) --
      (119.3000,166.2000) -- (118.2000,162.7000) -- (114.7000,164.3000) --
      (113.6000,161.4000) -- (109.5000,161.7000) -- (97.5000,171.7000) --
      (96.5000,167.6000) -- (91.9000,164.6000) -- (91.9000,161.1000) --
      (86.8000,152.3000) -- (83.5000,151.6000) -- (84.2000,148.5000) --
      (80.8000,148.9000) -- (78.9000,152.2000) -- (80.4000,144.8000) --
      (75.1000,148.4000) -- (74.5000,144.4000) -- (64.5000,148.5000) --
      (62.9000,145.7000) -- (57.8000,149.9000) -- (57.5000,156.9000) --
      (54.0000,157.4000) -- (53.8000,157.4000) -- (48.4000,154.0000) --
      (45.2000,154.7000) -- (44.5000,158.5000) -- (42.0000,155.5000) --
      (40.2000,159.1000) -- (40.1000,151.9000) -- (23.9000,156.6000) --
      (22.2000,153.6000) -- (10.8000,159.9000) -- (13.7000,161.4000) --
      (4.3000,160.4000) -- (2.1000,163.1000) -- (0.6000,173.2000) --
      (2.6000,175.8000) -- (5.6000,174.5000) -- (8.6000,175.9000) --
      (22.4000,171.7000) -- (18.2000,173.7000) -- (17.6000,177.5000) --
      (24.3000,175.9000) -- (23.8000,179.2000) -- (27.1000,180.6000) --
      (23.6000,181.3000) -- (30.6000,184.3000) -- (23.9000,182.7000) --
      (22.4000,179.9000) -- (18.2000,181.1000) -- (11.7000,179.8000) --
      (9.8000,177.1000) -- (8.1000,181.2000) -- (10.8000,187.0000) --
      (12.8000,184.0000) -- (15.9000,184.2000) -- (22.0000,187.8000) --
      (23.0000,191.4000) -- (21.2000,194.2000) -- (18.4000,192.7000) --
      (4.3000,194.1000) -- (2.6000,196.9000) -- (13.3000,200.9000) --
      (17.5000,207.9000) -- (16.2000,214.0000) -- (24.5000,214.7000) --
      (28.6000,210.1000) -- (28.6000,209.6000);

    % polyline5391
    \path[draw=c666666,line join=round,line cap=round,thin]
      (28.6000,209.6000) -- (31.9000,211.8000) -- (37.1000,210.0000) --
      (42.4000,216.6000) -- (45.5000,215.9000) -- (45.5000,212.7000) --
      (46.0000,215.9000) -- (48.9000,214.9000) -- (48.2000,218.0000) --
      (51.4000,218.8000) -- (55.6000,217.1000) -- (55.9000,213.6000) --
      (56.7000,222.3000) -- (59.5000,224.0000) -- (62.5000,222.5000) --
      (62.0000,215.9000) -- (62.7000,219.3000) -- (66.0000,220.0000) --
      (63.6000,222.4000) -- (65.3000,225.3000) -- (69.4000,227.1000) --
      (72.4000,225.7000) -- (69.7000,227.1000) -- (70.0000,230.3000) --
      (72.3000,233.4000) -- (71.5000,237.6000) -- (73.9000,240.2000) --
      (72.5000,237.0000) -- (75.6000,234.3000) -- (79.2000,233.4000) --
      (81.9000,235.5000) -- (80.5000,228.9000) -- (84.0000,234.3000) --
      (85.7000,231.5000) -- (89.8000,232.1000) -- (92.2000,234.9000) --
      (89.9000,237.6000) -- (82.9000,236.7000) -- (88.3000,241.0000) --
      (102.2000,239.8000) -- (105.3000,241.6000) -- (101.9000,241.8000) --
      (103.7000,244.7000) -- (98.7000,249.5000) -- (101.3000,256.5000) --
      (108.1000,257.6000) -- (110.4000,259.9000) -- (116.0000,255.8000) --
      (124.1000,255.1000) -- (116.5000,258.0000) -- (115.8000,265.4000) --
      (112.4000,266.9000) -- (119.0000,269.3000) -- (124.4000,274.9000) --
      (115.7000,284.2000) -- (116.1000,289.2000) -- (127.8000,302.9000) --
      (130.2000,310.8000) -- (139.3000,318.3000) -- (144.7000,319.6000) --
      (146.5000,323.5000) -- (153.1000,324.8000) -- (158.4000,329.1000) --
      (159.2000,326.0000) -- (162.4000,326.5000) -- (157.3000,337.3000) --
      (160.6000,338.4000) -- (165.2000,347.3000) -- (162.1000,347.9000) --
      (163.9000,353.9000) -- (160.6000,360.3000) -- (155.2000,364.0000) --
      (156.7000,368.8000) -- (165.5000,374.3000) -- (177.2000,386.9000) --
      (179.5000,395.2000) -- (179.5000,395.3000) -- (180.2000,398.4000) --
      (177.8000,402.1000) -- (175.0000,393.4000) -- (162.9000,381.6000) --
      (161.9000,378.4000) -- (158.6000,384.9000) -- (151.6000,441.2000) --
      (156.1000,432.4000) -- (161.1000,438.2000) -- (161.4000,441.4000) --
      (154.2000,440.4000) -- (150.6000,448.2000) -- (150.7000,453.5000) --
      (138.3000,509.7000) -- (134.1000,517.6000) -- (134.0000,517.7000) --
      (128.8000,524.7000) -- (121.8000,529.9000);

    % line5393
    \path[draw=c666666,line join=round,line cap=round,thin]
      (28.6000,209.6000) -- (28.9000,206.1000);

    % polyline5395
    \path[draw=c666666,line join=round,line cap=round,thin]
      (371.3000,597.9000) -- (364.8000,589.9000) -- (364.3000,570.1000) --
      (364.8000,559.1000) -- (371.2000,547.1000) -- (374.2000,544.0000) --
      (380.3000,539.5000) -- (387.9000,539.5000) -- (395.3000,531.3000) --
      (407.6000,522.4000) -- (417.1000,519.7000) -- (417.2000,519.9000) --
      (420.4000,525.3000) -- (423.7000,526.1000) -- (424.0000,526.1000) --
      (440.0000,526.6000) -- (442.3000,529.3000) -- (441.7000,532.4000) --
      (445.8000,533.4000) -- (456.7000,532.2000) -- (454.3000,529.6000) --
      (455.7000,526.8000) -- (458.8000,527.0000) -- (463.9000,533.2000) --
      (478.8000,531.5000) -- (483.2000,541.3000) -- (494.1000,544.0000) --
      (497.2000,543.1000) -- (497.3000,543.2000) -- (504.2000,547.2000) --
      (503.9000,550.4000) -- (513.9000,547.5000) -- (520.5000,549.3000) --
      (521.8000,552.5000) -- (524.1000,546.2000) -- (532.4000,547.7000) --
      (532.5000,544.4000) -- (538.9000,543.0000) -- (541.9000,540.2000) --
      (546.1000,540.7000) -- (548.6000,534.6000) -- (544.6000,532.7000) --
      (551.6000,523.6000) -- (557.6000,522.8000) -- (562.5000,514.2000) --
      (569.0000,511.1000) -- (570.1000,506.5000) -- (586.4000,497.6000);


    % polyline5401
    \path[draw=c666666,line join=round,line cap=round,thin]
      (257.4000,199.1000) -- (257.4000,199.1000) -- (256.2000,190.6000) --
      (264.6000,184.9000) -- (264.1000,179.7000) -- (264.2000,174.6000) --
      (258.9000,169.8000) -- (258.8000,166.1000);

    % polyline5403
    \path[draw=c666666,line join=round,line cap=round,thin]
      (273.1000,324.6000) -- (274.8000,321.5000) -- (271.9000,320.1000) --
      (269.0000,313.9000) -- (265.7000,313.9000) -- (260.6000,309.1000) --
      (261.4000,302.5000) -- (251.1000,287.9000) -- (250.8000,284.4000) --
      (244.8000,281.0000) -- (246.4000,284.3000) -- (235.2000,284.8000) --
      (232.0000,283.2000) -- (230.6000,276.3000) -- (227.3000,276.6000) --
      (227.0000,273.0000) -- (220.9000,269.4000) -- (220.8000,269.4000) --
      (227.3000,252.2000) -- (229.2000,241.8000) -- (230.0000,239.0000) --
      (236.9000,241.9000) -- (236.3000,238.3000) -- (239.5000,238.6000) --
      (248.1000,233.6000) -- (248.1000,233.5000) -- (248.8000,229.6000) --
      (252.4000,228.4000) -- (259.4000,216.5000) -- (259.0000,210.3000) --
      (257.1000,207.7000) -- (259.3000,205.4000) -- (259.8000,203.3000) --
      (257.4000,199.1000);

    % polyline5405
    \path[draw=c666666,line join=round,line cap=round,thin]
      (257.4000,199.1000) -- (257.4000,199.1000) -- (253.8000,198.3000) --
      (250.3000,194.3000) -- (247.1000,195.5000) -- (238.9000,189.9000) --
      (237.2000,180.4000) -- (234.1000,178.7000) -- (221.7000,185.3000) --
      (218.3000,185.2000) -- (218.4000,185.1000) -- (218.4000,181.8000) --
      (214.7000,180.1000) -- (212.5000,173.4000) -- (209.5000,173.1000) --
      (208.7000,176.2000) -- (198.5000,176.2000) -- (190.9000,179.8000) --
      (188.1000,177.9000) -- (185.4000,179.7000) -- (181.7000,175.6000) --
      (181.6000,175.6000) -- (172.0000,174.6000);

    % polyline5407
    \path[draw=c666666,line join=round,line cap=round,thin]
      (381.0000,153.9000) -- (370.6000,147.1000) -- (366.2000,142.5000) --
      (366.9000,139.3000) -- (364.9000,136.5000) -- (361.9000,135.7000) --
      (352.9000,139.0000) -- (350.0000,137.2000) -- (347.8000,139.6000) --
      (340.0000,138.5000) -- (327.9000,131.3000) -- (325.3000,133.1000) --
      (315.0000,130.6000) -- (307.9000,132.7000) -- (301.6000,131.5000) --
      (299.9000,127.9000);

    % polyline5409
    \path[draw=c666666,line join=round,line cap=round,thin]
      (258.8000,166.1000) -- (261.4000,163.2000) -- (268.2000,162.0000) --
      (274.1000,158.4000) -- (277.3000,160.2000) -- (284.1000,158.7000) --
      (283.8000,155.6000) -- (290.5000,147.9000);

    % polyline5411
    \path[draw=c666666,line join=round,line cap=round,thin]
      (356.3000,201.9000) -- (343.8000,204.9000) -- (333.5000,204.1000) --
      (336.4000,202.4000) -- (337.0000,199.3000) -- (332.3000,194.5000) --
      (331.6000,191.1000) -- (327.4000,190.5000) -- (324.3000,192.4000) --
      (322.6000,189.4000) -- (320.4000,192.6000) -- (312.7000,193.4000) --
      (312.6000,193.4000) -- (309.5000,181.5000) -- (306.0000,182.3000) --
      (303.5000,179.8000) -- (302.1000,173.5000) -- (295.2000,167.5000) --
      (294.6000,155.3000) -- (290.5000,147.9000);

    % polyline5413
    \path[draw=c666666,line join=round,line cap=round,thin]
      (285.8000,69.9000) -- (285.9000,70.0000) -- (299.9000,82.0000) --
      (303.8000,91.4000) -- (300.4000,96.7000) -- (301.9000,106.5000) --
      (300.4000,115.5000) -- (304.3000,124.9000) -- (299.9000,127.9000);

    % polyline5415
    \path[draw=c666666,line join=round,line cap=round,thin]
      (356.3000,201.9000) -- (360.7000,195.8000) -- (361.0000,189.3000) --
      (378.2000,185.9000);

    % polyline5417
    \path[draw=c666666,line join=round,line cap=round,thin]
      (361.6000,296.4000) -- (363.1000,286.5000) -- (363.4000,279.9000) --
      (360.8000,277.7000) -- (360.5000,271.0000) -- (357.0000,262.5000) --
      (354.0000,260.1000) -- (355.9000,253.7000) -- (353.7000,246.4000) --
      (355.4000,243.7000) -- (358.5000,242.9000) -- (356.0000,235.1000) --
      (352.7000,232.0000) -- (358.6000,228.2000) -- (360.4000,225.6000) --
      (359.1000,222.3000) -- (364.3000,216.8000) -- (363.9000,210.3000) --
      (356.3000,201.9000);

    % polyline5419
    \path[draw=c666666,line join=round,line cap=round,thin]
      (238.0000,112.5000) -- (236.4000,113.1000) -- (236.4000,113.2000) --
      (236.8000,123.1000) -- (240.0000,123.1000) -- (237.5000,125.4000) --
      (240.5000,135.9000) -- (238.9000,139.3000) -- (241.6000,143.9000) --
      (240.8000,146.4000) -- (241.0000,150.0000) -- (248.1000,150.6000) --
      (256.9000,161.2000) -- (254.9000,163.8000) -- (258.8000,166.1000);

    % polyline5421
    \path[draw=c666666,line join=round,line cap=round,thin]
      (172.0000,174.6000) -- (171.9000,188.3000) -- (170.1000,190.9000) --
      (172.3000,210.4000) -- (166.1000,212.9000) -- (162.1000,222.6000) --
      (161.5000,224.9000) -- (153.9000,221.0000) -- (143.8000,225.4000) --
      (142.0000,228.5000) -- (127.2000,229.0000) -- (121.4000,232.8000) --
      (120.9000,239.6000) -- (103.7000,244.7000);

    % polyline5423
    \path[draw=c666666,line join=round,line cap=round,thin]
      (149.5000,165.6000) -- (154.9000,176.4000) -- (157.9000,177.4000) --
      (165.8000,172.1000) -- (172.0000,174.6000);

    % polyline5425
    \path[draw=c666666,line join=round,line cap=round,thin]
      (220.9000,269.4000) -- (213.4000,276.2000) -- (210.9000,273.5000) --
      (200.0000,273.8000) -- (193.7000,274.7000) -- (189.6000,280.0000) --
      (175.8000,280.9000) -- (178.3000,287.3000) -- (183.0000,291.9000) --
      (188.4000,312.1000) -- (188.1000,320.4000) -- (190.9000,322.6000) --
      (180.7000,327.6000) -- (178.0000,325.7000) -- (170.8000,326.0000) --
      (172.3000,322.7000) -- (162.4000,326.5000);

    % polyline5427
    \path[draw=c666666,line join=round,line cap=round,thin]
      (381.0000,153.9000) -- (389.6000,142.4000) -- (386.8000,141.0000) --
      (387.8000,138.0000) -- (385.9000,135.2000) -- (392.0000,132.7000) --
      (388.4000,123.7000) -- (389.8000,120.7000) -- (398.8000,116.2000) --
      (404.8000,118.3000) -- (405.9000,115.1000) -- (405.5000,99.1000) --
      (408.6000,98.7000) -- (414.3000,90.6000) -- (412.7000,87.2000) --
      (414.3000,83.3000) -- (413.3000,76.9000);

    % polyline5429
    \path[draw=c666666,line join=round,line cap=round,thin]
      (409.0000,75.3000) -- (409.0000,75.2000) -- (391.4000,70.5000) --
      (385.6000,72.5000) -- (382.9000,70.6000) -- (380.0000,73.1000) --
      (365.9000,73.7000) -- (362.2000,70.9000) -- (352.6000,72.9000) --
      (351.6000,69.2000) -- (349.0000,71.2000) -- (345.8000,66.0000) --
      (344.9000,69.1000) -- (335.5000,65.2000) -- (332.7000,68.7000) --
      (330.8000,65.1000) -- (333.7000,58.7000) -- (320.3000,61.5000) --
      (312.2000,55.1000) -- (311.2000,51.7000) -- (297.4000,50.5000);

    % polyline5431
    \path[draw=c666666,line join=round,line cap=round,thin]
      (290.5000,147.9000) -- (288.6000,141.4000) -- (291.1000,138.4000) --
      (295.4000,138.5000) -- (299.9000,127.9000);

    % polyline5433
    \path[draw=c666666,line join=round,line cap=round,thin]
      (459.1000,458.0000) -- (464.2000,461.1000) -- (461.5000,465.1000) --
      (457.0000,466.4000) -- (454.7000,463.9000) -- (456.1000,460.5000) --
      (459.1000,458.0000);

    % polyline5435
    \path[draw=c666666,line join=round,line cap=round,thin]
      (424.0000,526.1000) -- (428.3000,519.4000) -- (434.8000,517.4000) --
      (433.6000,514.4000) -- (438.4000,509.0000) -- (442.0000,510.0000) --
      (444.2000,498.1000) -- (453.7000,487.9000) -- (446.6000,480.7000) --
      (446.4000,474.5000) -- (443.4000,469.5000);

    % polyline5437
    \path[draw=c666666,line join=round,line cap=round,thin]
      (443.4000,469.5000) -- (436.7000,465.7000) -- (433.6000,465.7000) --
      (430.7000,468.2000) -- (430.6000,464.6000) -- (422.7000,469.6000) --
      (413.9000,463.9000) -- (406.1000,447.9000) -- (403.8000,437.8000);

    % polyline5439
    \path[draw=c666666,line join=round,line cap=round,thin]
      (403.5000,331.7000) -- (399.1000,333.1000) -- (398.5000,336.5000) --
      (400.2000,350.5000) -- (394.1000,355.8000) -- (396.9000,358.9000) --
      (394.5000,366.0000) -- (399.7000,374.6000) -- (404.9000,378.8000) --
      (407.9000,387.3000) -- (404.6000,395.3000) -- (407.7000,394.3000) --
      (410.2000,396.6000) -- (413.4000,396.2000) -- (420.4000,392.8000) --
      (427.1000,395.3000) -- (427.9000,401.5000) -- (430.9000,400.4000) --
      (433.4000,403.1000) -- (432.3000,407.7000) -- (432.3000,411.1000) --
      (429.4000,409.8000) -- (427.8000,415.7000) -- (424.8000,418.2000) --
      (425.6000,421.4000) -- (421.7000,421.8000) -- (417.9000,428.5000) --
      (411.3000,430.5000) -- (403.8000,437.8000);

    % polyline5441
    \path[draw=c666666,line join=round,line cap=round,thin]
      (284.9000,419.2000) -- (281.1000,411.3000) -- (277.4000,410.6000) --
      (277.2000,407.2000) -- (274.6000,405.5000) -- (276.9000,403.4000) --
      (274.9000,400.1000) -- (278.9000,395.0000) -- (275.8000,389.6000) --
      (269.3000,387.2000) -- (271.5000,384.2000) -- (263.5000,377.9000) --
      (252.9000,379.1000) -- (252.9000,373.8000) -- (245.9000,370.4000) --
      (245.9000,370.5000) -- (239.6000,377.3000) -- (237.2000,383.3000) --
      (228.4000,389.9000) -- (227.5000,396.3000) -- (222.9000,402.8000) --
      (216.8000,403.6000) -- (214.5000,405.8000) -- (214.4000,406.3000) --
      (212.1000,411.9000) -- (201.6000,410.8000) -- (194.7000,405.9000) --
      (194.1000,399.5000) -- (186.9000,397.6000) -- (186.1000,394.3000) --
      (179.5000,395.3000);

    % polyline5443
    \path[draw=c666666,line join=round,line cap=round,thin]
      (538.4000,407.6000) -- (538.3000,407.6000) -- (531.5000,408.9000) --
      (529.3000,411.1000) -- (520.5000,407.2000) -- (518.8000,411.6000) --
      (520.8000,416.1000) -- (525.7000,419.6000) -- (525.3000,426.2000) --
      (521.8000,425.0000) -- (507.7000,429.3000) -- (506.6000,432.4000) --
      (500.5000,433.8000) -- (499.4000,437.3000) -- (499.0000,441.0000) --
      (492.5000,441.4000) -- (489.7000,449.0000) -- (493.1000,451.5000) --
      (490.3000,454.4000) -- (485.6000,452.8000) -- (482.0000,458.4000) --
      (483.7000,461.5000) -- (494.9000,467.8000) -- (494.7000,473.1000) --
      (493.6000,473.4000) -- (490.9000,473.5000) -- (486.1000,478.8000) --
      (480.3000,476.3000) -- (480.2000,473.0000) -- (469.7000,472.1000) --
      (469.1000,465.7000) -- (466.3000,468.1000) -- (463.6000,466.5000) --
      (453.8000,470.9000) -- (449.9000,466.1000) -- (443.3000,465.3000) --
      (443.4000,469.5000);

    % path5445
    \path[draw=c666666,line join=round,line cap=round,thin]
      (315.1000,422.3000) -- (315.1000,422.3000) -- (318.8000,420.2000) --
      (317.1000,416.4000) -- (320.6000,415.2000) -- (320.9000,411.3000) --
      (323.5000,409.1000) -- (321.4000,406.6000) -- (330.0000,396.0000) --
      (330.7000,392.1000) -- (337.7000,394.3000) -- (335.5000,378.7000) --
      (337.4000,375.9000) -- (336.1000,369.7000) -- (331.1000,363.5000) --
      (341.5000,353.9000) -- (339.0000,346.3000) -- (339.5000,341.7000) --
      (335.5000,332.3000) .. controls (332.7000,330.5000) and (330.0000,328.7000) ..
      (327.2000,326.8000) -- (325.9000,322.5000) -- (325.9000,322.4000) --
      (329.8000,315.8000) -- (339.7000,314.7000) -- (341.6000,312.0000) --
      (339.7000,306.0000) -- (342.2000,303.9000) -- (351.7000,300.6000) --
      (357.0000,296.4000) -- (361.6000,296.4000);

    % polyline5447
    \path[draw=c666666,line join=round,line cap=round,thin]
      (361.6000,296.4000) -- (368.1000,303.6000) -- (375.2000,303.7000) --
      (378.4000,301.8000) -- (383.3000,306.0000) -- (390.0000,299.2000) --
      (397.5000,314.4000) -- (406.3000,317.7000) -- (408.2000,320.4000) --
      (407.6000,326.5000) -- (403.5000,331.7000);

    % polyline5449
    \path[draw=c666666,line join=round,line cap=round,thin]
      (273.1000,324.6000) -- (275.9000,325.9000) -- (282.4000,324.7000) --
      (284.4000,327.4000) -- (287.5000,324.1000) -- (300.4000,323.0000) --
      (302.6000,320.2000) -- (317.4000,323.2000) -- (320.4000,322.3000) --
      (325.9000,322.5000);

    % polyline5451
    \path[draw=c666666,line join=round,line cap=round,thin]
      (273.1000,324.6000) -- (272.1000,324.3000) -- (270.7000,327.0000) --
      (266.9000,326.0000) -- (265.6000,329.4000) -- (259.2000,332.4000) --
      (255.8000,338.1000) -- (255.6000,347.7000) -- (260.9000,351.7000) --
      (260.2000,355.3000) -- (257.3000,356.5000) -- (252.8000,365.5000) --
      (249.7000,366.1000) -- (245.9000,370.4000);

    % polyline5453
    \path[draw=c666666,line join=round,line cap=round,thin]
      (506.6000,318.7000) -- (501.3000,326.1000) -- (495.2000,329.7000) --
      (490.8000,329.9000) -- (490.3000,326.8000) -- (486.6000,325.0000) --
      (482.1000,329.9000) -- (478.8000,330.1000) -- (478.7000,326.7000) --
      (475.4000,326.7000) -- (470.6000,318.0000);

    % polyline5455
    \path[draw=c666666,line join=round,line cap=round,thin]
      (470.6000,318.0000) -- (465.2000,314.3000) -- (459.1000,315.7000) --
      (452.5000,314.1000) -- (445.9000,337.5000) -- (439.5000,328.9000) --
      (437.7000,331.5000) -- (434.9000,329.8000) -- (432.4000,331.7000) --
      (430.0000,329.6000) -- (427.3000,331.8000) -- (427.0000,335.1000) --
      (420.6000,339.5000) -- (418.2000,337.1000) -- (409.3000,338.7000) --
      (403.5000,335.1000) -- (403.5000,331.7000);

    % polyline5457
    \path[draw=c666666,line join=round,line cap=round,thin]
      (284.9000,419.2000) -- (284.2000,424.8000) -- (282.6000,430.8000) --
      (277.5000,434.4000) -- (278.3000,438.4000) -- (271.4000,442.3000) --
      (266.2000,449.5000) -- (262.4000,453.0000) -- (265.4000,463.3000) --
      (256.9000,465.5000) -- (256.9000,468.8000) -- (259.6000,470.3000) --
      (254.7000,476.5000) -- (256.2000,479.6000) -- (253.0000,479.3000) --
      (249.0000,484.6000) -- (246.0000,486.1000) -- (236.3000,484.9000) --
      (230.7000,488.2000) -- (221.4000,487.4000) -- (219.8000,490.2000) --
      (215.6000,489.5000) -- (215.6000,489.5000) -- (215.3000,492.8000) --
      (212.4000,494.2000) -- (211.7000,490.9000) -- (201.2000,493.9000) --
      (201.8000,501.1000) -- (197.0000,513.5000) -- (198.8000,516.6000) --
      (206.2000,516.9000) -- (206.2000,517.0000) -- (209.0000,523.5000) --
      (207.4000,526.6000) -- (210.6000,526.6000) -- (211.1000,534.7000) --
      (208.1000,536.3000) -- (209.0000,539.8000) -- (200.0000,550.1000) --
      (200.4000,553.5000) -- (196.7000,554.4000) -- (194.9000,557.5000) --
      (193.5000,567.2000) -- (193.5000,567.3000);

    % polyline5459
    \path[draw=c666666,line join=round,line cap=round,thin]
      (315.1000,422.3000) -- (307.5000,422.2000) -- (301.6000,425.1000) --
      (294.0000,418.3000) -- (284.9000,419.2000);

    % polyline5461
    \path[draw=c666666,line join=round,line cap=round,thin]
      (360.4000,445.2000) -- (357.5000,439.4000) -- (358.0000,436.0000) --
      (355.4000,434.3000) -- (354.0000,429.5000) -- (348.1000,425.1000) --
      (342.8000,429.8000) -- (339.4000,439.9000) -- (335.1000,445.4000) --
      (328.0000,443.7000) -- (322.1000,447.3000) -- (318.5000,441.7000) --
      (320.4000,435.2000) -- (316.2000,428.5000) -- (315.1000,422.3000);

    % polyline5463
    \path[draw=c666666,line join=round,line cap=round,thin]
      (403.8000,437.8000) -- (393.6000,429.1000) -- (390.5000,428.8000) --
      (390.3000,432.1000) -- (383.1000,430.7000) -- (381.4000,424.3000) --
      (378.9000,422.4000) -- (377.7000,423.5000) -- (370.1000,429.9000) --
      (368.1000,427.3000) -- (360.4000,445.2000);

    % path5465
    \path[draw=c666666,line join=round,line cap=round,thin]
      (300.1000,588.5000) -- (310.0000,585.5000) -- (311.5000,582.6000) --
      (319.8000,582.4000) -- (314.3000,576.1000) -- (310.7000,577.5000) --
      (306.9000,575.8000) -- (305.9000,569.4000) -- (309.6000,568.7000) --
      (310.9000,565.3000) -- (308.2000,562.5000) -- (310.6000,560.0000) --
      (310.4000,556.5000) -- (308.0000,550.6000) -- (298.2000,546.3000) --
      (295.7000,540.1000) -- (305.2000,528.9000) -- (306.7000,531.8000) --
      (312.6000,529.9000) -- (313.1000,529.0000) -- (315.1000,531.6000) --
      (322.2000,532.6000) -- (323.4000,528.9000) -- (326.7000,528.3000) --
      (340.1000,529.9000) -- (343.9000,527.6000) -- (344.9000,524.2000) --
      (344.1000,513.4000) -- (350.9000,515.8000) .. controls (353.5000,514.1000) and
      (356.0000,512.5000) .. (358.5000,510.9000) -- (364.9000,511.0000) --
      (365.1000,501.2000) -- (372.1000,502.6000) -- (375.4000,496.7000) --
      (379.7000,495.5000) -- (379.3000,493.9000) -- (384.2000,488.0000) --
      (374.7000,483.4000) -- (377.8000,482.1000) -- (380.1000,477.7000) --
      (369.5000,472.7000) -- (367.8000,470.0000) -- (368.4000,460.2000) --
      (365.2000,450.3000) -- (360.4000,445.3000) -- (360.4000,445.2000);

    % polyline5467
    \path[draw=c666666,line join=round,line cap=round,thin]
      (470.6000,318.0000) -- (473.7000,316.8000) -- (472.8000,310.8000) --
      (476.0000,305.2000) -- (473.5000,295.5000) -- (470.7000,293.7000) --
      (476.3000,289.5000) -- (467.1000,284.6000) -- (466.6000,281.2000) --
      (467.8000,276.6000) -- (474.8000,269.1000) -- (476.8000,259.8000) --
      (478.1000,258.3000) -- (473.7000,249.2000) -- (470.8000,247.5000) --
      (476.1000,243.0000) -- (475.9000,239.4000) -- (473.6000,236.0000) --
      (470.5000,237.8000);

    % polyline5469
    \path[draw=c666666,line join=round,line cap=round,thin]
      (553.0000,240.9000) -- (550.6000,235.0000) -- (547.5000,235.2000) --
      (547.0000,225.1000) -- (538.4000,219.4000);

    % polyline5471
    \path[draw=c666666,line join=round,line cap=round,thin]
      (205.9000,536.5000) -- (206.0000,539.5000) -- (203.5000,537.7000) --
      (205.9000,536.5000);

    % polyline5473
    \path[draw=c666666,line join=round,line cap=round,thin]
      (548.4000,171.0000) -- (547.0000,182.3000) -- (552.8000,184.8000) --
      (551.6000,185.8000) -- (546.9000,201.1000) -- (541.6000,209.8000) --
      (541.5000,216.8000) -- (538.4000,219.4000);

    % polyline5475
    \path[draw=c666666,line join=round,line cap=round,thin]
      (569.3000,132.4000) -- (569.3000,132.5000) -- (567.8000,137.2000) --
      (566.2000,140.1000) -- (556.1000,140.4000) -- (546.0000,135.2000) --
      (545.0000,132.1000) -- (539.5000,144.2000) -- (545.6000,147.0000) --
      (543.8000,149.8000) -- (546.4000,151.6000) -- (548.0000,148.8000) --
      (551.1000,149.2000) -- (556.1000,153.0000) -- (552.7000,168.0000) --
      (550.2000,169.9000) -- (546.3000,169.6000) -- (548.4000,171.0000);

    % polyline5477
    \path[draw=c666666,line join=round,line cap=round,thin]
      (538.4000,219.4000) -- (537.4000,220.2000) -- (528.0000,212.7000) --
      (522.3000,215.2000) -- (516.8000,211.1000) -- (510.4000,213.0000) --
      (502.5000,208.2000) -- (493.4000,214.6000);

    % polyline5479
    \path[draw=c666666,line join=round,line cap=round,thin]
      (465.7000,98.9000) -- (463.7000,100.2000) -- (461.4000,103.2000) --
      (455.0000,100.9000) -- (452.8000,103.2000) -- (453.8000,113.0000) --
      (450.2000,118.7000) -- (451.7000,121.5000) -- (447.0000,125.6000) --
      (447.0000,125.7000) -- (449.5000,135.9000) -- (448.4000,140.4000) --
      (451.3000,141.9000) -- (450.6000,146.4000) -- (447.2000,147.5000) --
      (445.9000,154.1000) -- (449.5000,159.2000) -- (450.0000,163.8000) --
      (453.8000,169.6000) -- (472.8000,181.2000) -- (470.1000,183.9000) --
      (484.5000,194.6000) -- (482.5000,204.2000) -- (488.7000,209.4000) --
      (488.9000,213.0000) -- (491.9000,211.8000) -- (493.4000,214.6000) --
      (493.4000,214.6000) -- (493.6000,216.3000) -- (491.1000,219.9000) --
      (487.9000,219.4000) -- (485.5000,222.2000) -- (485.2000,231.5000) --
      (481.7000,232.7000) -- (479.2000,230.7000) -- (471.9000,235.0000) --
      (470.5000,237.8000) -- (465.1000,240.0000) -- (465.1000,236.8000) --
      (461.8000,235.0000) -- (451.4000,232.8000) -- (448.8000,227.4000) --
      (452.0000,224.7000) -- (450.8000,221.2000) -- (448.8000,218.7000) --
      (445.5000,218.4000) -- (445.9000,215.2000) -- (443.0000,214.0000) --
      (442.4000,211.0000) -- (438.4000,210.2000) -- (438.4000,210.2000) --
      (434.1000,209.6000) -- (428.9000,214.0000) -- (422.0000,215.1000) --
      (419.5000,217.3000) -- (416.2000,215.7000) -- (402.1000,217.5000) --
      (396.4000,205.2000) -- (392.4000,200.6000) -- (390.2000,203.1000) --
      (387.3000,193.8000) -- (382.3000,187.5000) -- (378.2000,185.9000);

    % polyline5481
    \path[draw=c666666,line join=round,line cap=round,thin]
      (378.2000,185.9000) -- (377.8000,176.3000) -- (381.1000,175.4000) --
      (384.4000,169.9000) -- (381.5000,168.0000) -- (377.6000,158.6000) --
      (380.5000,156.1000) -- (381.0000,153.9000);

    \begin{scope}% g5483
      % path5485
      \path[fill=c666666] (525.8000,548.3000) .. controls (525.8000,548.3000) and
        (525.8000,548.3000) .. (525.7000,548.3000) .. controls (525.6000,548.3000) and
        (525.6000,548.3000) .. (525.6000,548.3000) .. controls (525.6000,548.3000) and
        (525.6000,548.3000) .. (525.5000,548.4000) .. controls (525.5000,548.4000) and
        (525.5000,548.4000) .. (525.4000,548.4000) -- (525.4000,548.5000) .. controls
        (525.4000,548.5000) and (525.3000,548.6000) .. (525.3000,548.7000) .. controls
        (525.3000,548.7000) and (525.3000,548.7000) .. (525.3000,548.8000) .. controls
        (525.3000,548.8000) and (525.3000,548.9000) .. (525.3000,549.0000) .. controls
        (525.4000,549.0000) and (525.4000,549.1000) .. (525.4000,549.1000) .. controls
        (525.5000,549.2000) and (525.5000,549.2000) .. (525.6000,549.2000) .. controls
        (525.6000,549.2000) and (525.6000,549.2000) .. (525.7000,549.2000) --
        (525.8000,549.2000) -- (525.8000,549.2000) .. controls (525.9000,549.2000) and
        (525.9000,549.2000) .. (525.9000,549.2000) -- (525.9000,549.2000) --
        (525.9000,549.0000) .. controls (525.9000,549.0000) and (525.9000,549.0000) ..
        (525.8000,549.1000) .. controls (525.8000,549.1000) and (525.8000,549.1000) ..
        (525.7000,549.1000) .. controls (525.6000,549.1000) and (525.6000,549.1000) ..
        (525.6000,549.0000) .. controls (525.6000,549.0000) and (525.6000,549.0000) ..
        (525.5000,549.0000) -- (525.5000,548.9000) -- (525.5000,548.8000) .. controls
        (525.5000,548.7000) and (525.5000,548.7000) .. (525.5000,548.7000) .. controls
        (525.5000,548.6000) and (525.5000,548.5000) .. (525.5000,548.5000) .. controls
        (525.6000,548.5000) and (525.6000,548.5000) .. (525.6000,548.5000) .. controls
        (525.6000,548.4000) and (525.7000,548.4000) .. (525.7000,548.4000) .. controls
        (525.8000,548.4000) and (525.8000,548.4000) .. (525.8000,548.4000) .. controls
        (525.9000,548.5000) and (525.9000,548.5000) .. (525.9000,548.5000) --
        (525.9000,548.4000) .. controls (525.9000,548.3000) and (525.9000,548.3000) ..
        (525.8000,548.3000) -- cycle;

      % path5487
      \path[fill=c666666] (526.6000,548.4000) .. controls (526.6000,548.5000) and
        (526.6000,548.5000) .. (526.6000,548.5000) -- (526.6000,548.5000) .. controls
        (526.6000,548.6000) and (526.6000,548.6000) .. (526.6000,548.7000) --
        (526.6000,549.2000) -- (526.8000,549.2000) -- (526.8000,548.6000) .. controls
        (526.9000,548.5000) and (526.9000,548.5000) .. (526.9000,548.5000) .. controls
        (527.0000,548.4000) and (527.0000,548.4000) .. (527.1000,548.4000) --
        (527.1000,548.4000) .. controls (527.1000,548.5000) and (527.1000,548.5000) ..
        (527.1000,548.5000) .. controls (527.2000,548.5000) and (527.2000,548.5000) ..
        (527.2000,548.5000) .. controls (527.2000,548.6000) and (527.2000,548.6000) ..
        (527.2000,548.7000) -- (527.2000,549.2000) -- (527.4000,549.2000) --
        (527.4000,548.7000) .. controls (527.4000,548.6000) and (527.4000,548.5000) ..
        (527.4000,548.5000) .. controls (527.3000,548.4000) and (527.3000,548.4000) ..
        (527.3000,548.4000) .. controls (527.3000,548.3000) and (527.2000,548.3000) ..
        (527.1000,548.3000) -- (527.1000,548.3000) .. controls (527.0000,548.3000) and
        (527.0000,548.3000) .. (526.9000,548.4000) .. controls (526.9000,548.4000) and
        (526.8000,548.4000) .. (526.8000,548.5000) .. controls (526.8000,548.4000) and
        (526.7000,548.4000) .. (526.7000,548.4000) .. controls (526.6000,548.3000) and
        (526.6000,548.3000) .. (526.6000,548.3000) .. controls (526.5000,548.3000) and
        (526.5000,548.3000) .. (526.5000,548.3000) .. controls (526.4000,548.3000) and
        (526.4000,548.3000) .. (526.4000,548.3000) .. controls (526.4000,548.4000) and
        (526.3000,548.4000) .. (526.3000,548.4000) .. controls (526.3000,548.4000) and
        (526.3000,548.4000) .. (526.3000,548.5000) -- (526.3000,548.5000) --
        (526.3000,548.3000) -- (526.1000,548.3000) -- (526.1000,549.2000) --
        (526.3000,549.2000) -- (526.3000,548.6000) .. controls (526.3000,548.5000) and
        (526.4000,548.5000) .. (526.4000,548.5000) .. controls (526.4000,548.4000) and
        (526.5000,548.4000) .. (526.5000,548.4000) .. controls (526.6000,548.4000) and
        (526.6000,548.4000) .. (526.6000,548.4000) -- cycle;

      % path5489
      \path[fill=c666666] (527.8000,548.3000) -- (527.8000,548.3000) --
        (527.7000,548.3000) .. controls (527.7000,548.3000) and (527.6000,548.3000) ..
        (527.6000,548.4000) -- (527.6000,548.5000) .. controls (527.7000,548.5000) and
        (527.7000,548.5000) .. (527.8000,548.4000) -- (527.9000,548.4000) .. controls
        (527.9000,548.4000) and (528.0000,548.4000) .. (528.0000,548.5000) .. controls
        (528.1000,548.5000) and (528.1000,548.5000) .. (528.1000,548.5000) --
        (528.1000,548.6000) -- (528.1000,548.7000) -- (528.1000,548.7000) --
        (528.1000,548.7000) -- (528.0000,548.7000) -- (527.9000,548.7000) --
        (527.8000,548.7000) .. controls (527.7000,548.7000) and (527.7000,548.7000) ..
        (527.6000,548.7000) .. controls (527.6000,548.8000) and (527.6000,548.8000) ..
        (527.6000,548.8000) -- (527.6000,548.9000) .. controls (527.6000,549.0000) and
        (527.6000,549.0000) .. (527.6000,549.0000) .. controls (527.6000,549.1000) and
        (527.6000,549.2000) .. (527.6000,549.2000) .. controls (527.7000,549.2000) and
        (527.7000,549.2000) .. (527.8000,549.2000) .. controls (527.8000,549.2000) and
        (527.8000,549.2000) .. (527.9000,549.2000) -- (527.9000,549.2000) .. controls
        (528.0000,549.2000) and (528.0000,549.2000) .. (528.0000,549.2000) --
        (528.1000,549.2000) -- (528.1000,549.1000) -- (528.1000,549.2000) --
        (528.3000,549.2000) -- (528.3000,548.7000) .. controls (528.3000,548.6000) and
        (528.3000,548.6000) .. (528.3000,548.5000) -- (528.2000,548.5000) .. controls
        (528.2000,548.4000) and (528.1000,548.4000) .. (528.1000,548.4000) .. controls
        (528.1000,548.3000) and (528.0000,548.3000) .. (527.9000,548.3000) .. controls
        (527.9000,548.3000) and (527.9000,548.3000) .. (527.8000,548.3000) --
        cycle(528.1000,548.8000) -- (528.1000,549.0000) -- (528.1000,549.0000) ..
        controls (528.1000,549.0000) and (528.1000,549.0000) .. (528.0000,549.0000) --
        (528.0000,549.1000) .. controls (527.9000,549.1000) and (527.9000,549.1000) ..
        (527.9000,549.1000) -- (527.8000,549.1000) -- (527.8000,549.0000) .. controls
        (527.8000,549.0000) and (527.8000,549.0000) .. (527.7000,549.0000) --
        (527.7000,548.9000) .. controls (527.7000,548.9000) and (527.7000,548.8000) ..
        (527.8000,548.8000) -- (527.9000,548.8000) .. controls (528.0000,548.8000) and
        (528.0000,548.8000) .. (528.0000,548.8000) .. controls (528.1000,548.8000) and
        (528.1000,548.8000) .. (528.1000,548.8000) -- cycle;

      % path5491
      \path[fill=c666666] (529.2000,548.6000) .. controls (529.2000,548.5000) and
        (529.1000,548.5000) .. (529.1000,548.4000) .. controls (529.1000,548.4000) and
        (529.1000,548.3000) .. (529.0000,548.3000) -- (528.9000,548.3000) --
        (528.8000,548.3000) .. controls (528.8000,548.3000) and (528.7000,548.3000) ..
        (528.7000,548.4000) -- (528.6000,548.4000) -- (528.6000,548.4000) --
        (528.6000,548.4000) -- (528.6000,548.3000) -- (528.5000,548.3000) --
        (528.5000,549.6000) -- (528.6000,549.6000) -- (528.6000,549.1000) .. controls
        (528.6000,549.2000) and (528.7000,549.2000) .. (528.8000,549.2000) .. controls
        (528.8000,549.2000) and (528.8000,549.2000) .. (528.9000,549.2000) .. controls
        (528.9000,549.2000) and (528.9000,549.2000) .. (529.0000,549.2000) --
        (529.0000,549.2000) .. controls (529.1000,549.2000) and (529.1000,549.2000) ..
        (529.1000,549.2000) .. controls (529.1000,549.1000) and (529.1000,549.0000) ..
        (529.1000,549.0000) .. controls (529.2000,549.0000) and (529.2000,548.9000) ..
        (529.2000,548.9000) .. controls (529.3000,548.9000) and (529.3000,548.8000) ..
        (529.3000,548.7000) .. controls (529.3000,548.7000) and (529.3000,548.7000) ..
        (529.2000,548.6000) -- cycle(528.9000,548.4000) .. controls
        (528.9000,548.4000) and (528.9000,548.4000) .. (529.0000,548.5000) --
        (529.0000,548.5000) .. controls (529.1000,548.5000) and (529.1000,548.6000) ..
        (529.1000,548.7000) -- (529.1000,548.7000) .. controls (529.1000,548.8000) and
        (529.1000,548.9000) .. (529.1000,548.9000) .. controls (529.1000,548.9000) and
        (529.1000,549.0000) .. (529.0000,549.0000) .. controls (529.0000,549.0000) and
        (529.0000,549.0000) .. (528.9000,549.1000) -- (528.9000,549.1000) .. controls
        (528.8000,549.1000) and (528.8000,549.1000) .. (528.7000,549.0000) --
        (528.6000,549.0000) -- (528.6000,548.5000) -- (528.6000,548.5000) --
        (528.7000,548.5000) .. controls (528.7000,548.5000) and (528.8000,548.5000) ..
        (528.8000,548.4000) .. controls (528.8000,548.4000) and (528.8000,548.4000) ..
        (528.9000,548.4000) -- cycle;

    \end{scope}
  \end{scope}
\end{scope}
% \end{tikzpicture}

            \end{scope}
        },
    }
}

% % region rhone alpes
% % TODO: rendre les couleurs parametrables
% \definecolor{cffffff}{RGB}{255,255,255}
% \definecolor{c828282}{RGB}{130,130,130}
% \tikzset{
%     pics/region/.style args={scale #1}{
%         code={
%             \begin{scope}[y=0.80pt, x=0.80pt, yscale=-1.000000, xscale=1.000000, inner sep=0pt, outer sep=0pt, scale=#1]
%                 \input{../figures/lib/region}
%             \end{scope}
%         },
%     }
% }

\lstdefinestyle{xmlfig}{
  backgroundcolor=\color{superlightgray},
  stringstyle=\color{darkgray},
  basicstyle=\tiny\ttfamily,
  language=XML,
  frame=single,
  numbers=none,
  basicstyle=\tiny\ttfamily,
  keywordstyle=\bfseries,
}
