\usepackage{adjustbox}
\usepackage{arydshln,tabulary,multirow,booktabs,array,bigdelim}
\usepackage{graphicx}
\usepackage{pbox}
\usepackage{rotating}
\usepackage{tikz}
\usetikzlibrary{calc,mindmap,trees,fit,positioning}
\tikzset{concept/.append style={fill={none}}}

\newlength{\mytablewidth}
\newlength{\myfirstcolwidth}
\newlength{\mycolwidth}
\newlength{\mylastcolwidth}
\newcommand\mrows[1]{\multirow{#1}{\mycolwidth}{}}
% accolade verticale sur plusieurs lignes
% http://tex.stackexchange.com/a/218053/32098
\newcommand\multibrace[3]{\rdelim\}{#1}{3mm}[\pbox{#2}{#3}]}
% pour laisser une "marque" utilisable plus tard avec tikz
% utile pour tracer des fleches au dessus des tableaux par exemple
\newcommand\tikzmark[1]{\tikz[overlay,remember picture] \coordinate (#1);}
\newcommand\centbf[1]{\centering\textbf{#1}}
\newcommand\centit[1]{\centering\textit{#1}}

% checkmarks
% http://tex.stackexchange.com/a/132800/32098
\def\checkmark{\tikz\fill[scale=0.4](0,.35) -- (.25,0) -- (1,.7) -- (.25,.15) -- cycle;}
\def\scalecheck{\resizebox{\widthof{\checkmark}*\ratio{\widthof{x}}{\widthof{\normalsize x}}}{!}{\checkmark}}

% http://tex.stackexchange.com/questions/245825/how-to-draw-an-ekg-tracing-with-tikz/245839#245839
% http://tex.stackexchange.com/questions/173569/tikz-pic-parameter
% carte de france
% TODO: rendre les couleurs parametrables
\definecolor{cFFFFFF}{RGB}{255,255,255}
\definecolor{cE3ECF6}{RGB}{227,236,246}
\definecolor{cBCBD43}{RGB}{188,189,67}
\definecolor{c999999}{RGB}{153,153,153}
\definecolor{c666666}{RGB}{102,102,102} % frontieres et regions
\definecolor{cCCCCCC}{RGB}{204,204,204}
\tikzset{
    pics/france/.style args={scale #1}{
        code={
            \begin{scope}[y=0.80pt, x=0.80pt, yscale=-1.000000, xscale=1.000000, inner sep=0pt, outer sep=0pt, scale=#1]
                \input{figures/france}
            \end{scope}
        },
    }
}

% region rhone alpes
% TODO: rendre les couleurs parametrables
\definecolor{cffffff}{RGB}{255,255,255}
\definecolor{c828282}{RGB}{130,130,130}
\tikzset{
    pics/region/.style args={scale #1}{
        code={
            \begin{scope}[y=0.80pt, x=0.80pt, yscale=-1.000000, xscale=1.000000, inner sep=0pt, outer sep=0pt, scale=#1]
                \input{figures/region}
            \end{scope}
        },
    }
}
