\documentclass{article}
\usepackage[T1]{fontenc}
\usepackage[utf8]{inputenc}
\usepackage[french]{babel}
\usepackage{tikz}
\usetikzlibrary{mindmap,trees}

\tikzset{concept/.append style={fill={none}}}


\begin{document}
\pagestyle{empty}
\begin{tikzpicture}
\hspace*{-5cm}
  \path[mindmap,concept color=black,text=black]
    node[concept, scale=0.8] {Analyse d'une architecture d'entreprise}
    [clockwise from=0]
    child[concept color=black] {
      node[concept] {Sujet de l'analyse}
      [clockwise from=60]
      child { node[concept] {Structure} }
      child { node[concept] {Dynamique} }
      child { node[concept] {Statistiques} }
    }
    child[concept color=black] {
      node[concept] {Référence temporelle}
     [clockwise from=-30]
      child { node[concept] {Ex post} }
      child { node[concept] {Ex ante} }
    }
    child[concept color=black] { 
    	  node[concept] {Techniques} 
     [clockwise from=-60]
      child { node[concept] {Basée sur les experts} }
      child { node[concept] {À base de règles} }
      child { node[concept] {À base d'indicateurs} }
    }
    child[concept color=black] { node[concept] {Préoccupations}
    [clockwise from=-150]
      child { node[concept] {Fonction\-nelles} }
      child { node[concept] {Non Fonctionnelles} }
    }
    child[concept color=black] { node[concept] {Auto-référence}
    [clockwise from=-180] 
      child { node[concept] {Aucune} }
      child { node[concept] {un niveau} }
      child { node[concept] {plusieurs niveaux} }      
    }; 
       
\end{tikzpicture}

\end{document}