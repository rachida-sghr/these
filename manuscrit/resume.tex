\chapter*{Résumé}

Les Smart Grids sont des réseaux électriques intelligents permettant d’optimiser la
production, la distribution et la consommation d’électricité grâce à l’introduction des
technologies de l’information et de la communication sur le réseau électrique. Les Smart
Grids impactent fortement l’ensemble de l’architecture d’entreprise des gestionnaires de
réseaux électriques. Simuler une architecture d’entreprise permet aux acteurs concernés
d’anticiper de tels impacts.

Dès lors, l’objectif de cette thèse est de fournir des modèles, méthodes et outils permettant de
modéliser puis de simuler une architecture d’entreprise afin de la critiquer ou de la valider.
Dans ce contexte, nous proposons un framework multi-vues, nommé ExecuteEA, pour
faciliter la modélisation des architectures d’entreprise en automatisant l’analyse de leurs
structures et de leurs comportements par la simulation. ExecuteEA traite chacune des vues
métier, fonctionnelle et applicative selon trois aspects~:~informations, processus et objectifs.

Pour répondre au besoin d’alignement métier/IT, nous introduisons une vue supplémentaire~:~
la vue intégration. Dans cette vue nous proposons de modéliser les liens de cohérence inter et
intra vues.

Nous mettons, par ailleurs, à profit des techniques issues de l’ingénierie dirigée par les modèles
en tant que techniques support pour la modélisation et la simulation d’une architecture
d’entreprise. Notre validons ensuite notre proposition à travers un cas métier Smart Grid
relatif à la gestion d’une flotte de véhicules électriques.