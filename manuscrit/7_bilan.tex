%!TEX  root = main.tex
\chapter{Bilan et perspectives}
\label{ch:bilan}

\PartialToc

\section{Résumé des contributions et positionnement}



\subsubsection{Par rapport aux autres cadres d'architecture}
par rapport à ToGAF
Conformité à ZAchman mais on va plus loin en formalisant un métamodèle
On ne veut pas le détails mais la vue holistique, les liens etc.
La revue de la littérature ainsi que
l'analyse des pratiques courante d'EA Plusieurs raisons
motivent l'utilisation de ces points de vue. D'abord, la vue métier et la vue
applicative sont incontournables pour n'importe quel cadre d'architecture.
Ensuite, selon les cadres d'architecture, la vue fonctionnelle est modélisée de
deux manière~:~ elle est soit intégrée à la vue applicative sous forme de
services (Archimate, TOGAF, RM-ODP), soit modélisée à part entière dans une vue
dédiée (Club Urba, \gls{sgam}, Zachman). Nous prenons le parti de modéliser
explicitement les fonctions dans une vue dédiée. En effet, passer directement
de la vue du métier à la vue applicative peut être en quelque sorte brutal pour
l'architecte métier mais aussi pour l'architecte applicatif. La vue fonctionnelle
permet une transition progressive de la logique métier vers l'architecture
logicielle.

Nous ne modélisons pas les informations dans une vue dédiée contrairement aux
cadre RM-ODP ou \gls{sgam}. Nous explicitons les informations en tant qu'aspect
pour chacune des autres vues comme recommandé par le cadre Zachman (Le quoi de
la dimension horizontale).

            \subsubsection{Par rapport aux autres approches selon la classification de Buckl}

        faire le repère 

\section{Perspectives}

    \subsection{Model Typing}

    \subsection{Vue technique}
    lien vers les Smart Grids
    
    \subsection{As is - To be}
