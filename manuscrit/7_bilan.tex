%!TEX  root = main.tex
\chapter{Bilan et perspectives}
\label{ch:bilan}

\PartialToc

\section{Résumé des contributions et positionnement}

\section{Perspectives}

Les travaux de cette thèse, faisant partie du projet Pomme, sont de nature prospective et visent à fournir 
des orientations claires sur la manière dont l'IDM, souvent réservée au génie logiciel, peut bénéficier à l'EA.
Dès lors, bien que convaincu de l'intérêt que représentent le \emph{framework} ExecuteEA et son métamodèle sous-jacent
EAT-ME pour la concrétisation des apports de l'IDM à l'EA, de nombreuses perspectives restent envisageables.

Ces perspectives sont de trois types. Les unes concernent l'amélioration du \emph{framework} ExecuteEA et 
et de son métamodèle en apportant des compléments à ce qui a déjà été réalisé comme l'intégration
du point de vue technique. Les autres correspondent à l'investigation pour l'EA
de nouvelles méthodes et technologies IDM comme le \emph{Model Typing} par exemple. D'autres, enfin,
correspondent à des centres d'intérêt nouveaux ayant pour cadre l'EA comme la modélisation du plan de migration
d'une architecture d'entreprise actuelle à une architecture d'entreprise cible.

% la modélisation et la simulation facilitent la conception d'une architecture d'entreprise
% et répondre efficacement aux impératifs de cohérence et d'alignement. Nous avons proposé à cet effet
% de mener des travaux d'investigation sur comment .


    \subsection{Model Typing}

 Describes the system’s functional elements, their responsibilities,
interfaces, and primary interactions. A Functional view is the cornerstone of
most ADs and is often the first part of the description that stakeholders try
to read. It drives the shape of other system structures such as the information
structure, concurrency structure, deployment structure, and so on. It also has a 
significant impact on the system’s quality properties such as its ability to
change, its ability to be secured, and its runtime performance.) 

\subsubsection{Le Model Typing pour la cohérence inter- et intra-vue}

\textit{Model Typing} est une technique de l'IDM appliqué au développement
logiciel permettant de contrôler les types de modèles d'entrée des
transformations de modèle à leur exécution. Nous proposons d'appliquer les
principes du \textit{Model Typing} aux modèles d'EA et aux transformations de
modèle qui leurs sont associées. Par exemple, un processus métier utilise en
entrée et en sortie des modèles représentant des concepts métier. Un processus
peut donc être considéré comme une transformation de modèle. Ainsi, le
\textit{Model Typing} peut être utilisé pour l'intégration horizontale (i.e.
cohérence et orchestration des processus d'une même vue).



    \subsection{Vue technique}
    lien vers les Smart Grids
    
    \subsection{As is - To be}

