%!TEX  root = main.tex
\chapter{Bilan et perspectives}
\label{ch:bilan}

\section{Bilan}

Les entreprises d'aujourd'hui évoluent dans un contexte économique
et technologique en constante et rapide mutation. Les Smart Grids
en sont la parfaite illustration. Ces derniers sont en effet à l'origine d'un changement
de paradigme sans précédent pour les énergéticiens et impliquent de ce fait
l'adoption de nouveaux processus métier, de nouvelles technologies pour le réseau
électrique, de nouveaux usages et services pour les consommateurs, etc. 

Nous sommes convaincus que l'EA peut apporter, dans ce contexte,
une aide précieuse pour anticiper ce changement de paradigme en dotant l'ensemble
des parties prenantes (décideurs et experts métier compris) d'une
vision holistique afin d'aligner les préoccupations des uns et des autres, qu'elles portent sur
des aspects stratégiques ou foncièrement techniques. Dès lors, notre vision de l'EA ne se réduit
pas au SI, et encore moins à l'IT de l'entreprise. L'EA reste certes un moyen indispensable
à l’exécution de la stratégie d'une entreprise, mais nous croyons qu'elle peut aussi
avantageusement orienter et éclairer cette stratégie.

Dans cette optique, l'architecture d'entreprise doit satisfaire aux impératifs d'évolutivité et
de réactivité face aux mutations permanentes de l'environnement de l'entreprise. Cette ambition
ne peut se concrétiser sans, d'une part, la garantie d'une grande cohérence  entre les vues d'architecture
et, d'autre part, la garantie d'une grande flexibilité, voire interchangeabilité, de ses composants. Notre approche
se fait fort de répondre à ces besoins de cohérence et de flexibilité à travers l'intégration et
la simulation d'une architecture d'entreprise. 

En recourant à l'IDM comme cadre méthodologique et technologique,
nous avons proposé à cet effet des modèles, outils et méthodes permettant 1) d'unifier les artefacts
issus des différentes activités d'EA (documentation, analyse et conception)
et 2) d'en faciliter la validation à travers la simulation.

En effet, ces activités sont aujourd'hui fastidieuses, sources d'erreurs,
et débouchent sur des artefacts hétérogènes prenant le plus souvent la forme
de documents descriptifs. Ces modèles informels et purement contemplatifs
ne permettent pas d'analyser efficacement 
la structure et le comportement de l'architecture d'entreprise obtenue.

Dans ce contexte, nous avons proposé un \emph{framework}
d'EA dirigé par les modèles exécutables~: ExecuteEA. Ce framework fournit à la fois une approche
conceptuelle pour l'analyse de la structure et du comportement d'une architecture d'entreprise, un cadre structurant dont nous
avons formalisé le métamodèle (EAT-ME) ainsi qu'un ensemble de langages et de techniques issus de l'IDM et adaptés à la modélisation
et à l'analyse d'une architecture d'entreprise. 

L'analyse de la structure vise à s'assurer de la cohérence globale d'une architecture d'entreprise. Nous avons
proposé pour cela de rajouter aux points de vue traditionnellement utilisés en EA (tels que les points de vue métier, fonctionnel et
applicatif) un point de vue intégration. Ce point de vue permet de modéliser les liens de cohérence inter-vues (entre les autres vues)
et les liens de cohérence intra-vue (entre les entités d'une même vue).

De plus, le développement d'un prototype et son application au cas métier
de la gestion d'une flotte de véhicules électriques concrétise l'ensemble des
propositions autour du \emph{framework} ExecuteEA et du métamodèle EAT-ME
et fournit un outil d'analyse de la structure et du comportement
d'une architecture d'entreprise.

Enfin, la dernière contribution de ces travaux est relative à la délimitation de l'objet d'étude.
Cette délimitation a été, en soi, à l'origine d'une démarche de recherche à part en entière.
Nous avons ainsi pu jeter les bases des travaux ayant abouti à la défintion du \emph{framework} ExecuteEA.

\section{Perspectives}

Les travaux de cette thèse, faisant partie du projet POMME d'EDF R\&D, sont de nature prospective et visent à fournir 
des orientations claires sur la manière dont l'IDM, souvent réservée au génie logiciel, peut bénéficier à l'EA.
Dès lors, de nombreuses perspectives restent envisageables.

Ces perspectives sont de trois types. Les unes concernent l'amélioration du \emph{framework} ExecuteEA et 
et de son métamodèle en apportant des compléments à ce qui a déjà été réalisé, comme l'intégration
du point de vue technique. Les autres correspondent à l'investigation pour l'EA
de nouvelles méthodes et technologies IDM comme le \emph{Model Typing} par exemple. D'autres, enfin,
correspondent à des centres d'intérêt nouveaux ayant pour cadre l'EA, comme la modélisation du plan de migration
d'une architecture d'entreprise actuelle à une architecture d'entreprise cible. 

% la modélisation et la simulation facilitent la conception d'une architecture d'entreprise
% et répondre efficacement aux impératifs de cohérence et d'alignement. Nous avons proposé à cet effet
% de mener des travaux d'investigation sur comment .


\subsection{Model Typing}

% \textit{Model Typing} est une technique de l'IDM appliqué au développement
% logiciel permettant de contrôler les types de modèles d'entrée des
% transformations de modèle à leur exécution. Nous proposons d'appliquer les
% principes du \textit{Model Typing} aux modèles d'EA et aux transformations de
% modèle qui leurs sont associées. Par exemple, un processus métier utilise en
% entrée et en sortie des modèles représentant des concepts métier. Un processus
% peut donc être considéré comme une transformation de modèle. Ainsi, le
% \textit{Model Typing} peut être utilisé pour l'intégration horizontale (i.e.
% cohérence et orchestration des processus d'une même vue).

Une démarche IDM implique une multitude de modèles produits et manipulés par différentes transformations de modèles. Le maintien de la cohérence entre modèles et transformation de modèle fait l'objet d'une recherche active dans une discipline comme l'IDM. \cite{steel2007model} propose de typer les modèles en entrée des transformations en utilisant le Model Typing. Celui-ci permet de contrôler les modèles manipulés par les transformations. Autrement dit, une transformation ne prend en entrée qu'un certain type de modèle. Ainsi, tous les modèles conformes à ce type peuvent être manipulés par la transformation en question. Ceci a, de plus, l'avantage d'augmenter la réutilisabilité des transformations écrites en mettant en évidence les caractéristiques communes des modèles. Cette technique de typage est en outre outillée et intégrée dans le langage de méta-modélisation Kermeta\footnote{www.kermeta.org} qui est basé sur EMOF\footnote{Ecore Meta Object Facility} dans un environnement Eclipse. Il nous paraît donc intéressant de mettre en œuvre le Model Typing pour la vue intégration.

\subsection{Vue technique}

    Dans ces travaux nous avons traité les vue métier, fonctionnelle et applicative. Nous souhaiterions également
    intégrer la vue technique au \emph{framework} ExecuteEA afin d'obtenir une analyse de la totalité de
    l'architecture d'entreprise. Cette vue correspond à l'infrastructure technique du SI
    (matériel informatique et réseaux de télécommunication) et permettra donc de faire le lien direct avec les équipements
    du réseau électrique. En effet, les travaux présentés dans ce mémoire se poursuivent avec un nouveau sujet de thèse
    au département \gls{mire} qui se porte entièrement sur la modélisation et la simulation de la vue technique, en continuité
    avec le \emph{framework} ExectueEA.

\subsection{Migration de l'architecture actuelle vers une architecture cible}

    Le \emph{framework} ExecuteEA peut être utilisé pour créer des modèles d'architecture d'entreprise exécutables
    afin d'analyser les composants d'une entreprise dans son état courant ou cible de manière séparée. 
    Cependant, pour retracer le cycle de vie d'une architecture d'entreprise et la localisation d'une architecture d'entreprise à
    travers ce cycle de vie, nous avons besoin d'un \emph{framewrok} retraçant son évolution. Des recherches en ce sens sont en cours
    mais uniquement au niveau du SI d'une entreprise~\cite{metrailler_evolis_2014}. Dès lors, il nous semble intéressant d'enrichir le
    \emph{framework} ExecuteEA pour tenir compte de l'évolution d'une architecture d'entreprise à travers son cycle de vie.

Plus globalement, le recours systématique aux modèles exécutables à tous les
niveaux de l'architecture reflète un parti pris fort de notre approche~: celui de l'adéquation de ces langages aux profils des
différentes parties prenantes. Même si l'usage du langage UML est de plus en plus
légitimé pour des vues non techniques telles que la vue métier (de nombreux démonstrateurs de Smart Grids font par exemple appel aux diagrammes d'activité et aux diagrammes de classes pendant la phase de spécification), il n'est
cependant pas largement adopté par les experts métier et les architectes d'entreprise qui se tournent
le plus souvent vers de la documentation. Nous réfléchissons, pour cette raison, à d'autres
langages de modélisation exécutables plus spécifiques à ces profiles d'utilisateurs et à leur cœur de métier
en envisageant la création de \gls{dsml} pour la vue métier et pour la vue intégration. Les DSML sont de plus promus par l'IDM.

Enfin, nous continuons à nous consacrer à (1) l'intégration du \emph{framework} ExecuteEA
à la plate-forme de co-simulation des domaines SI, électrotechnique et télécommunication
développée dans le cadre du projet POMME, (2) l'intégration de l'approche conceptuelle proposée
dans la démarche de spécification de Use Case pour les Smart Grids normalisée par la \gls{cei}.



