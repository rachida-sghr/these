%!TEX  root = main.tex
\chapter{Bilan et perspectives}
\label{ch:bilan}

\PartialToc

\section{Résumé des contributions et positionnement}


\section{Positionnement par rapport aux autres \emph{frameworks} d'EA}

\subsubsection{Par rapport aux autres cadres d'architecture}
par rapport à ToGAF

Conformité à ZAchman mais on va plus loin en formalisant un métamodèle

On ne veut pas le détails mais la vue holistique, les liens etc.

La revue de la littérature ainsi que
l'analyse des pratiques courante d'EA Plusieurs raisons
motivent l'utilisation de ces points de vue. D'abord, la vue métier et la vue
applicative sont incontournables pour n'importe quel cadre d'architecture.
Ensuite, selon les cadres d'architecture, la vue fonctionnelle est modélisée de
deux manière~:~ elle est soit intégrée à la vue applicative sous forme de
services (Archimate, TOGAF, RM-ODP), soit modélisée à part entière dans une vue
dédiée (Club Urba, \gls{sgam}, Zachman). Nous prenons le parti de modéliser
explicitement les fonctions dans une vue dédiée. En effet, passer directement
de la vue du métier à la vue applicative peut être en quelque sorte brutal pour
l'architecte métier mais aussi pour l'architecte applicatif. La vue fonctionnelle
permet une transition progressive de la logique métier vers l'architecture
logicielle.

Nous ne modélisons pas les informations dans une vue dédiée contrairement aux
cadre RM-ODP ou \gls{sgam}. Nous explicitons les informations en tant qu'aspect
pour chacune des autres vues comme recommandé par le cadre Zachman (Le quoi de
la dimension horizontale).


\begin{figure}[!ht]
	\begin{tikzpicture}[
    mynode/.style={inner sep=0pt, circle,draw,font=\footnotesize,minimum size=2.3cm,align=center},
    mycolorednode/.style={inner sep=0pt, circle,draw,font=\footnotesize,minimum size=2.3cm,align=center,fill=gray!60}]
    \node at (0,0) [mycolorednode] (center) {Intégration\\et Simulation} ;
    \foreach \i in {0,...,7} {
        \def\nodestyle{mynode}
        \ifthenelse{\i=0}{\def\mytext{\textbf{A}\\ Vision}\def\nodestyle{mycolorednode}}{}
        \ifthenelse{\i=1}{\def\mytext{\textbf{B}\\ Architecture\\ métier}\def\nodestyle{mycolorednode}}{}
        \ifthenelse{\i=2}{\def\mytext{\textbf{C}\\ Architectures SI}\def\nodestyle{mycolorednode}}{}
        \ifthenelse{\i=3}{\def\mytext{\textbf{D}\\ Architectures\\ techniques}}{}
        \ifthenelse{\i=4}{\def\mytext{\textbf{E}\\ Opportunités \\ et solutions}}{}
        \ifthenelse{\i=5}{\def\mytext{\textbf{F}\\ Plan de\\ migration}}{}
        \ifthenelse{\i=6}{\def\mytext{\textbf{G}\\ Gouvernance}{}}
        \ifthenelse{\i=7}{\def\mytext{\textbf{H}\\ Gestion du\\ changement\\ d'architecture}}{}
        \node at (90+-45*\i:4cm) [\nodestyle] (\i) {\mytext} ;
    }
    \node at (0, 7.3cm) [mynode] (preliminaires) {Préliminaires} ;

    \draw[angle 60-angle 60] (preliminaires) -- (0);

    \draw[-angle 60] (0) -- (1);
    \draw[-angle 60] (1) -- (2);
    \draw[-angle 60] (2) -- (3);
    \draw[-angle 60] (3) -- (4);
    \draw[-angle 60] (4) -- (5);
    \draw[-angle 60] (5) -- (6);
    \draw[-angle 60] (6) -- (7);
    \draw[-angle 60] (7) -- (0);

    \draw[angle 60-angle 60] (center) -- (0);
    \draw[angle 60-angle 60] (center) -- (1);
    \draw[angle 60-angle 60] (center) -- (2);
    \draw[angle 60-angle 60] (center) -- (3);
    \draw[angle 60-angle 60] (center) -- (4);
    \draw[angle 60-angle 60] (center) -- (5);
    \draw[angle 60-angle 60] (center) -- (6);
    \draw[angle 60-angle 60] (center) -- (7);
\end{tikzpicture}

	\caption{Positionnement de ExecteEA par rapport\\ au cadre TOGAF}
	\label{fig:positionTogaf}
\end{figure}

            \subsubsection{Par rapport aux autres approches selon la classification de Buckl}

\begin{figure}[!ht]
	\begin{tikzpicture}[scale=0.9]
    \path[
        mindmap,
        every node/.style={concept, color=black},
        level 1/.append style={sibling angle=360/5, distance=1cm},
        grow cyclic]
    node {L'analyse en Architecture d'Entreprise}
    child {
        node {Sujet de l'analyse}
        child {
            node [fill=gray!30]{Structure}
        }
        child {
            node [fill=gray!30]{Dynamique}
        }
        child {
            node {Statistiques}
        }
    }
    child {
        node {Référence temporelle}
        child {
            node [fill=gray!30]{Ex post}
        }
        child {
            node [fill=gray!30]{Ex ante}
        }
    }
    child {
        node {Techniques}
        child {
            node [fill=gray!30]{Basée sur les experts}
        }
        child {
            node [fill=gray!30]{À base de règles}
        }
        child {
            node {À base d'indicateurs}
        }
    }
    child {
        node {Préoc\-cupations}
        child {
            node [fill=gray!30]{Fonction\-nelles}
        }
        child {
            node {Non Fonctionnelles}
        }
    }
    child {
        node {Auto\-référentialité}
        child {
            node [fill=gray!30]{Aucune}
        }
        child {
            node {un niveau}
        }
        child {
            node {plusieurs niveaux}
        }
    };
\end{tikzpicture}

	\caption{Positionnement de ExecteEA par rapport\\au schéma de classification de
	Buckl et al. \protect\cite{buckl2009classifying}}
	\label{fig:positionBuckl}
\end{figure}

\section{Perspectives}

    \subsection{Model Typing}

 Describes the system’s functional elements, their responsibilities,
interfaces, and primary interactions. A Functional view is the cornerstone of
most ADs and is often the first part of the description that stakeholders try
to read. It drives the shape of other system structures such as the information
structure, concurrency structure, deployment structure, and so on. It also has a 
significant impact on the system’s quality properties such as its ability to
change, its ability to be secured, and its runtime performance.) 

\subsubsection{Le Model Typing pour la cohérence inter- et intra-vue}

\textit{Model Typing} est une technique de l'IDM appliqué au développement
logiciel permettant de contrôler les types de modèles d'entrée des
transformations de modèle à leur exécution. Nous proposons d'appliquer les
principes du \textit{Model Typing} aux modèles d'EA et aux transformations de
modèle qui leurs sont associées. Par exemple, un processus métier utilise en
entrée et en sortie des modèles représentant des concepts métier. Un processus
peut donc être considéré comme une transformation de modèle. Ainsi, le
\textit{Model Typing} peut être utilisé pour l'intégration horizontale (i.e.
cohérence et orchestration des processus d'une même vue).



    \subsection{Vue technique}
    lien vers les Smart Grids
    
    \subsection{As is - To be}

