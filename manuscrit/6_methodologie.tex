\chapter{Méthodologie de la recherche}
\label{ch:methodo}

\PartialToc

La méthodologie de recherche permet non seulement de comprendre la mise en place 
d'une démarche de recherche mais aussi les résultats l'étude. Le but de ce 
chapitre est double. D'une part, nous veillons à démonter l'adéquation entre 
notre démarche et l'objet de recherche. D'autres part, ce chapitre éclaire le 
cheminement des travaux de recherche pour comprendre la construction de la 
démarche adoptée. 

Une démarche classique de recherche commence par la formulation d'une question 
de départ. La question qui a initié nos travaux est la suivante~: «~Comment 
simuler afin de les valider les SI des Smart Grids ?~».  Comme en témoigne notre 
l'état de l'art, cette question fait l'objet de très peu de travaux (section 
\ref{approche_simu_existante}). Une recherche exploratoire a donc été nécessaire 
pour mettre en évidence les caractéristiques d'un phénomène nouveau selon une 
démarche inductive. 

Cette phase exploratoire a été préalable à la définition du cadre d'architecture 
\textit{ExcuteEA}. La construction de ce cadre d'architecture et sa mise à 
l'épreuve ont fait l'objet d'une recherche explicative pour laquelle nos avons 
adopté une démarche déductive.

L'intérêt de cette partie est de traiter la question de la cohérence entre nos 
objectifs de recherche et la démarche que nous adoptons pour y répondre. Mais 
nous ne cherchons pas à donner l'impression que notre plan de recherche a été 
entièrement établi avant de le mettre en œuvre. À l'inverse, nous l'avons 
construit au fur et à mesure de nos interactions avec le terrain d'étude. 

Tout d'abord, nous présentons la démarche inductive entreprise pour délimiter 
notre objet d'étude. Nous reprenons les étapes de cette recherche exploratoire 
par ordre chronologique~: exploration de la vue métier, puis de la vue 
applicative et en fin de la vue fonctionnelle. Nous présentons ensuite la 
démarche déductive ayant abouti à la construction du cadre d'architecture 
ExecuteEA.

%"Quelle que soit la nature de la démarche, la capacité d'ouverture et de prise 
%en compte d'éléments nouveaux est primordiale." ici ou dans la conclusion ?

%Dans une démarche classique de recherche, la définition de l'objet d'étude est 
%préalable à la 
%Plusieurs méthodes de recherche
%démarches de recherches employée par ordres chronologiques
%Combinaisons de méthodes 
%Vocation exploratoires de nos recherches
%Nous ne cherchons pas ici à donner l'impression que nous avons entièrement mis 
%au point notre plan de recherche avant de le mettre en œuvre. Celui-ci s'est au 
%contraire construit au fur et à mesure de notre interaction avec le terrain.
%
%"la cohérence du protocole de recherche avec la nature des questions que nous 
%nous posons"
%"Mais un autre élément mérite d'être distingué : la délimitation de l'objet 
%d'étude pose des problèmes tels qu'elle constitue une recherche en elle-même"
%"Quelle que soit la nature de la démarche, la capacité d'ouverture et de prise 
%en compte d'éléments nouveaux est primordiale."


	\section{Délimitation de l'objet de recherche}
%	Q : Est-ce que le recours à la demarche inductive est bien justifié ?
%		Est-ce l'investigation satisfait bien les critères de cohérence interne ?
		
	Une attention singulière est apportée à la délimitation de l'objet d'étude. 
Cette délimitation est souvent le résultat de l'observation du terrain d'étude~: 
l'entreprise et son SI d'une manière générale et l'entreprise et son SI dans le 
cas particulier des Smart Grids. L'entreprise et son SI forment un système 
complexe dont l'observation n'est pas triviale. La délimitation de l'objet 
d'étude a donc été en soi à l'origine d'une démarche de recherche.
	
	Cette partie a pour objectif de démonter la pertinence d'une démarche inductive 
pour l'identification de notre objet d'étude. D'une part, une démarche inductive 
est utile pour formuler des hypothèses ou soulever des questions et pour aborder 
un problème qui a été peu étudié comme c'est le cas de la simulation des SI. 
Elle aboutit à des propositions générales à partir de cas particuliers~: c'est 
une démarche par exploration. 
	Cette démarche est donc adaptée pour~: 
	\begin{enumerate}
	\item délimiter l'objet de l'étude, c'est à dire identifié ce qui est dans le 
contexte de la simulation des SI et ce qui ne l'est pas~;
	\item jeter les bases d'une étude théorique ultérieure.
	\end{enumerate}
	
	D'autre part, l'induction est une démarche de recherche classique en sciences 
sociales. Elle correspond au raisonnement empiriste qui affirme que 
l'observation et l'expérience sont la source de la connaissance du monde réel et 
du concret \cite{madeleine2001methodes}. Nous cherchons en effet à comprendre 
notre objet d'étude empiriquement. Le recours à cette démarche est d'autant plus 
justifié par la nature socio-technique du SI. En définissant le SI, Robert Reix 
met en évidence sa composante sociale (\ref{ch:EA}). 
	
	%Une étude ethnographique est Notre démarche inductive est donc doublée d'.

	La volonté d'identification de notre objet d'étude est portée par la question 
suivante «~Qu'est ce que la simulation d'un SI d'entreprise ?~». Néanmoins, même 
empirique, une démarche de recherche doit nécessairement s'inscrire dans un 
cadre de cohérence. Nous avons donc veillé à construire un protocole 
d'investigation épistémologiquement valide et conforme aux critères de cohérence 
interne. Notre protocole d'investigation est axé sur l'observation et 
l'expérience. Il est constitué de trois grandes étapes ~:
		
		\begin{enumerate}
		
	\item observation du terrain d'étude, c'est à dire analyse des pratiques 
courantes des personnes concernées par la simulation. Nous avons identifié deux 
catégories de personnes susceptibles de nous intéresser~: les experts (en SI ou 
en simulation) et les personnes susceptibles d'instrumenter la simulation des SI 
pour leurs travaux de recherche ou d'ingénierie (il s'agit là des utilisateurs 
finaux). Nous avons privilégié les ingénieurs-chercheurs d'\gls{edf}~R\&D car le 
contexte \gls{cifre} de la thèse a facilité l'accès à ces 
personnes.L'observation aboutit à la formulation d'hypothèses «~aprioristes~». 
Ce type d'hypothèse est exploratoire car elles ont pour but de soulever des 
interrogations~;
	%observatio participative, démarche ethnographique
	
	\item développement d'un prototype de simulation tenant compte du résultat des 
observations de l'étape précédente. Le prototype n'a pas pour vocation de 
proposer une solution finale mais plutôt tester rapidement les hypothèses 
formulées précédemment~;
	
	\item validation ou mise à l'épreuve du prototype sur le terrain d'étude. Cette 
mise à l'épreuve commence par la définition d'un cas d'application pertinent 
permettant de vérifier les hypothèses formulées à l'étape d'observation. Elle se 
poursuit par la collecte et l'analyse du retour des personnes concernées. Le 
contexte \gls{cifre} a là aussi facilité les échanges avec les 
ingénieurs-chercheurs de EDF R\&D, et en particulier ceux du département 
\gls{mire}. Notre intégration à l'équipe des ingénieurs-cherches du département 
\gls{mire}, et en particulier à l'équipe du projet de simulation des Smart Grid, 
a contribué à la qualité des échanges avec les personnes interrogées.
	
		\end{enumerate}
		
	Le raisonnement par induction aboutit à des propositions générales à partir de 
cas singuliers. Nous avons donc commencer par décomposer le terrain d'étude, 
c'est à dire le SI de l'entreprise. Les bases théoriques de la discipline des SI 
ont permis de procéder à cette décomposition afin de mettre en évidence ses 
singularités. Les approches par points de vue sont largement utilisée pour 
traiter la complexité des SI en le décomposant en plusieurs vues~: la vue 
métier, la vue fonctionnelle, la vue applicative, la vue technique. Chaque vue 
correspond à la perspective d'un groupe de personnes aux profils différents mais 
complémentaires. Les investigations ont été menées sur les trois premières vue — 
métier, fonctionnelle, applicative. La vue technique n'a pas été traitée~: le 
temps nécessaire aux expérimentations est incompatible avec les délais de cette 
thèse et son financement dans le cadre d'une \gls{cifre}.
	
	Le protocole d'investigation est alors appliqué à chacune des vues métier, 
fonctionnelle et applicative. L'objectif de la démarche engagée est de définir 
l'objet d'étude, en ayant comme question de départ «~Qu'est ce que la simulation 
d'un SI d'entreprise ?~». Cependant, nous avons veillé à garder une capacité 
d'ouverture aux idées nouvelles. Nous présentons
	
		\subsection{Investigations menées pour la vue métier}
			La première étape d'observation a débuté avec un stage de fin d'étude de six 
mois que nous avons effectué au sein du département \gls{mire}. L'objectif du 
stage a consisté à explorer le sujet «~Simulation du SI des Smart Grids~» afin 
de préparer un sujet de thèse. Il s'est donc accordé avec l'objectif des 
investigations menées pour la vue métier.
			
			\subsubsection{Observation}
				Pour cette première phase d'observation, des entretiens ont été menés avec 
des experts SI internes à l'entreprise et mais aussi externes à celle-ci lors 
d'un séminaire professionnel ayant pour thème la modélisation des SI \footnote{Model Driven Day, 21 novembre 2001, Paris 
Cœur Défense}. Des entretiens ont aussi été menés avec des ingénieurs-chercheurs 
du département \gls{mire} ayant participé à des démonstrateurs Smart Grids 
européens pour mettre en évidence les pratiques de spécification de la 
composante SI des Smart Grids. 

				Les hypothèses formulées à l'issue des ces observations sont les suivantes~:
				\begin{itemize}
					\item la simulation de SI est une discipline peu étudiée~;
					\item la simulation des processus métier est pertinente pour les SI des 
Smart Grids dans le mesure ou elle permet de valider les scénarios élaborés pour 
les démonstrateurs, mais aussi pour les SI tout court~;
					\item lors de cette simulation, il est nécessaire de maintenir une 
cohérence entre le processus et les données qu'il manipule~;
					%\item les langages de modélisation exécutables présentent des avantages pour la simulation des SI.
				\end{itemize}
		
			\subsubsection{Prototypage}
				Le prototypage a nécessité une étude des outils existants proposant de 
simuler des processus métier, ce qui a permis d'identifier les outils suivant~: 
Enterprise Architect, Bonita, Amuse et Rhapsody. Il a ensuite essentiellement à 
tester leur capacité de simulation selon une grille d'évaluation. Le critère de 
sélection principal a été la capacité de l'outil à exécuter des diagrammes 
d'activité UML. En effet, c'est dans ce langage que sont représentés les 
processus métier dans les documents de spécification des démonstrateurs Smart 
Grid. Le deuxième critère d'évaluation retenu a été la possibilité d'ajout de 
nouvelles fonctionnalités à l'outil. À l'issue de cette étude comparative, 
l'outil Enterprise Architect dans sa version 9.2 a été retenu. En effet, à 
l'époque de l'étude, EA était le seul à pouvoir animer des diagrammes d'activité 
et à offrir la possibilité d'ajout de fonctionnalité par le mécanisme de plugin. 
C'était en outre l'outil de modélisation UML de référence de l'équipe 
d'ingénieurs-chercheurs au sein de laquelle nous avons effectué ce stage.
				
				Cependant, dans sa version 9.2, l'outil n'assure pas la cohérence entre les 
objets métier (modélisés avec diagramme de classe) et le processus (modélisé 
avec un diagramme d'activité) au cours de la simulation. De plus, l'outil ne 
gère pas la persistance des résultats de la simulation. La mise en cohérence a 
donc nécessité le développement d'un plugin que nous avons baptisé DataSimu. 
DataSimu permet de (1)~créer un jeu de donnée en entrée de la simulation à 
partir des concepts métier en instanciant un diagramme de classe (2)~simuler le 
processus métier avec ce jeu de donnée (3)~récupérer le jeu de données en sortie 
de la simulation et les stocker dans une base de données. Nous avons mené 
l'implémentation de DataSimu en binôme avec un étudiant en troisième année 
d'école d'ingénieur. L'interface graphique de DataSimu est illustrée par la 
figure. L'annexe (ANNEXE) détaille l'architecture de DataSimu et présente un 
manuel d'utilisation. 
				
\begin{figure}[!ht]
 \begin{center}
  \includegraphics[width=1\textwidth]{figures/6_methodologie/data_simu.png}
 \end{center}
 \caption{Interface Graphique de DataSimu}
 \label{fig:data_simu}
\end{figure}
				
			\subsubsection{Validation}
			Pour valider les hypothèses formulées à l'issue de l'observation, un cas 
d'application Smart Grid a été mis au point~: le pilotage d'une charge 
domestique. Ce cas d'application est issu de l'étude des spécifications des 
démonstrateurs Smart Grid ADDRESS et PREMIO introduits dans la 
section~\ref{sec:DemonstrateursSG}. Il s'agit de piloter une batterie de 
stockage d'énergie installée chez un client (particulier ou industriel). En 
fonction de l'état du réseau, une centrale de pilotage contrôle cette batterie 
(stockage d'énergie pour une utilisation ultérieure), tout en tenant compte des 
consignes du client. Ce cas métier a été modélisé et simulé avec l'outil 
Enterprise Architect doté du plugin DataSimu. Une description détaillée du cas 
métier et du déroulement de la simulation est donnée dans l'annexe 
\ref{annexe:DataSimu}. 
			
			Le prototype de simulation et sa mise en œuvre à travers le cas métier du 
pilotage d'une charge domestique ont été soumis aux experts SI du département 
\gls{mire} et aux ingénieurs-chercheurs contribuant aux démonstrateurs Smart 
Grid PREMIO et ADDRESS. Les entretiens suivant la démonstration ont validé (1) 
la pertinence de la simulation dans le contexte des SI des Smart Grids (2) la 
séparation du processus métier et des objets métier tout en maintenant une 
cohérence lors de la simulation. 
			
			Ces entretiens, assortis l'étude des outils de simulation des processus 
métier, ont permis de constater que la question de la simulation est peu abordé 
dans le contexte des SI. Ces travaux d'investigation ont de plus donné lieu à 
une publication \cite{seghiri2012animation} et ont été poursuivis par les 
travaux présentés dans cette thèse.
 
				\subsubsection{Conclusion}
			Cette première application du protocole d'investigation a conforté nos hypothèses initiales à savoir que la simulation est peu abordée dans le contexte des SI mais qu'elle est pertinente pour valider/critiquer les scénarios de cas métier Smart Grid avant leur implémentation.

			\subsection{Investigations menées pour la vue applicative} 
			\label{sec:investig_appli}
				La vue applicative tient une place de choix dans les SI des entreprises. Pour des SI fortement informatisé, il arrive même souvent que le SI soit réduit aux applications informatiques et à l'infrastructure qui les supportent. Cette constatation est encore plus avérée dans le cas des Smart Grids dont le principe est le déploiement de TIC sur le réseau électrique pour automatiser son pilotage. Le choix chronologique de poursuivre les investigations en abordant la vue métier découle de cette constatation.		
			
				\subsubsection{Observation}
				Les investigations menées pour la vue applicative ont nécessité d'approfondir nos connaissances du fonctionnement du réseau électrique. Pour cette deuxième phase d'observation, nous avons conduit des entretiens avec deux profiles de personnes~: des experts et des ingénieurs-chercheurs spécialisé dans le réseau électrique de distribution  appartement au département \gls{mire}. En effet, plus que les réseaux de transport, ce sont les réseaux de distribution qui sont concernée par les Smart Grids. Les réseaux de transports français sont déjà fortement automatisés. 
				L'objectif de ces entretiens est double~: approfondir nos connaissances du réseau électrique et comprendre les pratiques des personnes interrogées en matière de simulation.
				Les experts sus-mentionnés sont responsables de la conception d'applications pour la conduite du réseau électrique. Le langages de conception les plus utilisés sont les automates programmables. Les ingénieurs-chercheurs sont quand à eux responsable du développement des applications, le plus souvent en Matlab ou  C++.
				Les hypothèses formulées à l'issue de cette observation sont les suivante~:
				\begin{itemize}
					\item la simulation du SI pour les Smart Grid est liée à la simulation des réseaux électriques~;
					\item la simulation du SI des Smart Grid nécessite de traiter la problématique de l'hétérogénéité des modèles.
				\end{itemize}
				
				\subsubsection{Prototypage}
				
				Nous avons utilisé l'outil de modélisation hétérogène Ptolemy\footnote{http://ptolemy.eecs.berkeley.edu/}  pour développer un prototype de simulation illustré par la figure~\ref{fig:simu_ptolemy}. L'observation a permis d'identifier les éléments à modéliser, à savoir le SI et le réseau électrique. Ici l'hétérogénéité des modèles provient de leur dépendance au temps. La modélisation du comportement du SI et celui du réseau électrique fait intervenir deux domaines de calcul différents~: à temps discret pour le SI et à temps discret périodique pour le réseau électrique. 
				Ainsi, dans le modèle Ptolemy illustré par la figure ~\ref{fig:simu_ptolemy}, le domaine de calcul adopté pour le SI est le \textit{Descrete Event} (DE), celui adopté pour le réseau de distribution électrique est le \textit{Synchronous Data Flow} (SDF). Ptolemy permet d'adapter ces deux domaines de calcul pendant la simulation d'un cas métier et d'adresser l'hétérogénéité des modèles du SI et du réseau électrique. 
				
\begin{figure}[!ht]
 \begin{center}
  \includegraphics[trim = 0cm 8cm 0cm 0cm, width=1\textwidth]{figures/6_methodologie/simu_ptolemy.pdf}
 \end{center}
 \caption{Prototype Ptolemy pour une simulation hétérogène comprenant le SI (discret) et le réseau électrique (continu)}
 \label{fig:simu_ptolemy}
\end{figure}
		
				\subsubsection{Validation}
		Le cas d'application du pilotage d'une charge domestique n'a pas été utilisé pour le prototypage de la vue applicative. Les modèles de la vue applicative nécessitent un niveau de détail plus élevé que les modèles de la vue métier. Or les spécifications des démonstrateurs ADDRESS et PREMIO n'offrent pas le niveau de détails nécessaires. Le Use Case normalisé par \gls{enel} (voir section~\ref{sec:ENEL} pour la régulation de tension sur les réseau de distribution a été adopté et affiné par les discutions avec les ingénieurs-chercheurs du département \gls{mire} travaillant sur la même thématique. 
		
		Ils s'agit d'adapter la tension du réseau électrique en fonction de la charge et de la production pour maintenir un niveau de tension respectable. Les deux leviers d'action sont utilisés, les régleurs en charge et les \gls{der}. Les régleurs en charges pilotés à distance agissent sur le niveau de tension au niveau des postes sources. Le pilotage des \gls{der} permet de contrôler la quantité d'énergie qu'ils injectent sur le réseau. Ce cas métier fait intervenir un SI qui calcule les consignes envoyées aux DER et au régleur en charge et un réseau électrique qui réagit à ces consignes. Il est à noté que le SI réduit à sa vue purement applicative en ne de modélisant que l'application qui calcule les consignes.
		
		Ce cas métier a été modélisé puis simulé avec le prototype Ptolemy de modélisation et de simulation hétérogènes. Le comportement du SI est modélisé avec une machine à état qui calcule des consignes pour le DER et le régleur en charge en fonction du niveau de tension du réseau électrique. Le DER et le régleur en charge  régulent la tension du réseau électrique en appliquant ces consignes. Le réseau de distribution est composé d'un DER, d'un consommateur d'électricité et d'un régleur en charge dont les comportements font varier le niveau de tension.
		
		La simulation de la régulation de tension avec le prototype Ptolemy ont été soumis aux experts et ingénieurs chercheurs identifiés dans la phase d'observation. Bien que leurs retours ait permis de confirmer les hypothèses formulées concernant la problématique d'hétérogénéité, cette problématique a été écartée de notre périmètre de recherche. En effet, les ingénieurs-chercheurs spécialistes du réseau électrique de distribution utilisent leur propres outils de simulation. Le prototype développé a permis de mettre en évidence la problématique de l'hétérogénéité des modèles mais pas de les résoudre. Un projet de co-simulation des domaines SI, réseau électrique et télécommunication a été lancé, suite aux résultats de différents projets de simulation dans le département \gls{mire}, dont nos travaux de thèse pour le domaine SI.
		
		  
				\subsubsection{Conclusion}
		Cette deuxième application du protocole d'investigation pour la vue applicative a permis de confirmer nos hypothèses concernant la problématique de l'hétérogénéité des modèles de simulation mais surtout de réduire notre périmètre de recherches. En effet, l'hétérogénéité est traitée par un projet du département \gls{mire}, auquel nos travaux ont été intégrés. L'implémentation du cas métier de la régulation de tension à travers le prototype Ptolemy, et le retours qu'en ont fait les personnes interrogées, nous a en outre permis d'affiner le cas métier de la régulation de tension. Sa réutilisation pour les investigations menées pour la vue fonctionnelle en a été d'autant plus facilitée.   
	
		\subsection{Investigations menées pour la vue fonctionnelle} 
		La vue fonctionnelle est une vue charnière entre le métier et les applications qui les implémente. En effet, elle décompose chacune tâche métier en fonctions. Les investigations menées sur la vue fonctionnelle sont essentiels dans la mesure où cette vue permet de maintenir le lien entre le métier et l'IT. 
			\subsubsection{Observation}
		Pour cette dernière phase d'observation, nous avons mené des entretiens avec des personnes du domaine SI et des personnes du domaine du réseau électrique. Nos observations ont permis de constater que les ingénieurs-chercheurs du département \gls{mire} qui conçoivent des applications pour les réseaux électriques, utilisent leurs propres systèmes de notations et ne sont que très peu familiers avec les concepts du domaine SI telle que la vue fonctionnelle. Il s'est avéré en effet, qu'une distinction entre SI de gestion et SI industriel est faite au sein du département. Les applications qui automatisent la conduite du réseau relèvent des SI industriels. Les SI de gestion, ou SI transverses, correspondent à des SI intégrant des actions humaines dans leurs processus ou gérant des activités liées aux personnes.
		Ainsi, les hypothèses formulées suite à ses observations sont les suivantes~:
		\begin{itemize}
			\item la simulation des SI des Smart Grids est aussi pertinente pour les SI industriels que pour les SI de gestion~;
			\item le langage de modélisation su SI doit être compréhensibles par les parties prenantes~;
		\end{itemize}	 
			\subsubsection{Prototypage}
		Les hypothèses formulées à l'issue de l'observation impliquent le prototypage nécessite la mise au point préalable d'un cas métier. D'une part, les investigations pour la vue métier ont permis de validé la pertinence de la simulation d'un SI de gestion. En effet, le pilotage d'une batterie installée chez un client implique l'intervention systématique de ce dernier à travers son consentement/refus à stocker ou injecter l'énergie de sa batterie en fonction de la compensation tarifaire perçue. Il a été donc plus judicieux de s'orienter vers un SI industriel pour cette dernière application du protocole  d'investigation. D'autre part, la deuxième hypothèse portant sur la compréhension du langage de modélisation su SI par les partie prenante a orienté le prototypage vers la création d'un \gls{dsml}. La création d'un \gls{dsml} nécessite cependant de connaitre au préalable son domaine d'application, qui est ici le cas métier Smart Grid traité. 
		Le cas métier retenu est la régulation de tension d'un réseau électrique présentant une forte pénétration de \gls{der}. En effet, les ingénieurs-chercheurs du département \gls{mire} considèrent qu'il relève du domaine des SI industriel. Son implémentation pour la vue applicative a permis en outre de l'affiner. 
		Ainsi, le prototypage a consisté à développer un \gls{dsml} pour la vue fonctionnelle selon une démarche IDM. Après l'analyse du processus fonctionnel de la régulation de tension, nous avons développé un métamodèle. Dans ce métamodèle, illustré par la figure~\ref{fig:meta_dsml}, les concepts essentiels d'un processus fonctionnel de régulation de tension ont été définis en respectant les termes utilisés par les experts du réseau électrique~:
		
		\begin{itemize}
		
\item Événement (\textit{Event})

Ce sont les événements qui peuvent apparaitre sur le réseau de distribution. Il s'agit de la contrainte haute (la tension sur le réseau dépasse la tension réglementaire U\textsubscript{Max}), de la contrainte basse (la tension sur les réseau dépasse la tension réglementaire U\textsubscript{Min}) et contrainte à la fois haute et basse~;

\item Action de régulation (\textit{RegulationAction})

Ce sont les leviers à actionner pour adresser une contrainte : élever le niveau de tension via le régleur en charge (\textit{IncreasePad}), baisser le niveau de tension via le régleur en charge (\textit{DecreasePad}), ou en effaçant un DER (\textit{DeleteDER})~;

\item Contrainte à respecter (\textit{Constraint})

Le processus fonctionnel de régulation de tension est limité par des contraintes liée aux équipements du réseau. Le régleur en charge abaisse (respectivement élève) la tension sans dépasser une marge basse (LowMargin) (respectivement une marge haute(\textit{HighMargin})). Il n'est de plus pas possible de mettre le régleur en charge en butée haute ou basse (\textit{LowerStop, UpperStop}). En effet, quand le régleur en charge abaisse ou élève la tension, il change de plot. Le nombre de plot étant limité, il est interdit d'utiliser les plots extrêmes par mesure de sécurité~;

\item Structure de contrôle (\textit{ControlStructure})

Pour ce prototype, nous avons uniquement implémenté le \textit{If}. 

		\end{itemize}
		
\begin{figure}[!ht]
 \begin{center}
  \includegraphics[trim = 0cm 3cm 0cm 0cm, width=1\textwidth]{figures/6_methodologie/metamodele_dsml.pdf}
 \end{center}
 \caption{Métamodèle d'un processus fonctionnel de régulation de tension sur un réseau de distribution électrique}
 \label{fig:meta_dsml}
\end{figure} 

Une syntaxe concrète et une sémantique d'exécution ont aussi été conçues pour ce \gls{dsml}. Le DSML a ensuite été implémenté par un stagiaire dans l'environnement Eclipse. L'utilisation d'Eclipse est largement répondue dans la communauté de l'IDM. la fondation Eclipse héberge le projet \textit{Eclipse Modeling }qui propose des langages et outils dédiées au développement de \gls{dsml}. La sémantique d'exécution a été implémentée avec Kermeta. Kermeta offre la possibilité de définir de spécifier la sémantique d'exécution directement au niveau du métamodèle à l'aide un langage d'action. 

		
			\subsubsection{Validation}
Le prototype ainsi implémenté permet de créer et de simuler des processus fonctionnel pour la régulation de tension. La figure\ref{fig:proto_dsml} est une capture d'écran représentant le prototype. La partie droite correspond à la palette de création de processus de régulation où se trouvent les concepts du métamodèle sous leur forme graphique. La partie gauche correspond à un exemple de processus modélisé avec cette palette. 

\begin{figure}[!ht]
 \begin{center}
  \includegraphics[trim = 0cm 0cm 0cm 0cm, width=1\textwidth]{figures/6_methodologie/proto_dsml.pdf}
 \end{center}
 \caption{Exemple de processus fonctionnel de régulation de tension modélisé avec le prototype de \gls{dsml}}
 \label{fig:proto_dsml}
\end{figure} 

Le prototype a été soumis à des personnes appartenant aux profiles identifiés dans l'observation, c'est à dire des personnes du domaine SI et des personnes du domaine du réseau électrique. Les personnes du domaine SI sont principalement des architectes SI. Leur retour a été positif. Ils ont trouvé dans le DSML développé un moyen de modéliser des processus pour le métier et d'échanger avec les personnes du domaine électrique qui ne maitrisent pas toujours les langages traditionnellement utilisés dans le domaine SI comme UML. Le retour des personnes du domaine du réseau électrique n'a pas été aussi enthousiaste que celui des personnes du domaine SI. Il s'agit d'abord d'un problème de sémantique. En effet, il n'ont pas vu d'intérêt à modéliser des « fonctions » de conduite de réseau avec un nouveau DSML. Le terme « fonction » est associé au domaine purement électrique et non SI. Or ces fonctions sont habituellement modélisées avec les automates programmables ou encore le langage Matlab. Ces langages sont éprouvés pour le domaine du réseau électrique mais ne sont pas adaptés à la vue fonctionnelle du SI. Cette dernière ne met l'accent sur le détail de l'implémentation de la fonction mais plutôt sur l'orchestration des plusieurs fonctions et leur structuration en blocs fonctionnels.

L'analyse du résultat des entretiens menés avec les personnes du domaine SI et les personne du domaine des réseaux électrique a validé partiellement les hypothèses formulées à l'issue de l'observation. Ainsi, en montrant leur intérêt pour le DSML, les personnes du domaine SI ont affirmé l'intérêt de modéliser et de simuler le SI avec des langages compréhensibles par les personnes du métier, en l'occurrence celles du domaine du réseau électrique. 
En revanche, le retour des personnes du domaine du réseau électrique a, au regard de notre analyse, infirmé l'hypothèse selon laquelle a simulation des SI des Smart Grids est aussi pertinente pour les SI industriels que pour les SI de gestion. En effet, les SI industriels sont déjà l'objet de simulations (avec les automates programmables et Matlab par exemple).

			\subsubsection{Conclusion}
Cette dernière application du protocole d'investigation a cette étude a permis en outre d'identifier deux types de SI pour les Smart Grid. L'étude a confirmé la pertinence de la simulation de SI pour la vue fonctionnelle, mais seulement pour un SI de gestion. Les SI purement informatiques, autrement dit ceux qui ne font pas intervenir de tâche humaine et qui pilotent directement les équipements électriques, ne sont pas concernés par notre recherches. En effet, ils existent déjà des langages dédiée à leur modélisation et simulation (automates programmables et Matlab). Le terme «~SI industriel~» employé pour qualifier ces systèmes a été, à notre sens trompeuse dans le sens où il s'agit plutôt d'une restriction ou d'une spécialisation du terme SI. En effet, selon la définition de Reix (cf. section~\ref{sec:reix}), toute ressource intervenant sur le cycle de vie d'une information au sein d'une organisation fait partie du SI, y compris le personnel. 
	
	
		\subsection{Conclusion}
Redéfinition de l'objet d'étude
Reformulation de la question de départ -> ref problématique 
Le cœur de la problématique 
Adéquation de la démarche et de l'objectif de la recherche (objet d'étude)
	
La question de l'analyse par simulation du SI fait l'objet de 
Démarche exploratoire / Démarche ethnographique 
La délimitation du sujet de recherches est en soit à l'origine d'une démarche de 
recherche à part entière  inductive, se prête mieux au sujet nouveau, faisant 
l'objet de peu de travaux
Jeter les bases d'une étude ultérieure


Mettre en évidence les caractéristiques du phénomène et construire des 
hypothèses 
"Les hypothèses et même les questions sont susceptibles d'évoluer au fur et à 
mesure de la recherche"

"En retour, le travail empirique se verra régulièrement réorienté en fonction 
des approfondissements successifs du cadre théorique "

Étude sociologique -> entretien avec les experts, sur leur avis de nos 
prototypes mais aussi 

Plusieurs démarches d'investigation 
prototypage et entretien 
ou SI transverse et SI spécialisé ? SI purement logiciel / SI socio-technique 
(interaction humaine) 
Une différence d'échelle 
		 


	\section{Conceptualisation / construction du cadre d'architecture 
\textit{ExecuteEA}}
%À partir de la compréhension empirique de notre objet d'étude et de la 
formulation d'une problématique 
La reformulation de la question de départ abouti à la définition d'une 
problématique 
La question de départ est au cœur de la problématique 

permet la Définition du cadre théorique 

La proposition est au contraire déductive
Sujet identifiés, hypothèses formulées
systèmes similaires : SI et entreprise -> établir les similarités
Les hypothèses etc.  
Choix de la théorie à appliquer : IDM (hypothèse et conclusion)/ apport de l'IDM 
a-à l'EA, hypothèse (même problématique) à une échelle différente  ‹
Itérative et là Smook 

	\subsubsection{Conclusion}
	

L'observation menée à travers les entretiens et la revue des standards utilisés dans le domaine SI a permis d'identifier un nouveau langage de modélisation fUML. Encore en phase d'élaboration au sein de l'\gls{omg}, ce langage n'a pas pu être testé. Cependant, nous l'avons identifié comme candidat potentiel pour la simulation de diagrammes d'activité.

Le protocole d'investigation s'est de plus révélé adéquat avec notre terrain d'étude. En effet, l

Ces experts ne sont pas sensibilisés au domaine SI et considèrent les applications développées pour le réseau électrique comme partie intégrante de celui 

Les investigations menées pour la vue applicative a permis de recentrer la simulation des les SI des Smart Grids sur la simulation de leurs SI uniquement sans traiter la question de la co-simulation SI/réseau électrique comme expliqué dans la section~\ref{sec:investig_appli}. 
	

