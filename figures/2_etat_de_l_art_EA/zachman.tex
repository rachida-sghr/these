% Resources:
% Arrows: http://tex.stackexchange.com/a/60627/32098
% Rotating tikz label: http://tex.stackexchange.com/a/115565/32098

% HACK (and an ugly one) since booktabs breaks vertical separators, we use tikz
% to draw them. Alignment is pretty much custom. I just hope this is gonna work
% with no adjusment when putting this in the main document.
\setlength{\mytablewidth}{\textwidth}
\setlength{\myfirstcolwidth}{\dimexpr0.2\mytablewidth-2\tabcolsep\relax}
\setlength{\mycolwidth}{\dimexpr0.17\mytablewidth-2\tabcolsep\relax}
\newcolumntype{C}[1]{>{\centering\arraybackslash}m{#1}}
\newcommand\mycell[1]{{\tiny{#1}}}

\begin{adjustbox}{width=\mytablewidth,center}
    \scriptsize
    \noindent\begin{tabulary}{\mytablewidth}{m{\myfirstcolwidth}C{\mycolwidth}C{\mycolwidth}C{\mycolwidth}C{\mycolwidth}C{\mycolwidth}}

        % \cmidrule[\heavyrulewidth]{2-6}
        \multirow{2}{*}{}\tikzmark{zachmantopleft} \
        & \centbf{Données} \
        & \centbf{Fonctions} \
        & \centbf{Personnel} \
        & \centbf{Temps} \
        & \centbf{Motivation}\tikzmark{zachmantopright} \
        \tabularnewline
        & \centit{Quoi} \
        & \centit{Comment} \
        & \centit{Qui} \
        & \centit{Quand} \
        & \centit{Pourquoi} \
        \tabularnewline\midrule

        \tikzmark{zachmanlefttop}\textbf{Exécutif}\newline\textit{Planification} \
        & \tikzmark{tl1}{\tiny{Identification des données}} \
        & \tikzmark{tl2}{\tiny{Identification des processus}} \
        & {\tiny{Identification des responsabilités}} \
        & {\tiny{Identification des échéances}} \
        & \tikzmark{tl3}{\tiny{Identification des motivations}} \
        \tabularnewline\midrule

        \textbf{Management}\newline\textit{Définition} \
        & {\tiny{Définition des données}} \
        & {\tiny{Définition des processus}} \
        & {\tiny{Définition des responsabilités}} \
        & {\tiny{Définition des échéances}} \
        & {\tiny{Définition des motivations}} \
        \tabularnewline\midrule

        \textbf{Architecte}\newline\textit{Conception}  \
        & {\tiny{Conception des données}} \
        & {\tiny{Conception des processus}} \
        & {\tiny{Conception des responsabilités}} \
        & {\tiny{Conception des échéances}} \
        & {\tiny{Conception des motivations}} \
        \tabularnewline\midrule

        \textbf{Ingénieur}\newline\textit{Spécification} \
        & {\tiny{Spécification des données}} \
        & {\tiny{Spécification des processus}} \
        & {\tiny{Spécification des responsabilités}} \
        & {\tiny{Spécification des échéances}} \
        & {\tiny{Spécification des motivations}} \
        \tabularnewline\midrule

        \tikzmark{zachmanleftbottom}\textbf{Technicien}\newline\textit{Implémentation} \
        & {\tiny{Implémentation\newline des données}\tikzmark{br1}} \
        & {\tiny{Implémentation\newline des processus}\tikzmark{br2}} \
        & {\tiny{Implémentation\newline des responsabilités}} \
        & {\tiny{Implémentation\newline des échéances}} \
        & {\tiny{Implémentation\newline des motivations}\tikzmark{br3}} \
        \tabularnewline\bottomrule
    \end{tabulary}
    \begin{tikzpicture}[overlay,remember picture]

        % top horizontal arrow
        \draw[<->] let \p1=(zachmantopleft), \p2=(zachmantopright) in ($(\x1,\y1)+(1.6,0.4)$) -- node[label=Abstractions (colonnes)] {} ($(\x2,\y2)+(0.4,0.4)$);

        % left vertical arrow
        \draw[<->] let \p3=(zachmanlefttop), \p4=(zachmanleftbottom) in ($(\x3,\y3)+(-0.3,0.3)$) -- node[label={[label distance=-2ex, text depth=3ex, label position=above, rotate=90]above:Perspectives (lignes)}] {} ($(\x3,\y4)+(-0.3,-0.45)$);

        \draw[fill=gray,opacity=0.2,rounded corners=1pt] let \p1=(tl1), \p2=(br1) in ($(\x1, \y1)+(-0.1,-0.1)$) rectangle ($(\x2, \y2)+(0.4,0.2)$);
        \draw[fill=gray,opacity=0.2,rounded corners=1pt] let \p1=(tl2), \p2=(br2) in ($(\x1, \y1)+(-0.1,-0.1)$) rectangle ($(\x2, \y2)+(0.3,0.2)$);
        \draw[fill=gray,opacity=0.2,rounded corners=1pt] let \p1=(tl3), \p2=(br3) in ($(\x1, \y1)+(-0.1,-0.1)$) rectangle ($(\x2, \y2)+(0.25,0.2)$);
%         \def\zachmancolwidth{2.41}  % ~ column width (found manually)
%         \def\zachmanleftoffset{3.55}  % ~ first column offset (found manually)
%         \newcommand\drawcolsep[1]{%
%             \draw[dotted] let \p3=(zachmanlefttop), \p4=(zachmanleftbottom) in ($(\x3,\y3)+(\zachmanleftoffset+#1*\zachmancolwidth,0.24)$) -- ($(\x3,\y4)+(\zachmanleftoffset+#1*\zachmancolwidth,-0.59)$);}
%         \drawcolsep{0}
%         \drawcolsep{1}
%         \drawcolsep{2}
%         \drawcolsep{3}
%         \drawcolsep{4}
    \end{tikzpicture}
\end{adjustbox}
