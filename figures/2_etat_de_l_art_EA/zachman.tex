% Resources:
% Arrows: http://tex.stackexchange.com/a/60627/32098
% Rotating tikz label: http://tex.stackexchange.com/a/115565/32098

% HACK (and an ugly one) since booktabs breaks vertical separators, we use tikz
% to draw them. Alignment is pretty much custom. I just hope this is gonna work
% with no adjusment when putting this in the main document.
\setlength{\mytablewidth}{\textwidth}
\setlength{\myfirstcolwidth}{\dimexpr0.2\mytablewidth-2\tabcolsep\relax}
\setlength{\mycolwidth}{\dimexpr0.17\mytablewidth-2\tabcolsep\relax}
\newcommand\mycell[1]{{\tiny{#1}}}

\begin{adjustbox}{width=\mytablewidth,center}
    \scriptsize
    \noindent\begin{tabulary}{\mytablewidth}{m{\myfirstcolwidth}m{\mycolwidth}m{\mycolwidth}m{\mycolwidth}m{\mycolwidth}m{\mycolwidth}}

        % \cmidrule[\heavyrulewidth]{2-6}
        \multirow{2}{*}{}\tikzmark{zachmantopleft} \
        & \centbf{Données} \
        & \centbf{Fonctions} \
        & \centbf{Personnel} \
        & \centbf{Temps} \
        & \centbf{Motivation}\tikzmark{zachmantopright} \
        \tabularnewline
        & \centit{Quoi} \
        & \centit{Comment} \
        & \centit{Quand} \
        & \centit{Qui} \
        & \centit{Pourquoi} \
        \tabularnewline\midrule

        \tikzmark{zachmanlefttop}\textbf{Portée}\newline\textit{Planificateur} \
        & {\tiny{Liste de choses importantes pour l'entreprise}} \
        & {\tiny{Processus realises par l'entreprise}} \
        & {\tiny{List des unites organisationnelles}} \
        & {\tiny{Cycles d'affaires}} \
        & {\tiny{Strategies/objectifs de l'entreprise}} \
        \tabularnewline\midrule

        \textbf{Modèle métier}\newline\textit{Propriétaire} \
        & \mycell{Modèle semantique \color{red}{ou Diagramme des relations d'entité?}}  \
        & \mycell{Modèle de processus d'activité (Diagramme du Flux de Données physiques)}  \
        & \mycell{\color{red}{workflow model (Organigramme, avec des rôles ; sous-ensembles de compétence ; questions de sécurité.)}} \
        & \mycell{\color{red}{master schedule}} \
        & \mycell{Business plan} \
        \tabularnewline\midrule

        \textbf{Modèle système}\newline\textit{Concepteur}  \
        & \mycell{Modèle de données (entités fusionnées, entièrement normalisées)}  \
        & \mycell{Architecture d'application}  \
        & \mycell{Architecture d'interface humaine (rôles, données, accès)} \
        & \mycell{Diagramme de dépendance, histoire de la vie de l'entité (structure de processus)} \
        & \mycell{Modèle de principe économique} \
        \tabularnewline\midrule

        \textbf{Modèle}\newline\textbf{Technologique}\newline\textit{Réalisateur} \
        & \mycell{Architecture de données (tables et colonnes) ; carte des données légales}  \
        & \mycell{Conception de système : diagramme de structure, pseudo-code}  \
        & \mycell{Interface utilisateur (comment le système se comportera) ; conception de la sécurité} \
        & \mycell{Diagramme «~flux de contrôle~» (structure de contrôle)} \
        & \mycell{Conception du principe économique} \
        \tabularnewline\midrule

        \tikzmark{zachmanleftbottom}\textbf{Modèle détaillé}\newline\textit{Sous-traitant} \
        & \mycell{Conception de données (dénormalisé), conception physique de stockage}  \
        & \mycell{Conception détaillée de programme}  \
        & \mycell{Écrans, architecture de sécurité (qui peut voir quoi ?)} \
        & \mycell{Définitions de synchronisation} \
        & \mycell{Spécifications de règle dans la logique de programme} \
        \tabularnewline\bottomrule
    \end{tabulary}
    \begin{tikzpicture}[overlay,remember picture]

        % top horizontal arrow
        \draw[<->] let \p1=(zachmantopleft), \p2=(zachmantopright) in ($(\x1,\y1)+(1.6,0.4)$) -- node[label=Abstractions (colonnes)] {} ($(\x2,\y2)+(0.4,0.4)$);

        % left vertical arrow
        \draw[<->] let \p3=(zachmanlefttop), \p4=(zachmanleftbottom) in ($(\x3,\y3)+(-0.3,0.3)$) -- node[label={[label distance=-2ex, text depth=3ex, label position=above, rotate=90]above:Perspectives (lignes)}] {} ($(\x3,\y4)+(-0.3,-0.45)$);

%         % HACK: draw vertical column separators
%         \def\zachmancolwidth{2.41}  % ~ column width (found manually)
%         \def\zachmanleftoffset{3.55}  % ~ first column offset (found manually)
%         \newcommand\drawcolsep[1]{%
%             \draw[dotted] let \p3=(zachmanlefttop), \p4=(zachmanleftbottom) in ($(\x3,\y3)+(\zachmanleftoffset+#1*\zachmancolwidth,0.24)$) -- ($(\x3,\y4)+(\zachmanleftoffset+#1*\zachmancolwidth,-0.59)$);}
%         \drawcolsep{0}
%         \drawcolsep{1}
%         \drawcolsep{2}
%         \drawcolsep{3}
%         \drawcolsep{4}
    \end{tikzpicture}
\end{adjustbox}
