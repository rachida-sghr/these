% Resources:
% Arrows: http://tex.stackexchange.com/a/60627/32098
% Rotating tikz label: http://tex.stackexchange.com/a/115565/32098

% HACK (and an ugly one) since booktabs breaks vertical separators, we use tikz
% to draw them. Alignment is pretty much custom. I just hope this is gonna work
% with no adjusment when putting this in the main document.
\setlength{\mytablewidth}{\textwidth}
\setlength{\myfirstcolwidth}{\dimexpr0.25\mytablewidth-2\tabcolsep\relax}
\setlength{\mycolwidth}{\dimexpr0.16\mytablewidth-2\tabcolsep\relax}

\begin{adjustbox}{width=\mytablewidth,center}
    \scriptsize
    \begin{tabulary}{\mytablewidth}{m{\myfirstcolwidth}m{\mycolwidth}m{\mycolwidth}m{\mycolwidth}m{\mycolwidth}m{\mycolwidth}}

        \cmidrule[\heavyrulewidth]{2-6}
        \mrows{2}\tikzmark{zachmantopleft} \
        & \centbf{Donnees} \
        & \centbf{Fonctions} \
        & \centbf{Personnel} \
        & \centbf{Temps} \
        & \centbf{Motivation}\tikzmark{zachmantopright} \
        \tabularnewline
        & \centit{Quoi} \
        & \centit{Comment} \
        & \centit{Qui} \
        & \centit{Quand} \
        & \centit{Pourquoi} \
        \tabularnewline\midrule

        \tikzmark{zachmanlefttop}\textbf{Portee} \
        & \mrows{2} \
        & \mrows{2} \
        & \mrows{2} \
        & \mrows{2} \
        & \mrows{2} \
        \tabularnewline

        \textit{Planner} &&&&& \
        \tabularnewline\midrule

        \textbf{Modele metier} \
        & \mrows{2} \
        & \mrows{2} \
        & \mrows{2} \
        & \mrows{2} \
        & \mrows{2} \
        \tabularnewline
        % \textering(Conceptual) &&&&&& \
        % \tabularnewline
        \textit{Proprietaire} &&&&& \
        \tabularnewline\midrule

        \textbf{Modele systeme} \
        & \mrows{2} \
        & \mrows{2} \
        & \mrows{2} \
        & \mrows{2} \
        & \mrows{2} \
        \tabularnewline
        \textit{Concepteur} &&&&& \
        \tabularnewline\midrule

        \textbf{Modele} \
        & \mrows{2} \
        & \mrows{2} \
        & \mrows{2} \
        & \mrows{2} \
        & \mrows{2} \
        \tabularnewline
        \textbf{Technologique} \
        & \mrows{2} \
        & \mrows{2} \
        & \mrows{2} \
        & \mrows{2} \
        & \mrows{2} \
        \tabularnewline
        \textit{Realisateur} &&&&& \
        \tabularnewline\midrule

        \tikzmark{zachmanleftbottom}\textbf{Modele detaille} \
        & \mrows{2} \
        & \mrows{2} \
        & \mrows{2} \
        & \mrows{2} \
        & \mrows{2} \
        \tabularnewline
        % \textering(Out of context) &&&&&& \
        % \tabularnewline
        \textit{Sous-traitant} &&&&& \
        \tabularnewline\bottomrule
    \end{tabulary}
    \begin{tikzpicture}[overlay,remember picture]

        % top horizontal arrow
        \draw[<->] let \p1=(zachmantopleft), \p2=(zachmantopright) in ($(\x1,\y1)+(1.5,0.5)$) -- node[label=Abstractions (colonnes)] {} ($(\x2,\y2)+(0,0.5)$);

        % left vertical arrow
        \draw[<->] let \p3=(zachmanlefttop), \p4=(zachmanleftbottom) in ($(\x3,\y3)+(-0.3,0.2)$) -- node[label={[label distance=-2ex, text depth=3ex, label position=above, rotate=90]above:Perspectives (lignes)}] {} ($(\x3,\y4)+(-0.3,-0.5)$);

        % HACK: draw vertical column separators
        \def\zachmancolwidth{2}  % ~ column width (found manually)
        \def\zachmanleftoffset{3}  % ~ first column offset (found manually)
        \newcommand\drawcolsep[1]{%
            \draw[dotted] let \p3=(zachmanlefttop), \p4=(zachmanleftbottom) in ($(\x3,\y3)+(\zachmanleftoffset+#1*\zachmancolwidth,0.2)$) -- ($(\x3,\y4)+(\zachmanleftoffset+#1*\zachmancolwidth,-0.64)$);}
        \drawcolsep{0}
        \drawcolsep{1}
        \drawcolsep{2}
        \drawcolsep{3}
        \drawcolsep{4}
    \end{tikzpicture}
\end{adjustbox}
